\documentclass[11pt]{article}
\usepackage[utf8]{inputenc}
\usepackage[T1]{fontenc}
\usepackage{amsmath}
\usepackage{multicol}
\usepackage{geometry}
\usepackage{tikz}
\usetikzlibrary{shapes.geometric, arrows.meta, calc}
\usepackage{enumitem}
\usepackage{xcolor}
\usepackage{titlesec}
\usepackage{tcolorbox}

% Configurações de layout
\geometry{a4paper, left=1cm, right=1cm, top=0.5cm, bottom=1.2cm}
\setlength{\columnseprule}{0.4pt}
\setlength{\baselineskip}{1.0\baselineskip}

% Cores personalizadas
\definecolor{titleblue}{RGB}{0,80,150}
\definecolor{sectionred}{RGB}{180,0,0}
\definecolor{darkgreen}{RGB}{0,100,0}
\definecolor{explanationbg}{RGB}{240,248,255}

% Formatação de títulos
\titleformat{\section}{\normalfont\Large\bfseries\color{titleblue}}{\thesection}{1em}{}
\titleformat{\subsection}{\normalfont\large\bfseries\color{sectionred}}{\thesubsection}{1em}{}
\titleformat{\subsubsection}{\normalfont\normalsize\bfseries\color{darkgreen}}{\thesubsubsection}{1em}{}

\title{\textcolor{titleblue}{Atividade Avaliativa 2: Desigualdade Social}}
\author{Professor: Jefferson}
\date{}

\begin{document}

\maketitle
\vspace{-1cm}

\begin{center}
    \large{\textbf{Observação:} Respostas no caderno com letra legível. \quad Série: 3 Ano. Valor: 1,0}
\end{center}

\begin{multicols}{2}

\section*{Atividade}
\begin{enumerate}

\item \textbf{Conceito Básico}\\
Qual a diferença entre desigualdade social e pobreza?
\begin{tcolorbox}[colback=explanationbg,colframe=titleblue,title=Dica:]
Pense na distribuição de recursos vs. falta absoluta de recursos.
\end{tcolorbox}

\item \textbf{Indicadores}\\
Por que o Coeficiente de Gini varia entre 0 e 1? O que cada extremo representa?
\begin{tcolorbox}[colback=explanationbg,colframe=titleblue,title=Dica:]
0 = igualdade perfeita; 1 = desigualdade máxima.
\end{tcolorbox}

\item \textbf{Educação}\\
Como a desigualdade no acesso à educação de qualidade perpetua ciclos de pobreza?
\begin{tcolorbox}[colback=explanationbg,colframe=titleblue,title=Dica:]
Considere oportunidades de emprego e mobilidade social.
\end{tcolorbox}

\item \textbf{Gênero}\\
Por que mulheres geralmente recebem salários menores que homens para a mesma função?
\begin{tcolorbox}[colback=explanationbg,colframe=titleblue,title=Dica:]
Analise preconceitos estruturais e divisão sexual do trabalho.
\end{tcolorbox}

\item \textbf{Raça/Etnia}\\
No Brasil, por que a população negra tem menores índices de desenvolvimento humano?
\begin{tcolorbox}[colback=explanationbg,colframe=titleblue,title=Dica:]
Histórico escravocrata e desigualdades persistentes.
\end{tcolorbox}

\item \textbf{Regional}\\
Explique as diferenças de desenvolvimento entre Sudeste e Nordeste brasileiros.
\begin{tcolorbox}[colback=explanationbg,colframe=titleblue,title=Dica:]
Distribuição histórica de investimentos e infraestrutura.
\end{tcolorbox}

\item \textbf{Tecnologia}\\
Como o acesso desigual à internet amplia outras formas de desigualdade?
\begin{tcolorbox}[colback=explanationbg,colframe=titleblue,title=Dica:]
Educação remota, oportunidades de trabalho digital.
\end{tcolorbox}

\item \textbf{Políticas Públicas}\\
Bolsa Família e cotas raciais são exemplos de que tipo de política redistributiva?
\begin{tcolorbox}[colback=explanationbg,colframe=titleblue,title=Dica:]
Ações afirmativas para reduzir desigualdades.
\end{tcolorbox}

\item \textbf{Mobilidade Urbana}\\
Como a falta de transporte público de qualidade afeta os mais pobres?
\begin{tcolorbox}[colback=explanationbg,colframe=titleblue,title=Dica:]
Acesso a empregos, saúde e educação.
\end{tcolorbox}

\item \textbf{Trabalho Infantil}\\
Por que famílias pobres podem recorrer ao trabalho infantil, perpetuando a desigualdade?
\begin{tcolorbox}[colback=explanationbg,colframe=titleblue,title=Dica:]
Necessidade imediata de renda vs. educação.
\end{tcolorbox}

\end{enumerate}
\end{multicols}

\end{document}
