\documentclass[12pt]{article}
\usepackage[utf8]{inputenc}
\usepackage{amsmath}
\usepackage{multicol}
\usepackage{geometry}
\usepackage{tikz}
\usetikzlibrary{arrows.meta}
\usepackage{enumitem}
\usepackage{xcolor}
\usepackage{titlesec}

% Configurações de layout
\geometry{a4paper, left=1cm, right=1cm, top=0.5cm, bottom=1.2cm}
\setlength{\columnseprule}{0.4pt}
\setlength{\baselineskip}{1.0\baselineskip}

% Definir a cor das seções como azul
\titleformat{\section}
  {\normalfont\Large\bfseries\color{blue}}
  {\thesection}
  {1em}
  {}

\renewcommand{\thesubsection}{\textcolor{red}{\arabic{section}.\arabic{subsection}}}
\titleformat{\subsection}{\color{red}\normalfont\bfseries}{\thesubsection}{1em}{}
\title{\textcolor{blue}{Atividade: Ética Tecnológica, Desigualdade Social e Identidade Digital}}
\author{Professor: \underline{\hspace{4cm}}}
\date{}

\begin{document}

\maketitle

\begin{center}
\large{Nome: \underline{\hspace{8cm}} \quad Série-Turma: \underline{\hspace{3cm}}}
\end{center}

\begin{multicols}{2}

\section*{Instruções}
Responda às questões abaixo com base no filme \textit{Alita: Anjo de Combate} (2019) e em reflexões sobre a sociedade contemporânea.

\section*{Parte 1: Ética Tecnológica}
\begin{enumerate}
    \item No filme, Dr. Ido e Nova representam visões opostas sobre o uso da tecnologia. Compare suas posturas éticas.
    \item A tecnologia de Alita (corpo cibernético) pode ser usada para o bem ou para o mal. Dê exemplos de ambas as situações no filme.
    \item Quais são os limites éticos da manipulação genética e da inteligência artificial apresentados no filme?
    \item A ciborguização em Alita é voluntária ou imposta? Discuta os problemas éticos disso.
    \item Como o filme retrata a responsabilidade dos cientistas (como Ido) sobre suas criações?
    \item A tecnologia no filme é neutra ou seus impactos dependem de quem a controla? Justifique.
    \item O que o filme sugere sobre o direito à privacidade em um mundo com corpos hackeáveis?
    \item Relacione o uso de ciborgues no filme com os debates atuais sobre aprimoramento humano.
    \item Qual é o papel da consciência na definição do que é humano no filme?
    \item Como o filme aborda o dilema ético de criar vida artificial com sentimentos?
\end{enumerate}

\section*{Parte 2: Desigualdade Social}
\begin{enumerate}[resume]
    \item Compare as condições de vida em Iron City e Zalem. Que paralelos podemos fazer com a sociedade atual?
    \item Como o acesso à tecnologia aprofunda as desigualdades no filme?
    \item O esporte Motorball serve para que tipo de mobilidade social no filme?
    \item Que grupos sociais são mais vulneráveis em Iron City e por quê?
    \item Como o filme retrata a exploração dos corpos cibernéticos para benefício dos mais ricos?
    \item Que mecanismos de controle social são usados para manter a hierarquia entre Zalem e Iron City?
    \item O que a figura de Vector representa na manutenção das desigualdades?
    \item Como a violência é usada como ferramenta de opressão no filme?
    \item Que soluções o filme sugere para reduzir as desigualdades tecnológicas?
    \item Como a falta de acesso à tecnologia médica afeta os personagens mais pobres?
\end{enumerate}

\section*{Parte 3: Identidade Digital}
\begin{enumerate}[resume]
    \item O que define a identidade de Alita: suas memórias, seu corpo ou suas ações?
    \item Como a perda de memórias afeta a construção da identidade no filme?
    \item Compare a identidade de Alita antes e depois de receber o corpo Berserker.
    \item Que conflitos internos Alita enfrenta ao descobrir seu passado?
    \item Como as redes sociais atuais influenciam nossa identidade digital?
    \item O filme sugere que nossa identidade pode ser armazenada digitalmente. Isso é positivo ou perigoso?
    \item Como o filme aborda o direito ao esquecimento digital?
    \item Que paralelos podemos fazer entre a busca de Alita por sua identidade e os desafios da identidade online hoje?
    \item Como o corpo cibernético de Alita desafia noções tradicionais de identidade?
    \item O filme sugere que a identidade é fixa ou em constante transformação?
\end{enumerate}

\end{multicols}

\end{document}
