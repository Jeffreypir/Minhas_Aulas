\documentclass[12pt]{article}
\usepackage[utf8]{inputenc}
\usepackage{amsmath}
\usepackage{multicol}
\usepackage{geometry}
\usepackage{tikz}
\usetikzlibrary{arrows.meta}
\usepackage{enumitem}
\usepackage{xcolor}
\usepackage{titlesec}

% Configurações de layout
\geometry{a4paper, left=1cm, right=1cm, top=0.5cm, bottom=1.2cm}
\setlength{\columnseprule}{0.4pt}
\setlength{\baselineskip}{1.0\baselineskip}

% Definir a cor das seções como azul
\titleformat{\section}
  {\normalfont\Large\bfseries\color{blue}}
  {\thesection}
  {1em}
  {}

\renewcommand{\thesubsection}{\textcolor{red}{\arabic{section}.\arabic{subsection}}}
\titleformat{\subsection}{\color{red}\normalfont\bfseries}{\thesubsection}{1em}{}
\title{\textcolor{blue}{Gabarito: Ética Tecnológica, Desigualdade Social e Identidade Digital}}
\author{Professor: \underline{\hspace{4cm}}}
\date{}

\begin{document}

\maketitle

\begin{multicols}{2}

\section*{Parte 1: Ética Tecnológica}
\begin{enumerate}
    \item \textbf{Dr. Ido} usa a tecnologia para curar e ajudar, enquanto \textbf{Nova} a usa para controle e manipulação, representando os dilemas éticos entre benefício coletivo vs. poder.
    
    \item \textbf{Bem}: Alita salva vidas e protege os fracos. \textbf{Mal}: Ciborgues são usados para violência e caça às recompensas.
    
    \item Limites ultrapassados incluem: criação de armas autônomas, manipulação de memórias e experimentação em humanos sem consentimento.
    
    \item Mista: alguns optam por melhorias, outros são forçados (como caçadores de recompensas). Problema: autonomia corporal e trabalho análogo à escravidão.
    
    \item Ido demonstra remorso por criar o corpo Berserker, enquanto Nova não assume responsabilidade por suas ações.
    
    \item Depende do controle: a mesma tecnologia que cura (Ido) também mata (Nova), mostrando que não é neutra.
    
    \item Sugere riscos extremos: corpos podem ser controlados remotamente e memórias apagadas ou alteradas.
    
    \item Assim como atualmente debatemos próteses neurais, o filme mostra os riscos do aprimoramento descontrolado.
    
    \item A consciência e capacidade de escolha (como Alita proteger os outros) definem humanidade, não o corpo biológico.
    
    \item Aborda através do conflito de Alita: mesmo sendo artificial, ela desenvolve empatia e moralidade genuínas.
\end{enumerate}

\section*{Parte 2: Desigualdade Social}
\begin{enumerate}[resume]
    \item \textbf{Iron City} é marginalizada, enquanto \textbf{Zalem} tem tecnologia avançada. Paralelo: divisão digital entre países ricos e pobres.
    
    \item A elite de Zalem monopoliza tecnologia avançada, enquanto Iron City sobrevive com restos tecnológicos.
    
    \item É uma saída precária da pobreza, mas mantém o sistema desigual (como atletas profissionais em favelas).
    
    \item Trabalhadores não-ciborgues e ciborgues com peças obsoletas, por dependerem dos descartes de Zalem.
    
    \item Ciborgues pobres vendem partes do corpo para sobreviver, enquanto os ricos têm corpos de última geração.
    
    \item Promessas de ascensão a Zalem (falsa esperança) e repressão violenta através dos caçadores de recompensas.
    
    \item Intermediário corrupto que mantém o status quo em troca de privilégios pessoais.
    
    \item Através dos caçadores de recompensas e do Motorball, que divertem a população enquanto a oprimem.
    
    \item Revolução tecnológica (como Alita desafiar Zalem) e acesso democrático às inovações.
    
    \item Personagens como Hugo dependem do mercado negro para obter peças cibernéticas básicas.
\end{enumerate}

\section*{Parte 3: Identidade Digital}
\begin{enumerate}[resume]
    \item Principalmente suas \textbf{ações} e escolhas, já que suas memórias foram apagadas.
    
    \item Cria uma identidade fragmentada, levando-a a questionar se suas memórias definem quem ela é.
    
    \item Antes: frágil e dependente. Depois: recupera autonomia e propósito militar original.
    
    \item Entre sua programação original como arma e seu novo senso ético desenvolvido com Ido.
    
    \item Assim como Alita, construímos identidades online que podem divergir do "eu" físico.
    
    \item \textbf{Perigoso}: pode ser manipulada (como as memórias de Alita foram apagadas por Nova).
    
    \item Através do direito de Alita reconstruir sua história apesar da manipulação de suas memórias.
    
    \item Ambos envolvem: construção de múltiplas identidades, perda de privacidade e busca por autenticidade.
    
    \item Desafia ao mostrar que identidade não está vinculada a um corpo biológico estável.
    
    \item \textbf{Transformação}: ela se redefine constantemente através de novas experiências e memórias.
\end{enumerate}

\end{multicols}

\begin{center}
\textbf{Observação}: Respostas são sugestões baseadas na análise do filme. Aceitar variações argumentadas com exemplos válidos.
\end{center}

\end{document}
