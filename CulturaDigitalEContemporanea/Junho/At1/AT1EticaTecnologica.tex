\documentclass[11pt]{article}
\usepackage[utf8]{inputenc}
\usepackage[T1]{fontenc}
\usepackage{amsmath}
\usepackage{multicol}
\usepackage{geometry}
\usepackage{tikz}
\usetikzlibrary{shapes.geometric, arrows.meta, calc}
\usepackage{enumitem}
\usepackage{xcolor}
\usepackage{titlesec}
\usepackage{tcolorbox}

% Configurações de layout
\geometry{a4paper, left=1cm, right=1cm, top=0.5cm, bottom=1.2cm}
\setlength{\columnseprule}{0.4pt}
\setlength{\baselineskip}{1.0\baselineskip}

% Cores personalizadas
\definecolor{titleblue}{RGB}{0,80,150}
\definecolor{sectionred}{RGB}{180,0,0}
\definecolor{darkgreen}{RGB}{0,100,0}
\definecolor{explanationbg}{RGB}{240,248,255}

% Formatação de títulos
\titleformat{\section}{\normalfont\Large\bfseries\color{titleblue}}{\thesection}{1em}{}
\titleformat{\subsection}{\normalfont\large\bfseries\color{sectionred}}{\thesubsection}{1em}{}
\titleformat{\subsubsection}{\normalfont\normalsize\bfseries\color{darkgreen}}{\thesubsubsection}{1em}{}

\title{\textcolor{titleblue}{Atividade Avaliativa 1: Cultura Digital e Contemporânea \\Assunto: Ética Tecnológica}}
\author{Professor: Jefferson}
\date{}

\begin{document}

\maketitle
\vspace{-1cm}

\begin{center}
    \large{\textbf{Observação:} Respostas no caderno com letra legível. \quad Série: 3 Ano. Valor: 1,0}
\end{center}

\begin{multicols}{2}

\section*{Atividade}
\begin{enumerate}

\item \textbf{Privacidade de Dados}\\
Por que empresas devem obter consentimento explícito antes de coletar dados pessoais?
\begin{tcolorbox}[colback=explanationbg,colframe=titleblue,title=Dica:]
Pense nos princípios da LGPD (Lei Geral de Proteção de Dados).
\end{tcolorbox}

\item \textbf{Viés Algorítmico}\\
Um sistema de recrutamento automatizado rejeitou currículos com nomes nordestinos. Que princípio ético foi violado?
\begin{tcolorbox}[colback=explanationbg,colframe=titleblue,title=Dica:]
Analise discriminação algorítmica e justiça social.
\end{tcolorbox}

\item \textbf{Responsabilidade Tecnológica}\\
Quem é responsável quando um carro autônomo causa um acidente: fabricante, programador ou dono?
\begin{tcolorbox}[colback=explanationbg,colframe=titleblue,title=Dica:]
Considere a cadeia de desenvolvimento de IA.
\end{tcolorbox}

\item \textbf{Propriedade Intelectual}\\
É ético baixar livros piratas de autores vivos? Justifique.
\begin{tcolorbox}[colback=explanationbg,colframe=titleblue,title=Dica:]
Pense nos direitos do autor e acesso ao conhecimento.
\end{tcolorbox}

\item \textbf{Dilema das Redes Sociais}\\
"Se o serviço é gratuito, você é o produto". O que essa frase revela sobre ética tecnológica?
\begin{tcolorbox}[colback=explanationbg,colframe=titleblue,title=Dica:]
Analise modelos de negócios baseados em dados.
\end{tcolorbox}

\item \textbf{Deepfakes}\\
Quais os riscos éticos da tecnologia de vídeos falsos hiper-realistas?
\begin{tcolorbox}[colback=explanationbg,colframe=titleblue,title=Dica:]
Considere desinformação e danos à reputação.
\end{tcolorbox}

\item \textbf{Ética em IA}\\
Por que chatbots como ChatGPT precisam ter filtros contra discurso de ódio?
\begin{tcolorbox}[colback=explanationbg,colframe=titleblue,title=Dica:]
Pense na replicação de preconceitos pela IA.
\end{tcolorbox}

\item \textbf{Tecnologia Viciante}\\
Quais práticas antiéticas empresas usam para maximizar tempo de tela dos usuários?
\begin{tcolorbox}[colback=explanationbg,colframe=titleblue,title=Dica:]
Ex.: rolagem infinita, notificações constantes.
\end{tcolorbox}

\item \textbf{Monitoramento Digital}\\
É ético empresas monitorarem mensagens de funcionários em home office? Debate prós e contras.
\begin{tcolorbox}[colback=explanationbg,colframe=titleblue,title=Dica:]
Balanceie produtividade x privacidade.
\end{tcolorbox}

\item \textbf{Lixo Eletrônico}\\
Qual a responsabilidade ética de fabricantes sobre descarte de dispositivos obsoletos?
\begin{tcolorbox}[colback=explanationbg,colframe=titleblue,title=Dica:]
Considere obsolescência programada e impacto ambiental.
\end{tcolorbox}

\end{enumerate}
\end{multicols}

\end{document}
