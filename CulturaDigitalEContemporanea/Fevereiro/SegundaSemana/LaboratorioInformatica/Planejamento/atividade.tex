\documentclass{beamer}

\usepackage[utf8]{inputenc}
\usepackage[brazilian]{babel}
\usepackage{graphicx}
\usepackage{hyperref}
\usepackage{verbatim} % Para código-fonte

\usetheme{Madrid}
\usecolortheme{default}

\title{Atividade no Laboratório de Informática}
\subtitle{Redes Sociais, Marketing Digital, Programação e Robótica}
\author{Jefferson Bezerra dos Santos }
\date{}

\begin{document}

\begin{frame}
    \titlepage
\end{frame}

\begin{frame}{Sumário}
    \tableofcontents
\end{frame}

\section{Resumo}
\begin{frame}{Resumo}
    \begin{itemize}
        \item Exploração das redes sociais e sua influência no marketing digital.
        \item Introdução à programação e robótica como ferramentas para inovação.
        \item Análise das interações entre cultura popular, cultura digital e contemporaneidade.
        \item Reconhecimento das transformações culturais provocadas pelas tecnologias digitais.
        \item Realização de pesquisas sobre manifestações culturais populares e contemporâneas.
    \end{itemize}
\end{frame}

\section{Introdução}
\begin{frame}{Introdução}
    \begin{itemize}
        \item As redes sociais são plataformas poderosas para comunicação e marketing.
        \item O marketing digital utiliza estratégias online para influenciar públicos.
        \item A programação e a robótica promovem inovações tecnológicas e culturais.
        \item Esta atividade integra esses conceitos em uma experiência prática.
    \end{itemize}
\end{frame}

\section{Metodologia}
\begin{frame}{Metodologia}
    \begin{itemize}
        \item \textbf{Redes Sociais e Marketing Digital:}
        \begin{itemize}
            \item Uso de \textbf{Facebook Ads} e \textbf{Instagram Insights} para análise de engajamento.
            \item Simulação de campanhas utilizando \textbf{Canva} e \textbf{Google Trends}.
        \end{itemize}
        \item \textbf{Programação:}
            \begin{itemize}
                \item A programação é uma habilidade fundamental para criar soluções tecnológicas.
                \item O Python é uma linguagem de fácil aprendizagem, ideal para iniciantes.
                \item Neste exemplo, o código solicita ao usuário que digite seu nome e exibe uma mensagem personalizada.
                \item Vamos explorar mais exemplos práticos de como utilizar Python para resolver problemas simples.
            \end{itemize}
        \end{itemize}
\end{frame}

\begin{frame}[fragile]{Exemplo em Python}
\begin{verbatim}
print("Olá, mundo!")
nome = input("Digite seu nome: ")
print(f"Bem-vindo, {nome}!")
\end{verbatim}
\end{frame}

\begin{frame}{Metodologia (continuação)}
    \begin{itemize}
        \item \textbf{Robótica:}
        \begin{itemize}
            \item O Arduino é uma plataforma de prototipagem eletrônica popular para iniciantes e profissionais.
            \item O código a seguir acende e apaga um LED a cada segundo, utilizando um microcontrolador Arduino.
            \item Este tipo de automação simples pode ser expandido para sistemas mais complexos com sensores e atuadores.
        \end{itemize}
    \end{itemize}
\end{frame}

\begin{frame}[fragile]{Exemplo em Arduino}
\begin{verbatim}
void setup() {
    pinMode(13, OUTPUT);
}

void loop() {
    digitalWrite(13, HIGH);
    delay(1000);
    digitalWrite(13, LOW);
    delay(1000);
}
\end{verbatim}
\end{frame}

\begin{frame}{Robótica (continuação)}
    \begin{itemize}
        \item O código acima é utilizado para controlar um LED conectado ao pino 13 do Arduino.
        \item O método \texttt{digitalWrite} define o estado do pino como alto (ligado) ou baixo (desligado).
        \item O método \texttt{delay} faz o Arduino esperar por um tempo específico (em milissegundos) antes de executar a próxima ação.
        \item Este é o primeiro passo para trabalhar com automação e controle no Arduino.
    \end{itemize}
\end{frame}

\section{Objetivos}
\begin{frame}{Objetivos}
    \begin{itemize}
        \item \textbf{Geral:} Integrar redes sociais, marketing digital, programação e robótica em uma experiência prática.
        \item \textbf{Específicos:}
        \begin{itemize}
            \item Analisar as interações entre cultura popular e digital.
            \item Aplicar conceitos de marketing digital em projetos práticos.
            \item Desenvolver habilidades básicas de programação e robótica.
            \item Promover o trabalho em equipe e a criatividade.
        \end{itemize}
    \end{itemize}
\end{frame}

\section{Referências}
\begin{frame}{Referências}
    \begin{itemize}
        \item KOTLER, P. \textit{Marketing 4.0}. Editora Sextante, 2017.
        \item SILVA, M. \textit{Introdução à Programação com Python}. Novatec, 2019.
        \item Arduino. \textit{Official Arduino Website}. Disponível em: \url{https://www.arduino.cc}.
        \item BRASIL. Ministério da Educação. \textit{Base Nacional Comum Curricular (BNCC)}. Disponível em: \url{http://basenacionalcomum.mec.gov.br}.
    \end{itemize}
\end{frame}

\end{document}

