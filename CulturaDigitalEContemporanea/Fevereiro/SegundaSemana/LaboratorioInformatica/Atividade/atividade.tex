\documentclass[a4paper,12pt]{article}
\usepackage[left=2cm, right=2cm, top=1.5cm, bottom=2.5cm]{geometry}
\usepackage{graphicx}
\usepackage[brazil]{babel}
\usepackage[utf8]{inputenc}
\usepackage[T1]{fontenc}

\title{Planejamento Bimestral: \\ Marketing Digital, Canvas e PowerPoint}
\author{}
\date{}

\begin{document}

\maketitle

\section*{Habilidades Específicas (Baseadas na BNCC)}
\begin{itemize}
    \item (EM13CHS401) Analisar e interpretar diferentes estratégias de marketing digital e sua aplicação no mercado.
    \item (EM13CHS502) Aplicar conceitos de planejamento estratégico utilizando o Canvas.
    \item (EM13LP02) Produzir apresentações multimídia eficazes para comunicar ideias e projetos.
\end{itemize}

\section*{Objetos de Conhecimento}
\begin{itemize}
    \item Fundamentos do Marketing Digital (SEO, redes sociais, tráfego pago e orgânico).
    \item Modelo de Negócios Canvas (componentes e aplicação prática).
    \item Uso do PowerPoint para criar apresentações profissionais.
\end{itemize}

\section*{Procedimentos Metodológicos}
\begin{enumerate}
    \item Aula Expositiva: Introdução ao marketing digital e sua importância.
    \item Estudo de Caso: Análise de campanhas de marketing digital bem-sucedidas.
    \item Atividade Prática: Criar um modelo Canvas para um projeto fictício ou real.
    \item Workshop: Uso avançado do PowerPoint para apresentações persuasivas.
    \item Desafio Prático: Criar e apresentar um pitch de um produto ou serviço usando o Canvas e PowerPoint.
\end{enumerate}

\section*{Procedimentos Avaliativos}
\begin{itemize}
    \item Produção e análise de um modelo Canvas.
    \item Desenvolvimento de uma apresentação de marketing digital no PowerPoint.
    \item Avaliação individual e em grupo do pitch apresentado.
\end{itemize}

\section*{Atividade: Criando um Plano de Marketing Digital com Canvas e PowerPoint}

\subsection*{Objetivo}
Os alunos deverão criar um plano de marketing digital para um produto ou serviço fictício, estruturá-lo no modelo Canvas e apresentar a proposta usando PowerPoint.

\subsection*{Instruções}
\begin{enumerate}
    \item \textbf{Escolha um produto ou serviço} – Pode ser algo fictício ou real.
    \item \textbf{Preencha o Canvas} – Utilize os seguintes blocos:
    \begin{itemize}
        \item Proposta de Valor
        \item Segmento de Clientes
        \item Canais de Distribuição
        \item Relacionamento com Clientes
        \item Fontes de Receita
        \item Recursos Principais
        \item Atividades-Chave
        \item Parcerias Principais
        \item Estrutura de Custos
    \end{itemize}
    \item \textbf{Desenvolva um plano de marketing digital} – Inclua estratégias de redes sociais, SEO, anúncios pagos e conteúdos interativos.
    \item \textbf{Crie uma apresentação no PowerPoint} – A apresentação deve ter no mínimo 5 slides, contendo:
    \begin{itemize}
        \item Capa com o nome do projeto
        \item Objetivo do produto/serviço
        \item Modelo Canvas
        \item Estratégias de Marketing Digital
        \item Considerações finais
    \end{itemize}
\end{enumerate}

\subsection*{Critérios de Avaliação}
\begin{itemize}
    \item Criatividade e inovação (3 pontos)
    \item Organização do Canvas (3 pontos)
    \item Clareza e objetividade na apresentação (2 pontos)
    \item Uso adequado do PowerPoint (2 pontos)
\end{itemize}

\textbf{Dica:} Utilize imagens, gráficos e elementos visuais para tornar a apresentação mais atraente.

\end{document}

