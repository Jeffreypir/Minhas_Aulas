\documentclass[12pt]{article}
\usepackage[utf8]{inputenc}
\usepackage[brazil]{babel}
\usepackage{enumitem}

\title{Atividade: O Dilema das Redes}
\author{Jefferson}
\date{}

\begin{document}

\maketitle

\section*{Questões Abertas}

\begin{enumerate}[leftmargin=*]
    \item \textbf{Qual é o principal foco do documentário "O Dilema das Redes"?} \\
    \textbf{Resposta:} O documentário foca nos impactos negativos das redes sociais na sociedade, incluindo manipulação de comportamento, problemas de saúde mental, disseminação de desinformação e a exploração de dados pessoais por grandes empresas de tecnologia.

    \item \textbf{Como as redes sociais lucram com os usuários?} \\
    \textbf{Resposta:} As redes sociais lucram principalmente por meio da venda de anúncios personalizados, que são direcionados com base nos dados coletados dos usuários, como interesses, comportamentos e hábitos de navegação.

    \item \textbf{O que são "bolhas de filtro" e como elas afetam os usuários?} \\
    \textbf{Resposta:} Bolhas de filtro são ambientes criados por algoritmos que mostram apenas conteúdos alinhados às crenças e preferências do usuário. Isso limita a exposição a perspectivas diferentes, reforça preconceitos e polariza opiniões.

    \item \textbf{Quais são os efeitos das redes sociais na saúde mental, especialmente entre os jovens?} \\
    \textbf{Resposta:} O documentário destaca o aumento de ansiedade, depressão, baixa autoestima e dependência tecnológica, causados pela comparação social constante, cyberbullying e a busca por validação por meio de curtidas e comentários.

    \item \textbf{Como os algoritmos das redes sociais influenciam o comportamento dos usuários?} \\
    \textbf{Resposta:} Os algoritmos são projetados para manter os usuários engajados pelo maior tempo possível, mostrando conteúdos viciantes e personalizados. Isso pode levar à manipulação de escolhas, opiniões e até comportamentos.

    \item \textbf{Qual é o papel dos dados pessoais no modelo de negócio das redes sociais?} \\
    \textbf{Resposta:} Os dados pessoais são a base do modelo de negócio das redes sociais. Eles são coletados, analisados e vendidos para anunciantes, que os usam para direcionar anúncios de forma extremamente precisa.

    \item \textbf{Como o documentário aborda a questão da desinformação e das fake news?} \\
    \textbf{Resposta:} O documentário mostra que as redes sociais facilitam a disseminação de desinformação e fake news, pois conteúdos sensacionalistas e polêmicos tendem a viralizar mais rapidamente, gerando engajamento e lucro para as plataformas.

    \item \textbf{Quais são as críticas feitas aos ex-funcionários de grandes empresas de tecnologia no documentário?} \\
    \textbf{Resposta:} Os ex-funcionários criticam a falta de ética no desenvolvimento de tecnologias que exploram a psicologia humana para manter os usuários viciados, além da negligência em relação aos impactos negativos na sociedade.

    \item \textbf{Qual é a relação entre redes sociais e polarização política?} \\
    \textbf{Resposta:} As redes sociais contribuem para a polarização política ao criar bolhas de filtro que reforçam visões extremistas e ao promover conteúdos divisivos, que geram mais engajamento e, consequentemente, mais lucro.

    \item \textbf{Que soluções ou alternativas são sugeridas no documentário para mitigar os problemas causados pelas redes sociais?} \\
    \textbf{Resposta:} O documentário sugere maior regulação governamental, transparência das plataformas sobre o uso de dados e algoritmos, conscientização dos usuários e a redução do tempo gasto nas redes sociais.
\end{enumerate}

\end{document}
