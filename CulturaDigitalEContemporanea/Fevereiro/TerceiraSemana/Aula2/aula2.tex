\documentclass[12pt]{article}
\usepackage[utf8]{inputenc}
\usepackage{amsmath}
\usepackage{graphicx}
\usepackage{geometry}
\usepackage{xcolor}
\usepackage{enumitem}
\geometry{a4paper, left=1cm, right=1cm, top=1.5cm, bottom=1.5cm}

\title{Introdução ao Excel - Ensino Médio}
\author{Professor: Jefferson}
\date{}

\begin{document}

\maketitle

\section*{Aula 1: Introdução ao Excel}
\subsection*{Objetivos}
- Compreender a interface do Excel.
- Aprender a inserir e formatar dados em células.

\subsection*{Atividade 1: Inserindo Dados e Fazendo Operações Simples}
\textbf{Passos:}
\begin{itemize}
    \item Abra o Excel e crie uma nova planilha.
    \item Digite os seguintes números na coluna A:
    \begin{center}
        \begin{tabular}{|c|}
        \hline
        10 \\
        20 \\
        30 \\
        40 \\
        \hline
        \end{tabular}
    \end{center}
    \item Na célula B1, digite \texttt{=A1*2} e pressione Enter.
    \item Copie essa fórmula para as outras células abaixo.
    \item Observe os resultados e anote suas conclusões.
\end{itemize}

\section*{Aula 2: Uso de Fórmulas Básicas}
\subsection*{Objetivos}
- Aplicar fórmulas matemáticas básicas.
- Compreender o uso de referências de células.

\subsection*{Atividade 2: Média e Soma de Valores}
\textbf{Passos:}
\begin{itemize}
    \item Insira os seguintes valores na coluna A:
    \begin{center}
        \begin{tabular}{|c|}
        \hline
        5 \\
        15 \\
        25 \\
        35 \\
        \hline
        \end{tabular}
    \end{center}
    \item Na célula B1, digite \texttt{=SUM(A1:A4)} e pressione Enter.
    \item Na célula B2, digite \texttt{=AVERAGE(A1:A4)} e pressione Enter.
    \item Compare os resultados e anote suas observações.
\end{itemize}

\section*{Aula 3: Gráficos no Excel}
\subsection*{Objetivos}
- Criar gráficos a partir de tabelas.
- Analisar visualmente os dados.

\subsection*{Atividade 3: Criando um Gráfico de Colunas}
\textbf{Passos:}
\begin{itemize}
    \item Insira os seguintes dados:
    \begin{center}
        \begin{tabular}{|c|c|}
        \hline
        Produto & Vendas \\
        \hline
        Produto A & 50 \\
        Produto B & 80 \\
        Produto C & 30 \\
        \hline
        \end{tabular}
    \end{center}
    \item Selecione os dados e clique na aba \textbf{Inserir}.
    \item Escolha \textbf{Gráfico de Colunas}.
    \item Analise o gráfico gerado e descreva o que percebe.
\end{itemize}

\section*{Aula 4: Funções Condicionais e Tabelas Dinâmicas}
\subsection*{Objetivos}
- Aplicar funções condicionais para análise de dados.
- Criar tabelas dinâmicas para organização de informações.

\subsection*{Atividade 4: Usando a Função SE}
\textbf{Passos:}
\begin{itemize}
    \item Insira os seguintes dados:
    \begin{center}
        \begin{tabular}{|c|c|}
        \hline
        Nome & Nota \\
        \hline
        João & 75 \\
        Maria & 85 \\
        Pedro & 60 \\
        Ana & 90 \\
        \hline
        \end{tabular}
    \end{center}
    \item Na célula C1, digite \texttt{=SE(B1>=70, "Aprovado", "Reprovado")} e pressione Enter.
    \item Copie a fórmula para as outras células da coluna.
    \item Anote as respostas obtidas e compare os resultados.
\end{itemize}

\end{document}

