\documentclass[12pt]{article}
\usepackage[utf8]{inputenc}
\usepackage[brazil]{babel}
\usepackage{multicol} % Para dividir a página em colunas
\usepackage{geometry} % Para ajustar margens
\geometry{a4paper, left=2cm, right=2cm, top=2cm, bottom=2cm}

\title{Cultura Digital e Cultura Contemporânea}
\author{Professor: Jefferson}
\date{}

\begin{document}

% Remove a numeração da primeira página
\thispagestyle{empty}

\maketitle
% Remove a numeração de todas as páginas
 \pagenumbering{gobble}

\begin{center}
\large{Nome: \underline{\hspace{8cm}} \quad Turma: \underline{\hspace{3cm}}}
\end{center}

\vspace{1cm}

\begin{multicols}{2}


\section*{Letramento Digital e Cultura Contemporânea}

A cultura digital é um fenômeno que transformou profundamente a sociedade contemporânea. Com o advento da internet e das tecnologias digitais, novas formas de comunicação, expressão e interação surgiram, influenciando a cultura popular e a formação identitária das pessoas. A cultura popular, que antes se manifestava principalmente por meio de tradições locais, festas e mídias tradicionais, agora se expande e se reinventa no ambiente digital.

Redes sociais, plataformas de streaming, memes e vídeos virais são exemplos de como a cultura digital se apropria de elementos da cultura popular e os transforma em fenômenos globais. Além disso, a programação e a robótica têm um papel central nesse processo, permitindo a criação de novas formas de arte, entretenimento e comunicação.

No entanto, para navegar nesse universo digital, é essencial desenvolver habilidades de pesquisa e curadoria de informações. Com tanta informação disponível online, saber identificar fontes confiáveis e organizar dados de forma crítica é fundamental para o letramento digital.

\section*{Atividade}

1. \textbf{Análise de Conceitos.}  
   Explique, com suas palavras, o que é cultura digital e como ela se relaciona com a cultura popular. Dê um exemplo de manifestação cultural que surgiu ou se transformou no ambiente digital.

\vspace{1cm}

2. \textbf{Influências da Cultura Digital.}  
   Identifique e descreva duas formas pelas quais a cultura digital influencia a formação identitária e social das pessoas na contemporaneidade.

\vspace{1cm}

3. \textbf{Pesquisa e Curadoria.}  
   Imagine que você precisa pesquisar sobre o impacto das redes sociais na política. Quais critérios você usaria para selecionar fontes confiáveis? Liste pelo menos três critérios e justifique sua escolha.

\vspace{1cm}

4. \textbf{Transformações Culturais.}  
   As tecnologias digitais provocaram diversas transformações culturais. Escolha uma dessas transformações e explique como ela afeta o cotidiano das pessoas.

\vspace{1cm}

5. \textbf{Programação e Robótica.}  
   O que é programação e qual o seu papel na cultura digital? Dê um exemplo de como a programação ou a robótica pode impactar a cultura e a sociedade.

\vspace{1cm}

\end{multicols}

\end{document}
