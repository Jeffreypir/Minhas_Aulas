\documentclass[12pt]{beamer}
\usepackage[utf8]{inputenc}
\usepackage[brazil]{babel}
\usepackage{amsmath} % Para fórmulas matemáticas
\usepackage{xcolor} % Para usar cores
\usepackage{enumitem} % Para listas personalizadas

\usetheme{Berkeley}

% Definir cores para títulos e subtítulos
\setbeamercolor{frametitle}{fg=white}
\setbeamercolor{framesubtitle}{fg=green}
% Adicionar nota de rodapé em todos os slides
\setbeamertemplate{footline}{
    \vspace{0.1cm} % Espaço acima da nota
    \footnotesize % Tamanho do texto
    \hspace{2cm} Mestre em Modelagem Matemática e Computacional - UFPB
}

\title{\textcolor{white}{Pensamento Computacional: Condicionais no Excel}}
\author{Professor: Jefferson}
\date{}

\begin{document}

\frame{\titlepage}

% Slide do sumário
\begin{frame}
    \frametitle{Sumário}
    \tableofcontents
\end{frame}

\section{Introdução ao Pensamento Computacional}

\begin{frame}
    \frametitle{O que é Pensamento Computacional?}
    \framesubtitle{Definição e Componentes}
    O pensamento computacional é uma abordagem para resolver problemas que envolve:
    \begin{itemize}
        \item \textbf{Decomposição:} Dividir problemas complexos em partes menores.
        \item \textbf{Padrões:} Identificar similaridades entre problemas.
        \item \textbf{Abstração:} Focar apenas nos detalhes relevantes.
        \item \textbf{Algoritmos:} Criar passos para resolver o problema.
    \end{itemize}
\end{frame}

\section{Condicionais no Excel}

\begin{frame}
    \frametitle{O que são Condicionais?}
    \framesubtitle{Lógica de Decisão}
    Condicionais são estruturas que permitem tomar decisões com base em critérios específicos. No Excel, isso é feito com funções como:
    \begin{itemize}
        \item \textbf{SE (IF):} Executa uma ação se uma condição for verdadeira.
        \item \textbf{E (AND):} Verifica se todas as condições são verdadeiras.
        \item \textbf{OU (OR):} Verifica se pelo menos uma condição é verdadeira.
    \end{itemize}
    \textbf{Exemplo:\\} \texttt{=SE(A1 > 10; "Aprovado"; "Reprovado")}
\end{frame}

\section{Exemplos Práticos}

\begin{frame}
    \frametitle{Exemplo 1: Classificação de Notas}
    \framesubtitle{Uso da Função SE}
    \textbf{Problema:} Classificar notas de alunos como "Aprovado" ou "Reprovado".
    \begin{itemize}
        \item \textbf{Condição:} Se a nota for maior ou igual a 6, "Aprovado"; caso contrário, "Reprovado".
        \item \textbf{Fórmula: \\} \texttt{=SE(B2 >= 6; "Aprovado"; "Reprovado")}
    \end{itemize}
\end{frame}

\begin{frame}
    \frametitle{Exemplo 2: Desconto Progressivo}
    \framesubtitle{Uso de SE Aninhadas}
    \textbf{Problema:} Aplicar descontos com base no valor da compra:
    \begin{itemize}
        \item \textbf{Se valor $\leq $  R\$ 100:} Sem desconto.
        \item \textbf{Se valor $>$ R\$ 100 e $<$ R\$ 200:} 5\% de desconto.
        \item \textbf{Se valor $>$ R\$ 200:} 10\% de desconto.
    \end{itemize}
    \textbf{Fórmula:\\} 
    \texttt{=SE(B2 < 100; B2; SE(B2 < 200; B2 * 0.95; B2 * 0.90))}
\end{frame}

\begin{frame}
    \frametitle{Exemplo 3A: Validação de Dados com E}
    \framesubtitle{Verificação de Condições Múltiplas}

    \textbf{Problema:} Um clube esportivo permite acesso à piscina somente se:
    \begin{itemize}
        \item \textbf{Condição 1:} Idade $\geq$ 12 anos
        \item \textbf{Condição 2:} Possui exame médico válido (indicado por "S")
        \item \textbf{Condição 3:} Está usando touca (indicado por "S")
    \end{itemize}

    \textbf{Fórmula Excel:}
    \texttt{=E(B2 >= 12; C2 = "S"; D2 = "S")}

    \begin{exampleblock}{Exemplo Prático}
        \begin{tabular}{|c|c|c|c|c|}
            \hline
            \textbf{Nome} & \textbf{Idade} & \textbf{Exame} & \textbf{Touca} & \textbf{Situação} \\ \hline
            João & 15 & "S" & "N" & \textcolor{red}{Negado} \\ \hline
            Maria & 10 & "S" & "S" &  \textcolor{red}{Negado} \\ \hline
            Pedro & 14 & "S" & "S" &  \textcolor{green}{Permitido} \\ \hline
        \end{tabular}
    \end{exampleblock}
\end{frame}

\begin{frame}
    \frametitle{Exemplo 3B: Validação com OU}
    \framesubtitle{Atendimento a Múltiplos Critérios}

    \textbf{Problema:} Uma loja oferece frete grátis se:
    \begin{itemize}
        \item \textbf{Condição 1:} Compra acima de R\$ 300 \textbf{ou}
        \item \textbf{Condição 2:} É cliente premium (indicado por "S") \textbf{ou}
        \item \textbf{Condição 3:} CEP está na zona de entrega rápida leste.
    \end{itemize}

    \textbf{Fórmula Excel:}
    \texttt{=OU(B2 > 300; C2 = "S"; D2 = "Zona Leste")}

    \begin{exampleblock}{Casos de Teste}
        \begin{tabular}{|c|c|c|c|}
            \hline
            \textbf{Valor} & \textbf{Premium} & \textbf{Zona} & \textbf{Frete} \\ \hline
            R\$ 250 & "N" & "Oeste" & \textcolor{red}{Pago} \\ \hline
            R\$ 150 & "S" & "Norte" & \textcolor{green}{Grátis} \\ \hline
            R\$ 400 & "N" & "Leste" & \textcolor{green}{Grátis} \\ \hline
        \end{tabular}
    \end{exampleblock}
\end{frame}

\section{Exercícios Práticos}

\begin{frame}
    \frametitle{Exercício 1: Cálculo de Frete}
    \framesubtitle{Uso de Condicionais}

    \textbf{Problema:} Calcular o frete com base no peso do pacote:
    \begin{itemize}
        \item \textbf{Se peso $\leq$ 1 kg:} R\$ 10,00
        \item \textbf{Se peso $>$ 1 kg e $\leq$ 5 kg:} R\$ 20,00
        \item \textbf{Se peso $>$ 5 kg:} R\$ 30,00
    \end{itemize}

    \begin{exampleblock}{Exercício: Calcule o Frete}
        \begin{tabular}{|c|c|}
            \hline
            \textbf{Peso (kg)} & \textbf{Valor do Frete} \\ \hline
            0,8 & \rule{2cm}{} \\ \hline
            1,0 & \rule{2cm}{} \\ \hline
            1,5 & \rule{2cm}{} \\ \hline
            3,2 & \rule{2cm}{} \\ \hline
            5,0 & \rule{2cm}{} \\ \hline
            7,3 & \rule{2cm}{} \\ \hline
        \end{tabular}
    \end{exampleblock}
\end{frame}

\begin{frame}
    \frametitle{Exercício 2: Bonificação de Funcionários}
    \framesubtitle{Uso de SE Aninhadas}
    \textbf{Problema:} Calcular a bonificação de funcionários com base no desempenho:
    \begin{itemize}
        \item \textbf{Se desempenho = ``Excelente":} R\$ 500,00.
        \item \textbf{Se desempenho = ``Bom":} R\$ 300,00.
        \item \textbf{Se desempenho = ``Regular":} R\$ 100,00.
        \item \textbf{Caso contrário:} Sem bonificação.
    \end{itemize}

    \begin{exampleblock}{Calcule a bonificação para cada caso}
        \begin{tabular}{|c|c|}
            \hline
            \textbf{Desempenho} & \textbf{Bonificação (R\$)} \\ \hline
            Excelente & \rule{2cm}{} \\ \hline
            Bom & \rule{2cm}{} \\ \hline
            Regular & \rule{2cm}{} \\ \hline
            Satisfatório & \rule{2cm}{} \\ \hline
            Ruim & \rule{2cm}{} \\ \hline
        \end{tabular}
    \end{exampleblock}
\end{frame}

% Slide de encerramento
\begin{frame}
    \begin{center}
        \textbf{\textcolor{blue}{\Large Obrigado pela atenção!}} \\[0.5cm]
        \textbf{Pratique no Excel e explore as possibilidades!}
    \end{center}
\end{frame}

\end{document}
