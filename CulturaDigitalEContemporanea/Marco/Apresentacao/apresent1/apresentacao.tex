\documentclass[12pt]{beamer}
\usepackage[utf8]{inputenc}
\usepackage[brazilian]{babel}
\usepackage{amsmath}
\usepackage{xcolor}

\usetheme{Berkeley}

% Definir cores para títulos e subtítulos
\setbeamercolor{frametitle}{fg=white}
\setbeamercolor{framesubtitle}{fg=green}
% Adicionar nota de rodapé em todos os slides
\setbeamertemplate{footline}{
    \vspace{0.1cm}
    \footnotesize
    \hspace{2cm} Mestre em Modelagem Matemática e Computacional - UFPB
}

\title{\textcolor{white}{Cultura Digital \\ Introdução ao Pensamento Computacional}}
\author{Professor: Jefferson}
\date{}

\begin{document}

\frame{\titlepage}

% Slide do sumário
\begin{frame}
    \frametitle{Sumário}
    \tableofcontents
\end{frame}

\begin{frame}
    \section{O que é Pensamento Computacional?}
    \frametitle{O que é Pensamento Computacional?}
    \framesubtitle{Definição e Pilares}
    \begin{itemize}
        \item Abordagem para resolver problemas de forma eficiente.
        \item Quatro pilares principais:
        \begin{itemize}
            \item \textbf{Decomposição}: Dividir problemas complexos.
            \item \textbf{Reconhecimento de Padrões}: Identificar similaridades.
            \item \textbf{Abstração}: Focar nos detalhes relevantes.
            \item \textbf{Algoritmos}: Criar um passo a passo.
        \end{itemize}
    \end{itemize}
\end{frame}

\begin{frame}
    \section{Exemplo Prático: Planejando uma Festa}
    \frametitle{Exemplo Prático: Planejando uma Festa}
    \framesubtitle{Aplicando os Pilares}
    \begin{itemize}
        \item \textbf{Decomposição}:
        \begin{itemize}
            \item Escolher o local.
            \item Fazer a lista de convidados.
            \item Comprar alimentos e bebidas.
            \item Decorar o local.
        \end{itemize}
        \item \textbf{Reconhecimento de Padrões}:
        \begin{itemize}
            \item Convidados que gostam de música.
            \item Alimentos populares em festas.
        \end{itemize}
        \item \textbf{Abstração}:
        \begin{itemize}
            \item Focar nas áreas principais para decorar.
            \item Comprar apenas o necessário.
        \end{itemize}
        \item \textbf{Algoritmos}:
        \begin{itemize}
            \item Passo a passo para realizar a festa.
        \end{itemize}
    \end{itemize}
\end{frame}

\begin{frame}
    \section{Atividade Prática 1}
    \frametitle{Atividade Prática}
    \framesubtitle{Planejando uma Viagem}
    \begin{itemize}
        \item Aplique os quatro pilares do pensamento computacional para planejar uma viagem.
        \item Descreva cada etapa:
        \begin{itemize}
            \item Decomposição.
            \item Reconhecimento de Padrões.
            \item Abstração.
            \item Algoritmos.
        \end{itemize}
    \end{itemize}
\end{frame}

\begin{frame}
    \section{Atividade Prática}
    \frametitle{Atividade Prática 2}
    \framesubtitle{Resolvendo um Problema}
    \begin{itemize}
        \item Escolha um problema do seu dia a dia e aplique o pensamento computacional para resolvê-lo.
        \item Descreva cada etapa.

    \end{itemize}
\end{frame}



\begin{frame}
    \section{Conclusão}
    \frametitle{Conclusão}
    \begin{itemize}
        \item O pensamento computacional é uma ferramenta poderosa para resolver problemas.
        \item A prática constante é essencial para dominar os conceitos.
        \item Continue aplicando os pilares em diferentes situações!
    \end{itemize}
\end{frame}

\begin{frame}
    \frametitle{Referências}
    \framesubtitle{Livros e materiais utilizados}
    \begin{itemize}
        \item WING, Jeannette M. \textbf{Computational Thinking}. Communications of the ACM, 2006.
        \item GROVER, Shuchi; PEA, Roy. \textbf{Computational Thinking in K–12: A Review of the State of the Field}. Educational Researcher, 2013.
    \end{itemize}
\end{frame}

\begin{frame}
    \begin{center}
        \textbf{\textcolor{blue}{\Large Obrigado pela atenção!}} \\[0.5cm]
    \end{center}
\end{frame}

\end{document}
