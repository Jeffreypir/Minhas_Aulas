\documentclass[12pt]{beamer}
\usepackage[utf8]{inputenc}
\usepackage[brazil]{babel}
\usepackage{amsmath}
\usepackage{xcolor}

\usetheme{Berkeley}

% Definir cores para títulos e subtítulos
\setbeamercolor{frametitle}{fg=white}
\setbeamercolor{framesubtitle}{fg=green}
% Adicionar nota de rodapé em todos os slides
\setbeamertemplate{footline}{
    \vspace{0.1cm}
    \footnotesize
%    \hspace{2cm} Mestre em Modelagem Matemática e Computacional - UFPB
}

\title{\textcolor{white}{Cultura Digital \\  Algoritmos e Lógica de Programação}}
\author{Professor: Jefferson}
\date{}

\begin{document}

\frame{\titlepage}

% Slide do sumário
\begin{frame}
    \frametitle{Sumário}
    \tableofcontents
\end{frame}

\begin{frame}
    \section{O que é um Algoritmo?}
    \frametitle{O que é um Algoritmo?}
    \framesubtitle{Definição e Exemplo}
    \begin{itemize}
        \item Sequência de passos para resolver um problema.
        \item Exemplo: Fazer um sanduíche de queijo.
        \begin{enumerate}
            \item Pegue duas fatias de pão.
            \item Coloque uma fatia de queijo entre as fatias de pão.
            \item Leve o sanduíche ao forno por 5 minutos.
            \item Retire o sanduíche do forno.
            \item Sirva e aproveite!
        \end{enumerate}
    \end{itemize}
\end{frame}

\begin{frame}
    \section{Lógica de Programação}
    \frametitle{Lógica de Programação}
    \framesubtitle{Conceitos Básicos}
    \begin{itemize}
        \item \textbf{Variáveis}: Armazenam dados.
        \item \textbf{Estruturas Condicionais}: Tomam decisões com base em condições.
        \item \textbf{Laços de Repetição}: Repetem ações várias vezes.
    \end{itemize}
\end{frame}

\begin{frame}
    \section{Exemplo Prático: Calculadora Simples}
    \frametitle{Exemplo Prático: Calculadora Simples}
    \framesubtitle{Passo a Passo}
    \begin{enumerate}
        \item Peça ao usuário para digitar o primeiro número.
        \item Armazene o número em uma variável chamada \texttt{numero1}.
        \item Peça ao usuário para digitar o segundo número.
        \item Armazene o número em uma variável chamada \texttt{numero2}.
        \item Some os dois números e armazene o resultado em uma variável chamada \texttt{resultado}.
        \item Mostre o valor de \texttt{resultado} para o usuário.
    \end{enumerate}
\end{frame}

\begin{frame}
    \section{Atividade Prática}
    \frametitle{Atividade Prática}
    \framesubtitle{Criando um Algoritmo}
    \begin{itemize}
        \item Crie um algoritmo para fazer uma lista de compras.
        \item Descreva cada passo:
        \begin{itemize}
            \item Identifique os itens necessários.
            \item Organize os itens por categoria.
            \item Verifique os preços.
            \item Compre os itens.
        \end{itemize}
    \end{itemize}
\end{frame}

\begin{frame}
    \section{Conclusão}
    \frametitle{Conclusão}
    \begin{itemize}
        \item Algoritmos são a base da programação.
        \item A lógica de programação ajuda a organizar os passos de um algoritmo.
        \item Continue praticando com exemplos do dia a dia!
    \end{itemize}
\end{frame}

\begin{frame}
    \frametitle{Referências}
    \framesubtitle{Livros e materiais utilizados}
    \begin{itemize}
        \item WING, Jeannette M. \textbf{Computational Thinking}. Communications of the ACM, 2006.
        \item GROVER, Shuchi; PEA, Roy. \textbf{Computational Thinking in K–12: A Review of the State of the Field}. Educational Researcher, 2013.
    \end{itemize}
\end{frame}

\begin{frame}
    \begin{center}
        \textbf{\textcolor{blue}{\Large Obrigado pela atenção!}} \\[0.5cm]
    \end{center}
\end{frame}

\end{document}
