\documentclass[11pt]{article}
\usepackage[utf8]{inputenc}
\usepackage{multicol}
\usepackage{geometry}
\usepackage{enumitem}
\usepackage{xcolor}
\usepackage{titlesec}
\usepackage{graphicx}

\geometry{a4paper, left=0.5cm, right=0.5cm, top=1.5cm, bottom=1.5cm}
\setlength{\columnsep}{0.7cm}
\setlist{nosep,leftmargin=*}

\titleformat{\section}
  {\normalfont\large\bfseries\color{blue}}
  {\thesection}
  {1em}
  {}

\title{\color{blue}{Aula de PowerPoint -- Guia Passo a Passo}}
\author{Professor: Jefferson \\ \color{blue}{Pensamento Computacional}}
\date{}

\begin{document}

\maketitle

\begin{multicols}{2}

\section*{1. Abrindo o Programa}

\begin{enumerate}
\item \textbf{Passo 1:} Clique no botão Iniciar do Windows (ícone no canto inferior esquerdo)

\item \textbf{Passo 2:} Digite ``PowerPoint'' usando seu teclado

\item \textbf{Passo 3:} Quando aparecer o ícone do PowerPoint, clique nele com o botão esquerdo do mouse

\item \textbf{Passo 4:} Na tela inicial, clique em \textbf{``Apresentação em branco''} (é a primeira opção)
\end{enumerate}

\subsection*{O que você vê agora:}
Uma tela branca com:
\begin{itemize}
\item Menu superior (chamado de ``Faixa de Opções'')
\item Slide vazio à esquerda
\item Área de trabalho do slide à direita
\end{itemize}

\section*{2. Criando o Slide Título}

\begin{enumerate}
\item \textbf{Passo 1:} Observe as duas caixas no slide:
\begin{itemize}
\item ``Clique para adicionar título'' (parte de cima)
\item ``Clique para adicionar subtítulo'' (parte de baixo)
\end{itemize}

\item \textbf{Passo 2:} Clique na caixa de título e digite: \textbf{``Meu Primeiro PowerPoint''}

\item \textbf{Passo 3:} Clique na caixa de subtítulo e digite: \textbf{Seu Nome} e \textbf{Sua Turma}

\item \textbf{Passo 4:} Para mudar a cor:
\begin{enumerate}
\item Selecione o texto
\item Clique em ``Formatar'' no menu
\item Escolha ``Cor do Texto''
\item Selecione azul escuro
\end{enumerate}
\end{enumerate}

\subsection*{Dica Profissional:}
Títulos devem ter fonte grande (entre 36-44pt). Para alterar:
\begin{enumerate}
\item Selecione o texto
\item Na guia ``Página Inicial'', mude o número da fonte
\end{enumerate}

\end{multicols}

\begin{multicols}{2}

\section*{3. Adicionando Novos Slides}

\begin{enumerate}
\item \textbf{Passo 1:} Clique na guia ``Página Inicial'' no menu superior

\item \textbf{Passo 2:} Localize o botão ``Novo Slide'' (tem um ícone de um slide com uma estrela)

\item \textbf{Passo 3:} Clique na setinha abaixo do botão para ver os layouts

\item \textbf{Passo 4:} Escolha ``Título e Conteúdo'' (segunda opção)

\item \textbf{Passo 5:} Repita para criar:
\begin{itemize}
\item 1 slide de ``Comparação''
\item 1 slide de ``Seção''
\end{itemize}
\end{enumerate}

\subsection*{Atividade Prática 1:}
Crie esta sequência de slides:
\begin{enumerate}
\item Slide 1: Título (já feito)
\item Slide 2: Título e Conteúdo
\item Slide 3: Comparação
\item Slide 4: Seção
\item Slide 5: Título somente
\end{enumerate}

\section*{4. Inserindo Conteúdo}

\subsection*{Texto Básico:}
\begin{enumerate}
\item \textbf{Passo 1:} No slide 2, clique em ``Clique para adicionar título''

\item \textbf{Passo 2:} Digite: \textbf{``Minhas Férias''}

\item \textbf{Passo 3:} Na caixa de conteúdo, clique no ícone de marcadores

\item \textbf{Passo 4:} Digite:
\begin{itemize}
\item Onde fui
\item O que fiz
\item Quem encontrei
\item Fotos
\end{itemize}
\end{enumerate}

\subsection*{Listas Numeradas:}
\begin{enumerate}
\item \textbf{Passo 1:} Selecione o texto da lista
\item \textbf{Passo 2:} Clique em ``Marcadores'' $\rightarrow$ ``Numerados''
\end{enumerate}

\end{multicols}

\begin{multicols}{2}

\section*{5. Trabalhando com Imagens}

\begin{enumerate}
\item \textbf{Passo 1:} No slide 2, clique no ícone ``Imagens'' no centro da caixa de conteúdo

\item \textbf{Passo 2:} Navegue até onde salvou suas fotos

\item \textbf{Passo 3:} Selecione uma imagem e clique em ``Inserir''

\item \textbf{Passo 4:} Para redimensionar:
\begin{itemize}
\item Clique na imagem
\item Arraste os círculos nos cantos (mantenha Shift para não distorcer)
\end{itemize}

\item \textbf{Passo 5:} Para mover:
\begin{itemize}
\item Posicione o mouse até aparecer a seta quádrupla
\item Arraste para o local desejado
\end{itemize}
\end{enumerate}

\subsection*{Dica de Organização:}
\begin{itemize}
\item Imagens no topo ou à direita têm melhor visualização
\item Sempre deixe espaço em branco ao redor
\item Use no máximo 2 imagens por slide
\end{itemize}

\section*{6. Formatando o Design}

\begin{enumerate}
\item \textbf{Passo 1:} Clique na guia ``Design'' no menu superior

\item \textbf{Passo 2:} Passe o mouse sobre os temas para pré-visualizar

\item \textbf{Passo 3:} Escolha um tema claro (evite fundos escuros)

\item \textbf{Passo 4:} Para mudar cores:
\begin{itemize}
\item Clique em ``Variantes'' $\rightarrow$ ``Cores''
\item Escolha uma combinação harmoniosa
\end{itemize}
\end{enumerate}

\subsection*{Regras de Ouro:}
\begin{itemize}
\item Use no máximo 3 cores principais
\item Fontes sem serifa (Arial, Calibri) são mais legíveis
\item Contraste alto entre texto e fundo
\end{itemize}

\end{multicols}


\begin{multicols}{2}

\section*{7. Salvando Seu Trabalho}

\begin{enumerate}
\item \textbf{Passo 1:} Clique no menu ``Arquivo'' (canto superior esquerdo)

\item \textbf{Passo 2:} Selecione ``Salvar Como''

\item \textbf{Passo 3:} Escolha a pasta de seu computador

\item \textbf{Passo 4:} No campo ``Nome do arquivo'', digite: \textbf{Apresentacao\_[SeuNome]}

\item \textbf{Passo 5:} Verifique se o tipo é ``Apresentação PowerPoint (.pptx)''

\item \textbf{Passo 6:} Clique em ``Salvar''
\end{enumerate}

\subsection*{Dica de Segurança:}
Salve seu trabalho a cada 10 minutos (Ctrl+S)

\section*{8. Apresentando Seu Slide}

\begin{enumerate}
\item \textbf{Passo 1:} Vá para o slide 1 (clique nele no painel esquerdo)

\item \textbf{Passo 2:} Pressione F5 no teclado (ou clique em ``Apresentação de Slides'')

\item \textbf{Passo 3:} Para avançar:
\begin{itemize}
\item Clique com o mouse
\item Ou pressione a seta para direita
\end{itemize}

\item \textbf{Passo 4:} Para voltar: seta para esquerda

\item \textbf{Passo 5:} Para sair: tecla ESC
\end{enumerate}

\subsection*{Modo Apresentador:}
\begin{itemize}
\item Clique com botão direito $\rightarrow$ ``Mostrar Modo Apresentador''
\item Verá suas anotações enquanto o público vê apenas os slides
\end{itemize}

\end{multicols}

\end{document}
