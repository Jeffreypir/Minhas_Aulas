\documentclass[11pt]{article}
\usepackage[utf8]{inputenc}
\usepackage{multicol}
\usepackage{geometry}
\usepackage{enumitem}
\usepackage{xcolor}
\usepackage{titlesec}

\geometry{a4paper, left=1.5cm, right=1.5cm, top=1.5cm, bottom=1.5cm}

\titleformat{\section}
  {\normalfont\large\bfseries\color{blue}}
  {\thesection}
  {1em}
  {}

\title{\color{blue}{Atividade Prática - Funções Lógicas no Excel (Explicada)}}
\author{Professor: Jefferson \\ \color{blue}{Pensamento Computacional}}
\date{}

\begin{document}

\maketitle


\section*{Introdução às Condicionais}

As funções condicionais permitem automatizar decisões em planilhas. Aprenderemos:

\begin{itemize}
\item \texttt{SE} - Tomada de decisão simples
\item \texttt{E} - Todas as condições devem ser verdadeiras
\item \texttt{OU} - Pelo menos uma condição deve ser verdadeira
\end{itemize}

\section*{Parte 1: Função SE Básica}

\subsection*{Estrutura:}
\begin{verbatim}
=SE(teste_lógico; valor_se_verdadeiro; valor_se_falso)
\end{verbatim}

\subsection*{Exemplo Prático:}
Verificar se um aluno foi aprovado (nota $\geq$ 7):

\begin{tabular}{|c|c|}
\hline
\textbf{Nota} & \textbf{Resultado} \\
\hline
8,5 & =SE(A2$>=$7;"Aprovado";"Reprovado") \\
\hline
4,0 & (automaticamente preenchido) \\
\hline
\end{tabular}

\subsection*{Passo a Passo:}
1. Clique na célula B2 \\
2. Digite a fórmula acima \\
3. Pressione Enter \\
4. Arraste a alça de preenchimento para B3

\section*{Parte 2: Função E}

\subsection*{Quando Usar:}
Quando \textbf{todas} as condições devem ser atendidas.

\subsection*{Estrutura:}
\begin{verbatim}
=E(condição1; condição2; ...)
\end{verbatim}

\subsection*{Exemplo: Acesso ao Clube}
\begin{tabular}{|c|c|c|}
\hline
Idade & Exame & Acesso \\
\hline
16 & Sim & =SE(E(A2$>=$18;B2="Sim");"Liberado";"Negado") \\
\hline
20 & Sim & (resultado automático) \\
\hline
\end{tabular}

\subsection*{Análise:}
\begin{itemize}
\item 16 anos: Negado (falha na idade)
\item 20 anos: Liberado (ambas condições OK)
\end{itemize}


\section*{Parte 3: Função OU}

\subsection*{Quando Usar:}
Quando \textbf{pelo menos uma} condição deve ser atendida.

\subsection*{Estrutura:}
\begin{verbatim}
=OU(condição1; condição2; ...)
\end{verbatim}

\subsection*{Exemplo: Frete Grátis}
\begin{tabular}{|c|c|c|}
\hline
Valor & Premium & Frete \\
\hline
250 & Não & =SE(OU(A2$>=$300;B2="Sim");"Grátis";"Pago") \\
\hline
150 & Sim & (resultado automático) \\
\hline
\end{tabular}

\subsection*{Análise:}
\begin{itemize}
\item R\$250: Pago (nenhuma condição)
\item R\$150: Grátis (cliente premium)
\end{itemize}

\section*{Parte 4: Combinação E + OU}

\subsection*{Exemplo Complexo:}
Aprovar empréstimo se:
\begin{itemize}
\item Renda $\geq$ 2000 \textbf{e}
\item (Score $\geq$ 600 \textbf{ou} tem fiador)
\end{itemize}

\begin{tabular}{|c|c|c|c|}
\hline
Renda & Score & Fiador & Resultado \\
\hline
3000 & 550 & Sim & =SE(E(A2$>=$2000;OU(B2$>=$600;C2="Sim"));"Aprovado";"Negado") \\
\hline
1800 & 700 & Não & (resultado automático) \\
\hline
\end{tabular}

\subsection*{Análise:}
\begin{itemize}
\item Caso 1: Aprovado (renda OK e tem fiador)
\item Caso 2: Negado (renda insuficiente)
\end{itemize}

\section*{Exercício Prático}

Complete a tabela abaixo usando as regras:
\begin{itemize}
\item "Prioritário" se: categoria = "Alimento" \textbf{e} estoque $<$ 10
\item "Urgente" se: validade $<$ 30 dias \textbf{ou} quebra = "Sim"
\item "Normal" para outros casos
\end{itemize}

\begin{tabular}{|c|c|c|c|c|}
\hline
Produto & Categoria & Estoque & Validade & Status \\
\hline
Arroz & Alimento & 5 & 60 & \\
\hline
Leite & Alimento & 15 & 25 & \\
\hline
Sabão & Limpeza & 8 & 10 & \\
\hline
\end{tabular}

\end{document}
