\documentclass[12pt]{article}
\usepackage[utf8]{inputenc}
\usepackage{amsmath}
\usepackage{multicol}
\usepackage{geometry}
\usepackage{tikz}
\usetikzlibrary{arrows.meta}
\usepackage{enumitem}
\usepackage{xcolor}
\usepackage{titlesec}

\geometry{a4paper, left=1cm, right=1cm, top=1.5cm, bottom=1.5cm}

% Definir a cor das seções como azul
\titleformat{\section}
  {\normalfont\Large\bfseries\color{blue}}
  {\thesection}
  {1em}
  {}

\renewcommand{\thesubsection}{\textcolor{red}{\arabic{section}.\arabic{subsection}}}
\titleformat{\subsection}{\color{red}\normalfont\bfseries}{\thesubsection}{1em}{}
\title{\textcolor{blue}{Cultura Digital\\ Algoritmos e Lógica de Programação}}
\author{Professor: Jefferson}
\date{}

\begin{document}

\maketitle

\begin{center}
\large{Nome: \underline{\hspace{8cm}} \quad Série-Turma: \underline{\hspace{3cm}}}
\end{center}

\begin{multicols}{2}

\section*{O que é um Algoritmo?}

Um algoritmo é uma sequência de passos bem definidos para resolver um problema ou realizar uma tarefa. Ele é a base da programação e pode ser aplicado em diversas áreas, desde a matemática até o dia a dia.

\section*{Exemplo de Algoritmo: Fazer um Sanduíche}

Vamos criar um algoritmo para fazer um sanduíche de queijo:

\begin{enumerate}
    \item Pegue duas fatias de pão.
    \item Coloque uma fatia de queijo entre as fatias de pão.
    \item Leve o sanduíche ao forno por 5 minutos.
    \item Retire o sanduíche do forno.
    \item Sirva e aproveite!
\end{enumerate}

\section*{Lógica de Programação}

A lógica de programação é a forma como organizamos os passos de um algoritmo para que ele possa ser executado por um computador. Ela envolve conceitos como:

\begin{itemize}
    \item \textbf{Variáveis}: Espaços para armazenar dados.
    \item \textbf{Estruturas Condicionais}: Tomar decisões com base em condições.
    \item \textbf{Laços de Repetição}: Repetir ações várias vezes.
\end{itemize}

\section*{Exemplo Prático: Calculadora Simples}

Vamos criar um algoritmo para uma calculadora simples que soma dois números:

\begin{enumerate}
    \item Peça ao usuário para digitar o primeiro número.
    \item Armazene o número em uma variável chamada \texttt{numero1}.
    \item Peça ao usuário para digitar o segundo número.
    \item Armazene o número em uma variável chamada \texttt{numero2}.
    \item Some os dois números e armazene o resultado em uma variável chamada \texttt{resultado}.
    \item Mostre o valor de \texttt{resultado} para o usuário.
\end{enumerate}

\section*{Atividade Prática}

\subsection*{Atividade 1: Criando um Algoritmo}
Crie um algoritmo para fazer uma lista de compras. Descreva cada passo.

\subsection*{Atividade 2: Resolvendo um Problema com Lógica}
Escreva um algoritmo para calcular a média de três notas. Use variáveis e estruturas condicionais.

\end{multicols}

\end{document}
