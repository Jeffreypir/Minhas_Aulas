\documentclass[12pt]{article}
\usepackage[utf8]{inputenc}
\usepackage{amsmath}
\usepackage{multicol}
\usepackage{geometry}
\usepackage{tikz}
\usetikzlibrary{arrows.meta}
\usepackage{enumitem}
\usepackage{xcolor}
\usepackage{titlesec}

\geometry{a4paper, left=1cm, right=1cm, top=1.5cm, bottom=1.5cm}

% Definir a cor das seções como azul
\titleformat{\section}
  {\normalfont\Large\bfseries\color{blue}}
  {\thesection}
  {1em}
  {}

\renewcommand{\thesubsection}{\textcolor{red}{\arabic{section}.\arabic{subsection}}}
\titleformat{\subsection}{\color{red}\normalfont\bfseries}{\thesubsection}{1em}{}
\title{\textcolor{blue}{Aula 1: Introdução ao Pensamento Computacional}}
\author{Professor: Jefferson}
\date{}

\begin{document}

\maketitle

\begin{center}
\large{Nome: \underline{\hspace{8cm}} \quad Série-Turma: \underline{\hspace{3cm}}}
\end{center}

\begin{multicols}{2}

\section*{O que é Pensamento Computacional?}

O pensamento computacional é uma abordagem para resolver problemas de forma eficiente, utilizando conceitos da ciência da computação. Ele envolve quatro pilares principais:

\begin{itemize}
    \item \textbf{Decomposição}: Dividir um problema complexo em partes menores.
    \item \textbf{Reconhecimento de Padrões}: Identificar similaridades e padrões entre os problemas.
    \item \textbf{Abstração}: Focar apenas nos detalhes relevantes, ignorando o que não é importante.
    \item \textbf{Algoritmos}: Criar um passo a passo para resolver o problema.
\end{itemize}

\section*{Exemplo Prático: Planejando uma Festa}

Vamos aplicar o pensamento computacional para planejar uma festa.

\subsection*{1. Decomposição}
Divida o problema em tarefas menores:
\begin{itemize}
    \item Escolher o local.
    \item Fazer a lista de convidados.
    \item Comprar os alimentos e bebidas.
    \item Decorar o local.
\end{itemize}

\subsection*{2. Reconhecimento de Padrões}
Identifique padrões:
\begin{itemize}
    \item Convidados que gostam de música.
    \item Alimentos que são populares em festas.
\end{itemize}

\subsection*{3. Abstração}
Foque no essencial:
\begin{itemize}
    \item Não é necessário decorar todo o local, apenas as áreas principais.
    \item Compre apenas os alimentos e bebidas necessários.
\end{itemize}

\subsection*{4. Algoritmos}
Crie um passo a passo:
\begin{enumerate}
    \item Escolha o local.
    \item Faça a lista de convidados.
    \item Compre os alimentos e bebidas.
    \item Decore o local.
    \item Realize a festa.
\end{enumerate}

\section*{Atividade Prática}

\subsection*{Atividade 1: Planejando uma Viagem}
Aplique os quatro pilares do pensamento computacional para planejar uma viagem. Escreva os passos que você seguiria.

\subsection*{Atividade 2: Resolvendo um Problema}
Escolha um problema do seu dia a dia e aplique o pensamento computacional para resolvê-lo. Descreva cada etapa.

\end{multicols}

\end{document}
