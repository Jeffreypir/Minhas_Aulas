\documentclass[12pt]{article}
\usepackage[utf8]{inputenc}
\usepackage{amsmath}
\usepackage{multicol}
\usepackage{geometry}
\usepackage{tikz}
\usetikzlibrary{arrows.meta}
\usepackage{enumitem}
\usepackage{xcolor}
\usepackage{titlesec}

% Configurações de layout
\geometry{a4paper, left=1cm, right=1cm, top=0.5cm, bottom=1.2cm}
\setlength{\columnseprule}{0.4pt}
\setlength{\baselineskip}{1.0\baselineskip}

% Definir a cor das seções como azul
\titleformat{\section}
  {\normalfont\Large\bfseries\color{blue}}
  {\thesection}
  {1em}
  {}

\renewcommand{\thesubsection}{\textcolor{red}{\arabic{section}.\arabic{subsection}}}
\titleformat{\subsection}{\color{red}\normalfont\bfseries}{\thesubsection}{1em}{}
\title{\textcolor{blue}{Cultura Digital e Contemporânea em \textit{Alita: Anjo de Combate}}}
\author{Professor: \underline{\hspace{4cm}}}
\date{}

\begin{document}

\maketitle

\begin{center}
\large{Nome: \underline{\hspace{8cm}} \quad Série-Turma: \underline{\hspace{3cm}}}
\end{center}

\begin{multicols}{2}

\section*{Introdução à Cultura Digital e Contemporânea}

A cultura digital e contemporânea reflete como a tecnologia redefine nossa sociedade, identidade e relações humanas. O filme \textit{Alita: Anjo de Combate} (2019) é um ótimo exemplo para discutir esses temas, abordando:

\begin{itemize}
    \item \textbf{Fusão humano-máquina}: A hibridização entre corpo biológico e tecnologia.
    \item \textbf{Ética tecnológica}: Os limites da inteligência artificial e da manipulação genética.
    \item \textbf{Desigualdade social}: A divisão entre as elites tecnológicas e as populações marginalizadas.
    \item \textbf{Identidade digital}: A reconstrução da identidade em um mundo pós-humano.
\end{itemize}

\section*{Análise do Filme}

\subsection*{1. Fusão Humano-Máquina}
Em \textit{Alita}, os corpos são frequentemente modificados com próteses cibernéticas. Discuta:
\begin{itemize}
    \item Como a tecnologia redefine o que é ``ser humano"?
    \item Quais os benefícios e riscos dessa fusão?
\end{itemize}

\subsection*{2. Ética Tecnológica}
O filme mostra personagens como Dr. Dyson Ido e Nova, que usam a tecnologia de formas opostas:
\begin{itemize}
    \item Compare seus objetivos e métodos.
    \item A tecnologia é neutra ou depende de quem a controla?
\end{itemize}


\subsection*{3. Desigualdade Social}
A cidade de Iron City contrasta com Zalem, a cidade flutuante:
\begin{itemize}
    \item Como a tecnologia aprofunda ou reduz desigualdades?
    \item Há paralelos com o mundo atual?
\end{itemize}

\subsection*{4. Identidade Digital}
Alita é uma ciborgue que busca recuperar suas memórias:
\begin{itemize}
    \item O que nos define: memórias, corpos ou ações?
    \item Como a identidade é construída no mundo digital hoje?
\end{itemize}

\section*{Atividade Prática}

\subsection*{Atividade 1: Debate - ``Tecnologia e Humanidade"}
Divida a turma em dois grupos:
\begin{enumerate}
    \item \textbf{Pró-tecnologia}: Argumente como a fusão humano-máquina pode melhorar a sociedade.
    \item \textbf{Críticos}: Discuta os riscos de perder nossa humanidade.
\end{enumerate}

\subsection*{Atividade 2: Criação de um ``Manifesto Digital"}
Em grupos, elaborem um manifesto com:
\begin{itemize}
    \item 3 princípios éticos para o uso da tecnologia.
    \item 1 proposta para reduzir desigualdades digitais.
    \item 1 exemplo de como a tecnologia pode preservar a identidade humana.
\end{itemize}

\subsection*{Atividade 3: Analogias Contemporâneas}
Escolha uma cena do filme e relacione-a com:
\begin{itemize}
    \item Um avanço tecnológico real (ex.: inteligência artificial, próteses biônicas).
    \item Um problema social atual (ex.: vigilância digital, exclusão tecnológica).
\end{itemize}

\end{multicols}

\end{document}
