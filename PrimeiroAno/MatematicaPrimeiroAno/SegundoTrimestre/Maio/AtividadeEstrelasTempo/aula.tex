\documentclass[11pt]{article}
\usepackage[utf8]{inputenc}
\usepackage[T1]{fontenc}
\usepackage{amsmath}
\usepackage{multicol}
\usepackage{geometry}
\usepackage{tikz}
\usetikzlibrary{shapes.geometric, arrows.meta, calc}
\usepackage{enumitem}
\usepackage{xcolor}
\usepackage{titlesec}

% Configurações visuais
\geometry{a4paper, left=1cm, right=1cm, top=0.5cm, bottom=1.2cm}
\setlength{\columnseprule}{0.4pt}
\definecolor{titleblue}{RGB}{0,80,150}
\definecolor{sectionred}{RGB}{180,0,0}
\definecolor{darkgreen}{RGB}{0,100,0}

\titleformat{\section}{\normalfont\Large\bfseries\color{titleblue}}{\thesection}{1em}{}
\titleformat{\subsection}{\normalfont\large\bfseries\color{sectionred}}{\thesubsection}{1em}{}
\titleformat{\subsubsection}{\normalfont\normalsize\bfseries\color{darkgreen}}{\thesubsubsection}{1em}{}

\title{\textcolor{titleblue}{Matemática e Etnomatemática: Estrelas Além do Tempo}}
\author{Professor: Jefferson}
\date{}

\begin{document}

\maketitle
\vspace{-1cm}

\begin{center}
\large{Nome: \underline{\hspace{8cm}} \quad Turma: \underline{\hspace{3cm}}}
\end{center}

\begin{multicols}{2}

\section*{1. A Matemática da Desigualdade}

\subsection*{1.1 Caminho Injusto}
\begin{itemize}
    \item Katherine Johnson andava 800 metros até o banheiro.
    \item Isso fazia ela perder 36 minutos por dia.
    \item \textbf{Pergunta:} Quantas horas ela perdia por ano? (Considere 260 dias úteis)
\end{itemize}

\[ T = \frac{36 \times 260}{60} \quad \text{(tempo em horas)}\]

\subsection*{1.2 Diferença Salarial}
\begin{center}
\begin{tabular}{|c|c|}
\hline
\textbf{Cargo} & \textbf{Salário anual (1961)} \\
\hline
Mulher branca & \$5.000 \\
Mulher negra & \$2.000 \\
\hline
\end{tabular}
\end{center}

\textbf{Pergunta:} Dorothy Vaughan era chefe de 30 mulheres negras. Quanto a NASA economizava ao pagar menos?

\section*{2. Resistência com Matemática}

\subsection*{2.1 Algoritmo Discriminatório}
Dorothy aprendeu programação para poder usar o computador da NASA:

\begin{verbatim}
IF (cor_da_pele = negra) THEN
   acesso = negado
ELSE
   acesso = permitido
END IF
\end{verbatim}

\textbf{Reflexão:} O que esse código mostra sobre o racismo na época?

\subsection*{2.2 Outros Jeitos de Contar}
\begin{itemize}
    \item Os povos Yorubá usavam base 20: 45 = (2×20) + 5
    \item \textbf{Atividade:} Converta 89 para base 20.
    \item \textbf{Pergunta:} Por que esses saberes não eram usados na NASA?
\end{itemize}

\section*{3. Análises Críticas}

\subsection*{3.1 O que o Gráfico Mostra?}
\begin{center}
\begin{tikzpicture}[scale=0.8]
    \draw[->] (0,0) -- (5,0) node[below] {Ano};
    \draw[->] (0,0) -- (0,4) node[left] {Nº de mulheres negras};
    \draw (1,0.5) -- (1,1) node[above] {5};
    \draw (2,0.5) -- (2,1.2) node[above] {8};
    \draw (3,0.5) -- (3,1.5) node[above] {12};
    \draw (4,0.5) -- (4,0.8) node[above] {4};
    \node at (1,0) [below] {1958};
    \node at (4,0) [below] {1965};
\end{tikzpicture}
\end{center}

\textbf{Pergunta:} O que aconteceu com as funcionárias negras depois da Lei dos Direitos Civis (1964)?

\subsection*{3.2 Carta para a NASA}
Escreva um parágrafo explicando:
\begin{itemize}
    \item Como os cálculos de Katherine mostraram sua inteligência
    \item Use números do filme como prova
\end{itemize}

\section*{4. E Hoje em Dia?}

\subsection*{4.1 Quem faz doutorado?}
\begin{itemize}
    \item Em 2023, só 3\% dos PhD em matemática nos EUA eram negros.
    \item \textbf{Conta:} Se há 1.200 doutorandos, quantos são negros?
\end{itemize}

\subsection*{4.2 Tecnologia com Preconceito}
\begin{quote}
"Programas de reconhecimento facial erram 35\% das vezes para rostos negros." (MIT, 2022)
\end{quote}

\textbf{Discussão:} Como isso lembra os erros do computador IBM no filme?

\end{multicols}

\end{document}

































































































































