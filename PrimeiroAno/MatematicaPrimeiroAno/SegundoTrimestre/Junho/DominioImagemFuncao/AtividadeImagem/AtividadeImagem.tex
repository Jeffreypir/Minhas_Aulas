\documentclass[11pt]{article}
\usepackage[utf8]{inputenc}
\usepackage[T1]{fontenc}
\usepackage{amsmath}
\usepackage{multicol}
\usepackage{geometry}
\usepackage{tikz}
\usepackage{enumitem}
\usepackage{xcolor}
\usepackage{titlesec}
\usepackage{tcolorbox}

% Configurações de layout
\geometry{a4paper, left=1cm, right=1cm, top=0.5cm, bottom=1.2cm}
\setlength{\columnseprule}{0.4pt}
\setlength{\baselineskip}{1.0\baselineskip}

% Cores personalizadas
\definecolor{titleblue}{RGB}{0,80,150}
\definecolor{sectionred}{RGB}{180,0,0}
\definecolor{darkgreen}{RGB}{0,100,0}
\definecolor{explanationbg}{RGB}{240,248,255}

% Formatação de títulos
\titleformat{\section}{\normalfont\Large\bfseries\color{titleblue}}{\thesection}{1em}{}
\titleformat{\subsection}{\normalfont\large\bfseries\color{sectionred}}{\thesubsection}{1em}{}

\title{\textcolor{titleblue}{Atividade Avaliativa: Imagem de Funções do 1º Grau}}
\author{Professor: Jefferson}
\date{}

\begin{document}

\maketitle
\vspace{-1cm}

\begin{center}
    \large{\textbf{Observação:} Respostas no caderno com letra legível. \quad Série: 1º Ano. Valor: 1,0}
\end{center}

\begin{multicols}{2}

\section*{Atividade}
\begin{enumerate}

\item \textbf{Definição Simples}\\
Se uma função do 1º grau é dada por \( f(x) = 2x + 1 \), qual é o valor de \( f(3) \)?
\begin{tcolorbox}[colback=explanationbg,colframe=titleblue,title=Dica:]
Substitua \( x \) por 3 na expressão.
\end{tcolorbox}

\item \textbf{Imagem Básica}\\
Qual é a imagem da função \( g(x) = 5 \) para qualquer valor de \( x \)?
\begin{tcolorbox}[colback=explanationbg,colframe=titleblue,title=Dica:]
Função constante sempre retorna o mesmo valor.
\end{tcolorbox}

\item \textbf{Cálculo Direto}\\
Dada \( h(x) = -x + 4 \), calcule \( h(0) \) e \( h(1) \).
\begin{tcolorbox}[colback=explanationbg,colframe=titleblue,title=Dica:]
Substitua \( x \) por 0 e depois por 1.
\end{tcolorbox}

\item \textbf{Gráfico Simples}\\
Se o gráfico de \( f(x) = 3x - 2 \) é uma reta, qual é o valor de \( y \) quando \( x = 1 \)?
\begin{tcolorbox}[colback=explanationbg,colframe=titleblue,title=Dica:]
Calcule \( f(1) \).
\end{tcolorbox}

\item \textbf{Problema Contextualizado}\\
Uma lanchonete vende sucos por R\$ 3,00 cada. Se \( f(x) = 3x \) representa o custo para \( x \) sucos, quanto custam 5 sucos?
\begin{tcolorbox}[colback=explanationbg,colframe=titleblue,title=Dica:]
Calcule \( f(5) \).
\end{tcolorbox}

\item \textbf{Valores Extremos}\\
Dada \( f(x) = 2x + 1 \) com \( x \in \{1, 2, 3\} \), qual é o maior valor da imagem?
\begin{tcolorbox}[colback=explanationbg,colframe=titleblue,title=Dica:]
Calcule \( f(1) \), \( f(2) \) e \( f(3) \).
\end{tcolorbox}

\item \textbf{Função Decrescente}\\
Se \( g(x) = -x + 5 \), qual é o valor de \( g(2) \)?
\begin{tcolorbox}[colback=explanationbg,colframe=titleblue,title=Dica:]
Substitua \( x \) por 2 na função.
\end{tcolorbox}

\item \textbf{Comparação Simples}\\
Qual função tem imagem infinita: \( f(x) = 4 \) ou \( g(x) = 2x - 1 \)?
\begin{tcolorbox}[colback=explanationbg,colframe=titleblue,title=Dica:]
Funções não-constantes do 1º grau têm imagem infinita.
\end{tcolorbox}

\item \textbf{Transformação Fácil}\\
Se \( f(x) = x + 2 \), qual é a imagem de \( g(x) = f(x) - 1 \)?
\begin{tcolorbox}[colback=explanationbg,colframe=titleblue,title=Dica:]
Subtraia 1 da função original.
\end{tcolorbox}

\item \textbf{Último Exercício}\\
Dada \( f(x) = \frac{1}{2}x + 3 \), calcule \( f(4) \).
\begin{tcolorbox}[colback=explanationbg,colframe=titleblue,title=Dica:]
Substitua \( x \) por 4 e simplifique.
\end{tcolorbox}

\end{enumerate}
\end{multicols}

\end{document}
