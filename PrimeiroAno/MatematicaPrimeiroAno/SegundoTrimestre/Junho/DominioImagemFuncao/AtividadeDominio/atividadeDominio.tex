\documentclass[11pt]{article}
\usepackage[utf8]{inputenc}
\usepackage[T1]{fontenc}
\usepackage{amsmath}
\usepackage{multicol}
\usepackage{geometry}
\usepackage{tikz}
\usetikzlibrary{shapes.geometric, arrows.meta, calc}
\usepackage{enumitem}
\usepackage{xcolor}
\usepackage{titlesec}
\usepackage{tcolorbox}

% Configurações de layout
\geometry{a4paper, left=1cm, right=1cm, top=0.5cm, bottom=1.2cm}
\setlength{\columnseprule}{0.4pt}
\setlength{\baselineskip}{1.0\baselineskip}

% Cores personalizadas
\definecolor{titleblue}{RGB}{0,80,150}
\definecolor{sectionred}{RGB}{180,0,0}
\definecolor{darkgreen}{RGB}{0,100,0}
\definecolor{explanationbg}{RGB}{240,248,255}

% Formatação de títulos
\titleformat{\section}{\normalfont\Large\bfseries\color{titleblue}}{\thesection}{1em}{}
\titleformat{\subsection}{\normalfont\large\bfseries\color{sectionred}}{\thesubsection}{1em}{}
\titleformat{\subsubsection}{\normalfont\normalsize\bfseries\color{darkgreen}}{\thesubsubsection}{1em}{}

\title{\textcolor{titleblue}{Atividade Avaliativa: Domínio de Funções }}
\author{Professor: Jefferson}
\date{}

\begin{document}

\maketitle
\vspace{-1cm}

\begin{center}
    \large{\textbf{Observação:} Respostas no caderno com letra legível. \quad Série: 1 Ano. }
\end{center}

\begin{multicols}{2}

\section*{Atividade}
\begin{enumerate}

\item \textbf{Função Polinomial}\\
Determine o domínio de $f(x) = 5x^3 - 2x + 7$
\begin{tcolorbox}[colback=explanationbg,colframe=titleblue,title=Dica:]
Funções polinomiais estão definidas para todos os números reais. Não há restrições de denominador ou raiz.
\end{tcolorbox}

\item \textbf{Função Racional}\\
Encontre o domínio de $g(x) = \dfrac{x+2}{x-5}$
\begin{tcolorbox}[colback=explanationbg,colframe=titleblue,title=Dica:]
Em funções racionais, o denominador não pode ser zero. Resolva $x-5 \neq 0$.
\end{tcolorbox}

\item \textbf{Função com Raiz Quadrada}\\
Qual o domínio de $h(x) = \sqrt{x-4}$?
\begin{tcolorbox}[colback=explanationbg,colframe=titleblue,title=Dica:]
Para raízes quadradas, o radicando deve ser $\geq 0$. Resolva $x-4 \geq 0$.
\end{tcolorbox}

\item \textbf{Raiz no Denominador}\\
Determine o maior domínio possível para $f(x) = \dfrac{1}{\sqrt{x+3}}$
\begin{tcolorbox}[colback=explanationbg,colframe=titleblue,title=Dica:]
Duas condições: 1) denominador $\neq 0$  \\ 2) raiz quadrada $> 0$ (já que está no denominador).
\end{tcolorbox}

\item \textbf{Denominador Quadrático}\\
Para $f(x) = \dfrac{x}{x^2-9}$, determine os valores excluídos do domínio
\begin{tcolorbox}[colback=explanationbg,colframe=titleblue,title=Dica:]
Resolva: $x^2-9 = 0$. Valores que zeram o denominador são excluídos.
\end{tcolorbox}

\item \textbf{Combinação de Restrições}\\
Determine o domínio de $f(x) = \sqrt{7-x} + \dfrac{1}{x+2}$
\begin{tcolorbox}[colback=explanationbg,colframe=titleblue,title=Dica:]
Duas partes: 1) $\sqrt{7-x}$ requer $7-x \geq 0$ e 2) $\dfrac{1}{x+2}$ requer $x+2 \neq 0$.
\end{tcolorbox}

\item \textbf{Função Logarítmica}\\
Qual o domínio da função $g(x) = \log(x-1)$?
\begin{tcolorbox}[colback=explanationbg,colframe=titleblue,title=Dica:]
O argumento do logaritmo deve ser $> 0$. Resolva $x-1 > 0$.
\end{tcolorbox}

\item \textbf{Raiz no Numerador e Denominador}\\
Determine o domínio de $h(x) = \dfrac{\sqrt{x}}{x^2-4}$
\begin{tcolorbox}[colback=explanationbg,colframe=titleblue,title=Dica:]
1) Numerador: $\sqrt{x}$ requer $x \geq 0$; \\  2) Denominador: $x^2-4 \neq 0$.
\end{tcolorbox}

\item \textbf{Valor Absoluto no Denominador}\\
Encontre o domínio de $f(x) = \dfrac{1}{|x|-2}$
\begin{tcolorbox}[colback=explanationbg,colframe=titleblue,title=Dica:]
Todo número em módulo $|x|$ é positivo; \\
Resolva $|x|-2 \neq 0$.
\end{tcolorbox}

\item \textbf{Raiz Cúbica e Denominador}\\
Determine o domínio de $g(x) = \sqrt[3]{x^2-1} + \dfrac{1}{x}$
\begin{tcolorbox}[colback=explanationbg,colframe=titleblue,title=Dica:]
Raiz cúbica não tem restrição, mas $\dfrac{1}{x}$ requer $x \neq 0$.
\end{tcolorbox}

\section*{Desafio}

\item \textbf{Função com Raiz de Quociente}\\
Determine o domínio de $f(x) = \sqrt{\dfrac{x+1}{x-2}}$
\begin{tcolorbox}[colback=explanationbg,colframe=titleblue,title=Dica:]
Duas condições: 1) $\dfrac{x+1}{x-2} \geq 0$ e 2) $x-2 \neq 0$. Resolva a inequação racional.
\end{tcolorbox}

\item \textbf{Logaritmo com Argumento Quadrático}\\
Encontre o domínio de $g(x) = \ln(x^2-4)$
\begin{tcolorbox}[colback=explanationbg,colframe=titleblue,title=Dica:]
Argumento deve ser $>0$: $x^2-4>0$. Fatore e resolva a inequação quadrática.
\end{tcolorbox}

\item \textbf{Combinação Complexa}\\
Qual o domínio de $h(x) = \dfrac{\sqrt{x-3}}{\log_2(x-1)}$?
\begin{tcolorbox}[colback=explanationbg,colframe=titleblue,title=Dica:]
1) Numerador: $x-3 \geq 0$; 2) Denominador: $\log_2(x-1) \neq 0$ e $x-1>0$.
\end{tcolorbox}

\item \textbf{Raiz Quadrática}\\
Determine o maior domínio possível para $f(x) = \sqrt{9-x^2}$
\begin{tcolorbox}[colback=explanationbg,colframe=titleblue,title=Dica:]
Resolva $9-x^2 \geq 0$. Fatore como diferença de quadrados e analise o sinal.
\end{tcolorbox}

\item \textbf{Função por Partes}\\
Para a função:
\[ f(x) = \begin{cases} 
\dfrac{1}{x} & \text{se } x < 0 \\
\sqrt{x} & \text{se } x \geq 0 
\end{cases} \]
determine seu domínio
\begin{tcolorbox}[colback=explanationbg,colframe=titleblue,title=Dica:]
Analise cada parte separadamente: 1) $x<0$ com $x\neq0$ e 2) $x\geq0$.
\end{tcolorbox}

\item \textbf{Raiz com Valor Absoluto}\\
Determine o domínio de $f(x) = \dfrac{1}{\sqrt{|x|-3}}$
\begin{tcolorbox}[colback=explanationbg,colframe=titleblue,title=Dica:]
Duas condições: 1) $|x|-3 > 0$ (raiz no denominador) e 2) $|x|-3 \neq 0$.
\end{tcolorbox}

\item \textbf{Função Trigonométrica}\\
Encontre o domínio de $g(x) = \sqrt{\sin x}$ (considerando apenas $[0,2\pi]$)
\begin{tcolorbox}[colback=explanationbg,colframe=titleblue,title=Dica:]
$\sin x \geq 0$. Determine os intervalos em $[0,2\pi]$ onde isso ocorre.
\end{tcolorbox}

\item \textbf{Exponencial com Denominador}\\
Qual o domínio de $h(x) = \dfrac{e^x}{x^2-5x+6}$?
\begin{tcolorbox}[colback=explanationbg,colframe=titleblue,title=Dica:]
A exponencial não tem restrição, mas o denominador não pode ser zero. Fatore $x^2-5x+6$.
\end{tcolorbox}

\item \textbf{Logaritmo dentro de Raiz}\\
Determine o domínio de $f(x) = \sqrt{\log_{10}(x-1)}$
\begin{tcolorbox}[colback=explanationbg,colframe=titleblue,title=Dica:]
Duas condições: 1) $\log_{10}(x-1) \geq 0$ e 2) $x-1 > 0$.
\end{tcolorbox}

\item \textbf{Raiz Quarta com Denominador}\\
Encontre o domínio de $g(x) = \dfrac{\sqrt[4]{x-5}}{x^2-16}$
\begin{tcolorbox}[colback=explanationbg,colframe=titleblue,title=Dica:]
1) Raiz quarta: $x-5 \geq 0$; 2) Denominador: $x^2-16 \neq 0$.
\end{tcolorbox}

\end{enumerate}
\end{multicols}


\end{document}
