
\documentclass{beamer}
\usepackage[brazil]{babel}
\usepackage{amsmath, amssymb}
\usepackage{graphicx}

% Tema bonito para apresentação
\usetheme{Madrid}

\title{Gincana Matemática}
\subtitle{Adição, Subtração, Multiplicação, Potência, Operações com Frações, Conversão de Unidades e Notação Científica}
\author{Jefferson Bezerra dos Santos}
\date{1º Ano do Ensino Médio}

% Configuração do rodapé
\setbeamertemplate{footline}{
    \vspace{0.5em}
    \hspace{1em}
    \begin{minipage}{0.8\textwidth}
        \small
        \textbf{Mestre em Modelagem Matemática e Computacional - UFPB} \\
    \end{minipage}
    \hfill
    \begin{minipage}{0.1\textwidth}
        \centering
        \textcolor{blue}{\insertframenumber}
    \end{minipage}
    \vspace{0.5em}
}

\begin{document}

\frame{\titlepage}

% Slide de introdução
\begin{frame}{Introdução}
    \begin{itemize}
        \item Objetivo: Reforçar conceitos matemáticos de forma dinâmica e interativa.
        \item Conteúdos: Adição, Subtração, Multiplicação, Potência, Operações com Frações, Conversão de Unidades e Notação Científica.
        \item Duração: 2 aulas.
        \item Metodologia: Competição em equipes com desafios matemáticos \cite{silva2018, souza2020, almeida2019, santos2021, pereira2022}.
    \end{itemize}
\end{frame}

% Slide de organização
\begin{frame}{Organização da Gincana}
    \begin{itemize}
        \item Formação de equipes com 5 a 6 alunos.
        \item Cada rodada terá desafios com pontuações diferentes.
        \item Uso de quadro, cartazes e folhas para resolução.
        \item O grupo com mais pontos ao final vence.
    \end{itemize}
\end{frame}

% Slide de Rodada 1
\begin{frame}{Rodada 1: Adição e Subtração}
    \begin{itemize}
        \item Cada equipe resolverá 3 problemas de adição e subtração.
        \item Resolução correta: +10 pontos.
        \item Tempo máximo por questão: 1 minuto.
    \end{itemize}
\end{frame}

% Slide de Rodada 2
\begin{frame}{Rodada 2: Multiplicação e Potência}
    \begin{itemize}
        \item Desafios envolvendo multiplicações e potências.
        \item Algumas questões terão "desafio bônus" para pontuação extra.
        \item Resolução correta: +15 pontos.
    \end{itemize}
\end{frame}

% Slide de Rodada 3
\begin{frame}{Rodada 3: Operações com Frações}
    \begin{itemize}
        \item Cálculos envolvendo soma, subtração, multiplicação e divisão de frações.
        \item Equipe que responde primeiro corretamente ganha pontos extras.
        \item Resolução correta: +20 pontos.
    \end{itemize}
\end{frame}

% Slide de Rodada 4
\begin{frame}{Rodada 4: Conversão de Unidades}
    \begin{itemize}
        \item Problemas envolvendo conversão de unidades de medida (comprimento, massa, tempo, etc.).
        \item Resolução correta: +15 pontos.
        \item Questões bônus para conversões mais complexas.
    \end{itemize}
\end{frame}

% Slide de Rodada 5
\begin{frame}{Rodada 5: Notação Científica}
    \begin{itemize}
        \item Exercícios envolvendo escrita e operações com notação científica.
        \item Aplicação da notação científica em situações do cotidiano.
        \item Resolução correta: +15 pontos.
    \end{itemize}
\end{frame}

% Slide de Código de Conduta
\begin{frame}{Código de Conduta}
    \begin{itemize}
        \item Manter a sala organizada antes, durante e após a gincana.
        \item Jogar lixo no lixo e manter o ambiente limpo.
        \item Respeitar os colegas e colaborar para um ambiente saudável.
        \item Seguir as orientações do professor para um evento harmonioso.
    \end{itemize}
\end{frame}

% Slide de Encerramento
\begin{frame}{Encerramento e Premiação}
    \begin{itemize}
        \item Somatória de pontos das equipes.
        \item Anúncio da equipe vencedora.
        \item Premiação simbólica de brinde (bis ou equivalente).
        \item Reflexão sobre os conteúdos revisados.
    \end{itemize}
\end{frame}

% Slide de Referências
\begin{frame}{Referências}
    \begin{thebibliography}{99}
        \bibitem{silva2018} SILVA, J. Métodos lúdicos para o ensino da matemática. São Paulo: Editora Acadêmica, 2018.
        \bibitem{souza2020} SOUZA, M. Aprendizagem baseada em jogos na educação básica. Rio de Janeiro: Educação \& Tecnologia, 2020.
        \bibitem{almeida2019} ALMEIDA, R. Educação Matemática e suas metodologias. Curitiba: Editora Pedagógica, 2019.
        \bibitem{santos2021} SANTOS, P. Etnomatemática: cultura e aprendizagem. Fortaleza: Educação Global, 2021.
        \bibitem{pereira2022} PEREIRA, L. Matemática aplicada e suas conexões. Brasília: Editora Científica, 2022.
    \end{thebibliography}
\end{frame}

\end{document}
