\documentclass{beamer}
\usepackage[brazil]{babel}
\usepackage{amsmath, amssymb}

\usetheme{Madrid}

\title{Gincana Matemática}
\subtitle{Adição, Subtração, Multiplicação, Potência, Operações com Frações, Conversão de Unidades, Notação Científica e
Grandezas}
\date{1º Ano do Ensino Médio}

\begin{document}

\frame{\titlepage}

% Aula 1 - Rodada 1: Adição e Subtração
\begin{frame}{Rodada 1: Adição e Subtração}
    1.(Fácil) João tem R\$ 35,00 e ganha mais R\$ 27,00. Quanto ele tem agora?
    \begin{itemize}
        \item[a)] R\$ 50,00
        \item[b)] \alert<2->{R\$ 62,00}
        \item[c)] R\$ 58,00
        \item[d)] R\$ 63,00
        \item[e)] R\$ 61,00
    \end{itemize}
\end{frame}

\begin{frame}{Rodada 1: Adição e Subtração}
    2.(Médio) Um atleta percorreu 12,5 km pela manhã e 8,7 km à tarde. Qual foi a distância total percorrida?
    \begin{itemize}
        \item[a)] 19,8 km
        \item[b)] 20,5 km
        \item[c)] \alert<2->{21,2 km}
        \item[d)] 18,6 km
        \item[e)] 22,1 km
    \end{itemize}
\end{frame}

\begin{frame}{Rodada 2: Multiplicação e Potência}
    3.(Fácil) Qual o resultado de $7 \times 8$?
    \begin{itemize}
        \item[a)] 42
        \item[b)] \alert<2->{56}
        \item[c)] 54
        \item[d)] 62
        \item[e)] 48
    \end{itemize}
\end{frame}

\begin{frame}{Rodada 3: Notação Científica}
    4.(Fácil) Qual a notação científica de $5.000.000$?
    \begin{itemize}
        \item[a)] $5 \times 10^3$
        \item[b)] $5 \times 10^5$
        \item[c)] \alert<2->{$5 \times 10^6$}
        \item[d)] $5.0 \times 10^7$
        \item[e)] $50 \times 10^5$
    \end{itemize}
\end{frame}

\begin{frame}{Rodada 3: Notação Científica}
    5.(Fácil) Qual a ordem de grandeza de $7.000.000$?
    \begin{itemize}
        \item[a)] $10^3$
        \item[b)] $10^5$
        \item[c)] $10^6$
        \item[d)] $10^8$
        \item[e)] \alert<2->{$10^7$}
    \end{itemize}
\end{frame}


\begin{frame}{Rodada 3: Notação Científica}
    6.(Médio) Qual a notação científica de $0,000032$?
    \begin{itemize}
        \item[a)] $3.2 \times 10^{-4}$
        \item[b)] $3.2 \times 10^{-3}$
        \item[c)] \alert<2->{$3.2 \times 10^{-5}$}
        \item[d)] $32 \times 10^{-6}$
        \item[e)] $0.32 \times 10^{-3}$
    \end{itemize}
\end{frame}
\begin{frame}{Rodada 4: Grandezas }
    7.(Fácil) Se 3 funcionários levam 6 horas para concluir um serviço, quanto tempo levariam 6 funcionários, mantendo a produtividade?
    \begin{itemize}
        \item[a)] 2 horas
        \item[b)] \alert<2>{3 horas}
        \item[c)] 6 horas
        \item[d)] 9 horas
        \item[e)] 12 horas
    \end{itemize}
\end{frame}

\begin{frame}{Rodada 5: Grandezas }
    8.(Médio) Um carro a 60 km/h leva 4 horas para percorrer um trajeto. Quanto tempo levaria se estivesse a 80 km/h?
    \begin{itemize}
        \item[a)] 3 horas
        \item[b)] 3.5 horas
        \item[c)] \alert<2>{3 horas}
        \item[d)] 5 horas
        \item[e)] 2.5 horas
    \end{itemize}
\end{frame}

\begin{frame}{Rodada 5: Grandezas }
    9.(Médio) Uma fábrica produz 1700 peças em 8 horas de trabalho. Quantas peças seriam produzidas em 12 horas, mantendo o mesmo ritmo de produção?
    \begin{itemize}
        \item[a)] 2300 peças
        \item[b)] 2600 peças
        \item[c)] \alert<2>{2550 peças}
        \item[d)] 2700 peças
        \item[e)] 2500 peças
    \end{itemize}
\end{frame}

\begin{frame}{Rodada 5: Grandezas }
    10.(Difícil) Um carro percorre 300 km com 20 litros de combustível. Quantos litros de combustível serão necessários
    para percorrer 450 km, se o carro passar a consumir 20\% mais combustível por quilômetro devido a uma mudança no tipo de estrada?

\begin{itemize}
\item[a)] 35 litros
\item[b)] 38 litros
\item[c)] \alert<2>{36 litros}
\item[d)] 40 litros
\item[e)] 42 litros
\end{itemize}
\end{frame}

\begin{frame}{Rodada 6: Regra de três }
    11.(Fácil) Um vendedor recebe 5\% de comissão sobre suas vendas. Se ele vendeu R\$ 12.000,00, quanto recebeu de comissão?

\begin{itemize}
    \item[a)] R\$ 500,00
    \item[b)] R\$ 900,00
    \item[c)] \alert<2>{R\$ 600,00}
    \item[d)] R\$ 700,00
    \item[e)] R\$ 550,00
\end{itemize}

\end{frame}


\begin{frame}{Rodada 6: Regra de três  }
    12.(Fácil) Um produto custa R\$ 200,00. Se ele sofre um aumento de 15\%, qual será seu novo preço?

\begin{itemize}
    \item[a)] R\$ 230,00
    \item[b)] R\$ 220,00
    \item[c)] \alert<2>{R\$ 230,00}
    \item[d)] R\$ 240,00
    \item[e)] R\$ 210,00
\end{itemize}
\end{frame}

\begin{frame}{Rodada 6: Regra de três}
    13,(Fácil) Um investimento de R\$ 5.000,00 rendeu 8\% em um ano. Qual foi o valor do rendimento?

\begin{itemize}
    \item[a)] \alert<2>{R\$ 400,00}
    \item[b)] R\$ 450,00
    \item[c)] R\$ 490,00
    \item[d)] R\$ 500,00
    \item[e)] R\$ 350,00
\end{itemize}
\end{frame}

\begin{frame}{Rodada 6: Regra de três }
    14.(Médio) Um carro foi vendido com um desconto de 12\% sobre o preço de R\$ 25.000,00. Qual foi o valor pago pelo carro?

\begin{itemize}
    \item[a)] R\$ 22.900,00
    \item[b)] R\$ 22.500,00
    \item[c)] \alert<2>{R\$ 22.000,00}
    \item[d)] R\$ 23.000,00
    \item[e)] R\$ 21.500,00
\end{itemize}

\end{frame}

\begin{frame}{Rodada 6: Regra de três }
    15.(Difícil) Um trabalhador recebeu um aumento de 10\% no salário, passando a ganhar R\$ 2.200,00. Qual era o salário antes do aumento?

\begin{itemize}
    \item[a)] R\$ 2.200,00
    \item[b)] R\$ 1.980,00
    \item[c)] R\$ 2.100,00
    \item[d)] \alert<2>{R\$ 2.000,00}
    \item[e)] R\$ 1.950,00
\end{itemize}

\end{frame}





\end{document}

