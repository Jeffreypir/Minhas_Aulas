\documentclass[12pt]{article}
\usepackage[utf8]{inputenc}
\usepackage{amsmath}
\usepackage{multicol}
\usepackage{geometry}
\usepackage{tikz}
\usetikzlibrary{arrows.meta}
\usepackage{enumitem} % Para listas personalizadas
\usepackage{xcolor} % Para usar cores
\usepackage{titlesec} % Para personalizar títulos


\geometry{a4paper, left=1cm, right=1cm, top=1.5cm, bottom=1.5cm}

% Definir a cor das seções como azul
\titleformat{\section}
  {\normalfont\Large\bfseries\color{blue}} % Formato do título
  {\thesection} % Número da seção
  {1em} % Espaço entre número e título
  {} % Código antes do título

\renewcommand{\thesubsection}{\textcolor{red}{\arabic{section}.\arabic{subsection}}}
\titleformat{\subsection}{\color{red}\normalfont\bfseries}{\thesubsection}{1em}{}
\title{\textcolor{blue}{Grandezas:Regra de Três Composta}}
\author{Professor: Jefferson}
\date{}


\begin{document}

\maketitle

\begin{center}
\large{Nome: \underline{\hspace{8cm}} \quad Série-Turma: \underline{\hspace{3cm}}}
\end{center}


\begin{multicols}{2}

\section*{Introdução à Regra de Três Composta}

\subsection*{O que é Regra de Três Composta?}
A regra de três composta é utilizada quando temos três ou mais grandezas relacionadas, que podem ser direta ou inversamente proporcionais. Diferente da regra de três simples, que envolve apenas duas grandezas, a composta exige análise das relações entre todas as grandezas envolvidas.

\subsection*{Passos para Resolver}
1. Identifique todas as grandezas envolvidas. \newline
2. Verifique se são direta ou inversamente proporcionais. \newline  
3. Monte a proporção isolando a grandeza de interesse  e resolva a equação.

\subsection*{Exemplo 1:}
\textbf{Problema:} Se 4 operários, trabalhando 6 horas por dia, constroem um muro em 10 dias, quantos dias levarão 6 operários, trabalhando 8 horas por dia, para construir o mesmo muro?

\textbf{Esquema de Proporcionalidade:}
\begin{itemize}
    \item Operários (O) $\rightarrow$ Dias (D): \textbf{Inversamente proporcional} (↑ Operários, ↓ Dias).
    \item Horas por dia (H) $\rightarrow$ Dias (D): \textbf{Inversamente proporcional} (↑ Horas, ↓ Dias).
\end{itemize}

\textbf{Resolução:}
1. Montamos o esquema:
\[
\frac{O_1}{O_2} \times \frac{H_1}{H_2} = \frac{D_1}{D_2}
\]
2. Substituímos os valores e analisamos a proporção:
\[
    \underbrace{\frac{4}{6}}_{inversamente} \times \underbrace{\frac{6}{8}}_{inversamente} = \frac{10}{x}
\]

3. Organizamos a proporção e resolvemos:

\[
    \frac{6}{4} \times \frac{8}{6} = \frac{10}{x}
\]

\[
\frac{2}{1} = \frac{10}{x} \Rightarrow 2x = 10  \Rightarrow x = 5 \text{ dias.}
\]

\subsection*{Exemplo 2:}
\textbf{Problema:} Se 5 máquinas, trabalhando 8 horas por dia, produzem 200 peças em 4 dias, quantas peças serão produzidas por 7 máquinas, trabalhando 6 horas por dia, em 10 dias?

\textbf{Esquema de Proporcionalidade:}
\begin{itemize}
    \item Máquinas (M) $\rightarrow$ Peças (P): \textbf{Diretamente proporcional} (↑ Máquinas, ↑ Peças).
    \item Horas por dia (H) $\rightarrow$ Peças (P): \textbf{Diretamente proporcional} (↑ Horas, ↑ Peças).
    \item Dias (D) $\rightarrow$ Peças (P): \textbf{Diretamente proporcional} (↑ Dias, ↑ Peças).
\end{itemize}

\textbf{Resolução:}
1. Montamos o esquema:
\[
\frac{M_1}{M_2} \times \frac{H_1}{H_2} \times \frac{D_1}{D_2} = \frac{P_1}{P_2}
\]
2. Substituímos os valores e analisamos a proporção:

\[
\frac{5}{7} \times \frac{8}{6} \times \frac{4}{10} = \frac{200}{x}
\]
3. Organizamos a proporção e resolvemos::
\[
    \underbrace{\frac{5}{7}}_{diretamente} \times \underbrace{\frac{8}{6}}_{diretamente}
    \times\underbrace{\frac{4}{10}}_{diretamente} = \frac{200}{x}
\]

\[
\frac{160}{420} = \frac{200}{x} \Rightarrow x = 525 \text{ peças.}
\]

\section*{Aplicações Práticas da Regra de Três Composta}

\subsection*{Exemplo Prático 1}
\textbf{Problema:} Uma fábrica produz 1200 unidades de um produto em 5 dias, trabalhando 8 horas por dia com 10 máquinas. Quantas unidades serão produzidas em 7 dias, trabalhando 10 horas por dia com 12 máquinas?

\textbf{Esquema de Proporcionalidade:}
\begin{itemize}
    \item Máquinas (M) $\rightarrow$ Unidades (U): \textbf{Diretamente proporcional} (↑ Máquinas, ↑ Unidades).
    \item Horas por dia (H) $\rightarrow$ Unidades (U): \textbf{Diretamente proporcional} (↑ Horas, ↑ Unidades).
    \item Dias (D) $\rightarrow$ Unidades (U): \textbf{Diretamente proporcional} (↑ Dias, ↑ Unidades).
\end{itemize}

\textbf{Resolução:}
1. Montamos a proporção:
\[
\frac{M_1}{M_2} \times \frac{H_1}{H_2} \times \frac{D_1}{D_2} = \frac{U_1}{U_2}
\]
2. Substituímos os valores:
\[
\frac{10}{12} \times \frac{8}{10} \times \frac{5}{7} = \frac{1200}{x}
\]
3. Resolvemos:
\[
\frac{400}{840} = \frac{1200}{x} \Rightarrow x = 2520 \text{ unidades.}
\]

\subsection*{Exemplo Prático 2}
\textbf{Problema:} Um carro consome 40 litros de combustível para percorrer 500 km com 4 passageiros. Quantos litros serão necessários para percorrer 750 km com 5 passageiros, considerando que o consumo aumenta proporcionalmente com o número de passageiros?

\textbf{Esquema de Proporcionalidade:}
\begin{itemize}
    \item Passageiros (P) $\rightarrow$ Litros (L): \textbf{Diretamente proporcional} (↑ Passageiros, ↑ Litros).
        \newline

    \item Distância (D) $\rightarrow$ Litros (L): \textbf{Diretamente proporcional} (↑ Distância, ↑ Litros).
\end{itemize}

\textbf{Resolução:}
1. Montamos a proporção:
\[
\frac{P_1}{P_2} \times \frac{D_1}{D_2} = \frac{L_1}{L_2}
\]
2. Substituímos os valores:
\[
\frac{4}{5} \times \frac{500}{750} = \frac{40}{x}
\]
3. Resolvemos:
\[
\frac{2000}{3750} = \frac{40}{x} \Rightarrow x = 75 \text{ litros.}
\]

\subsection*{Atividade}
\textbf{Problema 1:} Uma equipe de 6 pedreiros constrói uma casa em 60 dias, trabalhando 8 horas por dia. Quantos dias serão necessários para que 8 pedreiros, trabalhando 6 horas por dia, construam a mesma casa?

\textbf{Esquema de Proporcionalidade:}
\begin{itemize}
    \item Pedreiros (P) $\rightarrow$ Dias (D): \textbf{Inversamente proporcional} (↑ Pedreiros, ↓ Dias).
    \item Horas por dia (H) $\rightarrow$ Dias (D): \textbf{Inversamente proporcional} (↑ Horas, ↓ Dias).
\end{itemize}

\textbf{Resolução:}
1. Montamos a proporção:
\[
\frac{P_1}{P_2} \times \frac{H_1}{H_2} = \frac{D_2}{D_1}
\]
2. Substituímos os valores:
\[
\frac{6}{8} \times \frac{8}{6} = \frac{x}{60}
\]
3. Resolvemos:
\[
\frac{48}{48} = \frac{x}{60} \Rightarrow x = 60 \text{ dias.}
\]

\textbf{Problema 2:} Uma torneira enche um tanque em 12 horas com uma vazão de 5 litros por minuto. Se a vazão for aumentada para 8 litros por minuto, em quanto tempo o tanque estará cheio?

\textbf{Esquema de Proporcionalidade:}
\begin{itemize}
    \item Vazão (V) $\rightarrow$ Tempo (T): \textbf{Inversamente proporcional} (↑ Vazão, ↓ Tempo).
\end{itemize}

\textbf{Resolução:}
1. Montamos a proporção:
\[
\frac{V_1}{V_2} = \frac{T_2}{T_1}
\]
2. Substituímos os valores:
\[
\frac{5}{8} = \frac{x}{12}
\]
3. Resolvemos:
\[
x = \frac{5 \times 12}{8} = 7,5 \text{ horas.}
\]

\end{multicols}

\end{document}
