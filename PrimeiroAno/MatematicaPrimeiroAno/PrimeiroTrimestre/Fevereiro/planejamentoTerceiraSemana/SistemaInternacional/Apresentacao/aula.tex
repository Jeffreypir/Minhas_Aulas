\documentclass[12pt]{beamer}
\usepackage[utf8]{inputenc}
\usepackage[brazil]{babel}
\usepackage{amsmath} % Para fórmulas matemáticas
\usepackage{xcolor} % Para usar cores

\usetheme{Berkeley}

% Definir cores para títulos e subtítulos
\setbeamercolor{frametitle}{fg=white}
\setbeamercolor{framesubtitle}{fg=green}

\title{\textcolor{white}{Unidades de Medida, Sistema Internacional (SI) e Conversão de Unidades}}
\author{Professor: Jefferson}
\date{}

\begin{document}

\frame{\titlepage}

% Slide do sumário
\begin{frame}
    \frametitle{Sumário}
    \tableofcontents
\end{frame}

\begin{frame}
    \section{Introdução às Unidades de Medida}
    \frametitle{Introdução às Unidades de Medida}
    \framesubtitle{O que são Unidades de Medida?}
    \begin{itemize}
        \item Unidades de medida são padrões utilizados para quantificar grandezas físicas.
        \item Exemplos: comprimento, massa, tempo, temperatura.
    \end{itemize}

    \section{Grandezas Fundamentais e Derivadas}
    \framesubtitle{Grandezas Fundamentais e Derivadas}

    \begin{itemize}
        \item \textbf{Grandezas Fundamentais}: Comprimento (metro), massa (quilograma), tempo (segundo).
        \item \textbf{Grandezas Derivadas}: Área (metro quadrado), velocidade (metro por segundo).
    \end{itemize}
\end{frame}

\begin{frame}
\frametitle{Exemplos de Grandezas Fundamentais}
\framesubtitle{Grandezas fundamentais são as bases do sistema de medição}

\begin{enumerate}
    \item \textbf{Comprimento}:
    \begin{itemize}
        \item \textbf{Definição}: Mede a distância entre dois pontos.
        \item \textbf{Unidade}: Metro (m).
        \item \textbf{Exemplo}: 5 metros (m).
    \end{itemize}

    \item \textbf{Massa}:
    \begin{itemize}
        \item \textbf{Definição}: Mede a quantidade de matéria em um objeto.
        \item \textbf{Unidade}: Quilograma (kg).
        \item \textbf{Exemplo}: 2 quilogramas (kg).
    \end{itemize}

    \item \textbf{\textbf{Tempo}}:
    \begin{itemize}
        \item \textbf{Definição}: Mede a duração de eventos.
        \item \textbf{Unidade}: Segundo (s).
        \item \textbf{Exemplo}: 10 segundos (s).
    \end{itemize}
\end{enumerate}
\end{frame}

\begin{frame}
\frametitle{Exemplos de Grandezas Derivadas}
\framesubtitle{Grandezas derivadas são combinações de grandezas fundamentais}

\begin{enumerate}
    \item \textbf{Velocidade}:
    \begin{itemize}
        \item \textbf{Definição}: Mede a distância percorrida por unidade de tempo.
        \item \textbf{Unidade}: Metro por segundo (m/s) ou quilômetro por hora (km/h).
        \item \textbf{Exemplo}: 60 km/h.
    \end{itemize}

    \item \textbf{Área}:
    \begin{itemize}
        \item \textbf{Definição}: Mede a extensão de uma superfície.
        \item \textbf{Unidade}: Metro quadrado (m²).
        \item \textbf{Exemplo}: \( 10 \, \text{m} \times 5 \, \text{m} = 50 \, \text{m}^2 \).
    \end{itemize}

    \item \textbf{Densidade}:
    \begin{itemize}
        \item \textbf{Definição}: Mede a massa por unidade de volume.
        \item \textbf{Unidade}: Quilograma por metro cúbico (kg/m³).
        \item \textbf{Exemplo}: \( 1000 \, \text{kg/m}^3 \).
    \end{itemize}
\end{enumerate}
\end{frame}

\begin{frame}
    \frametitle{Atividade 1}
    \begin{enumerate}
        \item Liste três grandezas físicas e suas respectivas unidades de medida.
        \item Explique a diferença entre grandezas fundamentais e derivadas.
    \end{enumerate}
\end{frame}
% Frame para Resolução 1
\begin{frame}
\frametitle{Resolução 1}
\framesubtitle{Liste três grandezas físicas e suas respectivas unidades de medida}

Três grandezas físicas e suas unidades de medida são:
\begin{itemize}
    \item \textbf{Comprimento}: Metro (m).
    \item \textbf{Massa}: Quilograma (kg).
    \item \textbf{Tempo}: Segundo (s).
\end{itemize}
\end{frame}

% Frame para Resolução 2
\begin{frame}
\frametitle{Resolução 2}
\framesubtitle{Explique a diferença entre grandezas fundamentais e derivadas}

\begin{itemize}
    \item \textbf{Grandezas Fundamentais}:
    São as grandezas básicas, que servem como base para definir todas as outras. Elas são independentes e não podem ser expressas em termos de outras grandezas. Exemplos incluem:
    \begin{itemize}
        \item Comprimento (metro).
        \item Massa (quilograma).
    \end{itemize}

    \item \textbf{Grandezas Derivadas}:
    São grandezas que são combinações das grandezas fundamentais. Elas são definidas a partir das grandezas fundamentais por meio de relações matemáticas. Exemplos incluem:
    \begin{itemize}
        \item Área (metro quadrado, \( \text{m}^2 \)), que é derivada do comprimento.
        \item Velocidade (metro por segundo, \( \text{m/s} \)), que é derivada do comprimento e do tempo.
    \end{itemize}
\end{itemize}
\end{frame}
\begin{frame}
    \section{Sistema Internacional de Unidades (SI)}
    \frametitle{Sistema Internacional de Unidades (SI)}
    \framesubtitle{O que é o SI?}
    \begin{itemize}
        \item O SI é o padrão global de unidades de medida.
        \item Adotado pela maioria dos países.
    \end{itemize}

    \framesubtitle{Unidades Básicas do SI}
    \begin{itemize}
        \item Comprimento: Metro (m).
        \item Massa: Quilograma (kg).
        \item Tempo: Segundo (s).
        \item Corrente elétrica: Ampere (A).
        \item Temperatura: Kelvin (K).
        \item Quantidade de matéria: Mol (mol).
        \item Intensidade luminosa: Candela (cd).
    \end{itemize}

    \framesubtitle{Exemplo}
    \begin{itemize}
        \item 1 km = 1000 m.
        \item 1 kg = 1000 g.
        \item 1 h = 3600 s.
    \end{itemize}
\end{frame}
\begin{frame}
    \frametitle{Exemplo:}
    % Exemplo de regra de três simples
    Converter 3 km para metros: 
    \begin{align*}
        1 \, \text{km} & \rightarrow 1000 \, \text{m}, \\
        3 \, \text{km} & \rightarrow x \, \text{m}. \\
        \\
        \\
    \phantom{} \\ % Linha em branco 1
    \phantom{} \\ % Linha em branco 2
    \phantom{} \\ % Linha em branco 3
    \phantom{} \\ % Linha em branco 4
    \end{align*}
\end{frame}


\begin{frame}
    \frametitle{Atividade 2:}
    \begin{enumerate}
        \item Converta 2,5 km para metros.
        \item Quantos segundos existem em 1,5 horas?
        \item Explique por que o SI é importante para a ciência e a tecnologia.
    \end{enumerate}

\end{frame}

\begin{frame}
    \frametitle{Resolução:}
        1. Converta 2,5 km para metros.
        \vspace{6cm}
    
\end{frame}

\begin{frame}
    \frametitle{Resolução:}
        2. Quantos segundos existem em 1,5 horas?
        \vspace{6cm}
    
\end{frame}

\begin{frame}
    \frametitle{Resolução:}
        3. Explique por que o SI é importante para a ciência e a tecnologia.
        \vspace{6cm}
    
\end{frame}

\begin{frame}
    \section{Conversão de Unidades}
    \frametitle{Conversão de Unidades}
    \textcolor{green}{Por que Converter Unidades?}
    \begin{itemize}
        \item Padronizar medições.
        \item Facilitar a comunicação entre diferentes sistemas de medida.
    \end{itemize}

    \framesubtitle{Fatores de Conversão}
    \begin{itemize}
        \item 1 km = 1000 m.
        \item 1 m = 100 cm.
        \item 1 kg = 1000 g.
        \item 1 h = 60 min = 3600 s.
    \end{itemize}

\end{frame}
\begin{frame}
    \frametitle{Atividade 3:}
    \begin{enumerate}
        \item Converta 4500 g para quilogramas.
        \item Converta 2,5 horas para segundos.
        \item Explique como você faria para converter 1,2 m para centímetros.
    \end{enumerate}
\end{frame}

\begin{frame}
\frametitle{Resolução:}
1. Converta 4500 g para quilogramas.
\vspace{6cm}
\end{frame}

\begin{frame}
2. Converta 2,5 horas para segundos.
\vspace{6cm}
\end{frame}

\begin{frame}
3. Explique como você faria para converter 1,2 m para centímetros.
\vspace{6cm}
\end{frame}



\begin{frame}
    \section{Aplicações Práticas do SI e Conversões}
    \frametitle{Aplicações Práticas do SI e Conversões}
    \framesubtitle{Aplicações do SI no Cotidiano}
    \begin{itemize}
        \item Engenharia, medicina, comércio e ciência.
        \item Exemplos:
            \begin{itemize}
                \item Medição de distâncias em mapas (km).
                \item Dosagem de medicamentos (mg ou mL).
                \item Controle de tempo em competições esportivas (s).
            \end{itemize}
    \end{itemize}

    \framesubtitle{Exemplo}
    \begin{itemize}
        \item Um carro percorre 120 km em 2 horas. Qual é a sua velocidade média em m/s?
            \begin{align*}
                \text{Velocidade} &= \frac{\text{Distância}}{\text{Tempo}} \\
                &= \frac{120 \, \text{km}}{2 \, \text{h}} = 60 \, \text{km/h} \\
                &= 60 \times \frac{1000 \, \text{m}}{3600 \, \text{s}} \approx 16,67 \, \text{m/s}.
            \end{align*}
    \end{itemize}
\end{frame}

\begin{frame}
    \frametitle{Atividade 4}
    \begin{enumerate}
        \item Um avião voa a 900 km/h. Converta essa velocidade para m/s.
        \item Uma caixa contém 2,5 kg de arroz. Quantos gramas de arroz há na caixa?
        \item Pesquise uma aplicação do SI em sua comunidade e descreva como ele é utilizado.
    \end{enumerate}
\end{frame}

\begin{frame}
\frametitle{Resolução:}
1. Um avião voa a 900 km/h. Converta essa velocidade para m/s.
\vspace{6cm}
\end{frame}

\begin{frame}
2. Uma caixa contém 2,5 kg de arroz. Quantos gramas de arroz há na caixa?
\vspace{6cm}
\end{frame}

\begin{frame}
3. Pesquise uma aplicação do SI em sua comunidade e descreva como ele é utilizado.
\vspace{6cm}
\end{frame}

\begin{frame}
\section{Referências}
\frametitle{Referências}
\framesubtitle{Livros e materiais utilizados}

\begin{itemize}
    \item IEZZI, Gelson et al. \textbf{Matemática: Ciência e Aplicações}. 9ª ed. São Paulo: Saraiva, 2017.

    \item IEZZI, Gelson et al. \textbf{Fundamentos de Matemática Elementar}. Volume 1. São Paulo: Atual Editora, 2013.
    \item IEZZI, Gelson; DOLCE, Osvaldo; DEGENSZAJN, David; PÉRIGO, Roberto; ALMEIDA, Nilze. \textbf{Matemática: Volume Único}. São Paulo: Atual Editora, 2013.
    \item TIPLER, Paul A.; MOSCA, Gene. \textbf{Física para Cientistas e Engenheiros}. 6ª ed. Rio de Janeiro: LTC, 2014.
\end{itemize}
\end{frame}

% Slide de título (primeiro slide)
\begin{frame}
\titlepage
\end{frame}

% Slide de encerramento (último slide)
\begin{frame}
\begin{center}
    \textbf{\textcolor{blue}{\Large Obrigado pela atenção!}} \\[0.5cm]
    \small{Professor: Jefferson} \\
\end{center}
\end{frame}
\end{document}
