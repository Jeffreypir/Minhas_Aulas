\documentclass[12pt]{article}
\usepackage[utf8]{inputenc}
\usepackage[brazil]{babel}
\usepackage{geometry}
\geometry{a4paper, left=2cm, right=2cm, top=2cm, bottom=2cm}
\usepackage{amsmath} % Para fórmulas matemáticas
\usepackage{multicol} % Para dividir a página em colunas
\usepackage{enumitem} % Para listas personalizadas
\usepackage{xcolor} % Para usar cores
\usepackage{titlesec} % Para personalizar títulos

% Definir a cor das seções como azul
\titleformat{\section}
  {\normalfont\Large\bfseries\color{blue}} % Formato do título
  {\thesection} % Número da seção
  {1em} % Espaço entre número e título
  {} % Código antes do título

\title{\textcolor{blue}{Atividade: Unidades de Medida, Sistema Internacional (SI) e Conversão de Unidades}}
\author{Professor: Jefferson}
\date{}

% Remove a numeração de todas as páginas
\pagestyle{empty}

\begin{document}

\maketitle
\thispagestyle{empty}

\begin{center}
\large{Nome: \underline{\hspace{8cm}} \quad Série-Turma: \underline{\hspace{3cm}}}
\end{center}

\vspace{1cm}

\begin{multicols}{2}

\section*{Questões}

1. (ENEM) Um medicamento deve ser administrado na proporção de 1,5 mg por quilograma de massa corporal. Se uma pessoa
    pesa 70 kg, qual é a quantidade de medicamento que ela deve tomar ?\newline

2. (SAEPE) Um avião voa a uma velocidade de 900 km/h. Qual é a sua velocidade em m/s ?\newline  

3. (ENEM) Uma piscina tem 12 metros de comprimento, 6 metros de largura e 1,5 metros de profundidade. Qual é o volume de
    água, em litros, necessário para encher completamente essa piscina ? (\textbf{Considere que 1 m³ = 1000
    litros.}) \newline 

4. (SAEPE) Um carro percorre 180 km em 2 horas. Qual é a sua velocidade média em m/s ?\newline  

5. (ENEM) Uma caixa contém 2,5 kg de arroz. Quantos gramas de arroz há na caixa ?\newline  

6. (SAEPE) Um recipiente tem capacidade para 5 litros de água. Quantos mililitros (mL) de água cabem nesse recipiente ?\newline  

7. (ENEM) Um estudante mediu o comprimento de uma mesa e encontrou 1,2 metros. Qual é o comprimento da mesa em centímetros ?\newline  

8. (SAEPE) Uma torneira despeja 2 litros de água por minuto. Quantos litros de água serão despejados em 1 hora ?\newline  

9. (ENEM) Um terreno retangular tem 50 metros de comprimento e 30 metros de largura. Qual é a área do terreno em hectares ?\newline (Considere que 1 hectare = 10.000 m².)  

10. (SAEPE) Um recipiente tem capacidade para 3,5 litros de água. Quantos mililitros (mL) de água cabem nesse recipiente ?\newline  

\end{multicols}

\newpage % Página para o gabarito

\end{document}
