\documentclass[11pt]{article}
\usepackage[brazil]{babel}
\usepackage[utf8]{inputenc}
\usepackage[T1]{fontenc}
\usepackage{amsmath}
\usepackage{multicol}
\usepackage{geometry}
\usepackage{tikz}
\usepackage{enumitem}
\usepackage{xcolor}
\usepackage{titlesec}
\usepackage{graphicx}
\usepackage{hyperref}

% Configurações de layout
\geometry{a4paper, left=2cm, right=2cm, top=2cm, bottom=2cm}
\setlength{\columnseprule}{0.4pt}

% Cores personalizadas
\definecolor{titleblue}{RGB}{0,80,150}
\definecolor{sectionred}{RGB}{180,0,0}
\definecolor{darkgreen}{RGB}{0,100,0}
\definecolor{examplebg}{RGB}{240,248,255}

% Formatação de títulos
\titleformat{\section}{\normalfont\Large\bfseries\color{titleblue}}{\thesection}{1em}{}
\titleformat{\subsection}{\normalfont\large\bfseries\color{sectionred}}{\thesubsection}{1em}{}
\titleformat{\subsubsection}{\normalfont\normalsize\bfseries\color{darkgreen}}{\thesubsubsection}{1em}{}

\title{\textcolor{titleblue}{Apostila de Matemática - Revisão do Primeiro Trimestre}}
\author{Professor: Jefferson}
\date{}

\begin{document}

\maketitle

\tableofcontents

\section{Introdução}
Esta apostila contém uma revisão completa de matemática básica do primeiro trimestre através de 10 questões resolvidas detalhadamente, abordando os seguintes tópicos:

\begin{itemize}
\item Notação científica e potências de 10
\item Razões, proporções e escalas
\item Equações do 1º grau e problemas algébricos
\item Regra de três simples e composta
\item Geometria básica e cálculos de perímetro
\item Relações de proporcionalidade
\end{itemize}

\section{Notação Científica}

\subsection{Conceito Teórico}
A notação científica é uma forma de escrever números muito grandes ou muito pequenos utilizando potências de 10. Um número está em notação científica quando é escrito na forma $a \times 10^n$, onde $1 \leq a < 10$ e $n$ é um número inteiro.

\subsection{Questão 1}
\subsubsection*{Enunciado}
A distância média da Terra ao Sol é de aproximadamente 149.600.000 km. Represente este valor em notação científica.

\subsubsection*{Resolução Passo a Passo}
\begin{enumerate}
\item Valor original: 149.600.000 km
\item Movemos a vírgula para depois do primeiro dígito: 1,496
\item Contamos quantas casas decimais movemos: 8 casas
\item Portanto: $1,496 \times 10^8$ km
\end{enumerate}

\subsubsection*{Resposta}
$\boxed{1,496 \times 10^8\ \text{km}}$

\subsection{Questão 2}
\subsubsection*{Enunciado}
O diâmetro de um átomo de hidrogênio é aproximadamente $1,06 \times 10^{-10}$ metros, enquanto o diâmetro de uma bola de futebol é aproximadamente $2,2 \times 10^{-1}$ metros. Quantas vezes a bola de futebol é maior que o átomo?

\subsubsection*{Resolução}
\[
\frac{2,2 \times 10^{-1}}{1,06 \times 10^{-10}} = \frac{2,2}{1,06} \times 10^{-1-(-10)} \approx 2,075 \times 10^9
\]

\subsubsection*{Resposta}
$\boxed{2,075 \times 10^9}$ vezes (aproximadamente 2 bilhões de vezes maior)

\section{Geometria Básica - Perímetro}

\subsection{Conceito Teórico}
O perímetro é a medida do contorno de uma figura geométrica. Para um retângulo, o perímetro é calculado por $P = 2 \times (comprimento + largura)$. Conhecer o perímetro é essencial em situações como calcular a quantidade de material necessário para cercar um terreno.

\subsection{Questão 3}
\subsubsection*{Enunciado}
Um terreno retangular tem perímetro de 80 metros. Sabendo que a largura é 3/5 do comprimento, determine as medidas deste terreno.

\subsubsection*{Resolução}
\begin{enumerate}
\item Seja $x$ = comprimento
\item Então largura = $\frac{3}{5}x$
\item Perímetro: $2(x + \frac{3}{5}x) = 80$
\item $2(\frac{8}{5}x) = 80$
\item $\frac{16}{5}x = 80$
\item $x = \frac{80 \times 5}{16} = 25$ m
\item Largura: $\frac{3}{5} \times 25 = 15$ m
\end{enumerate}

\subsubsection*{Resposta}
$\boxed{\text{Comprimento} = 25\ \text{m}, \text{Largura} = 15\ \text{m}}$

\section{Equações do 1º Grau}

\subsection{Conceito Teórico}
Uma equação do primeiro grau é uma igualdade algébrica que pode ser reduzida à forma $ax + b = 0$, onde $a$ e $b$ são constantes e $x$ é a incógnita. Para resolver:
\begin{enumerate}
\item Elimine parênteses (usando a distributiva)
\item Isole os termos com $x$ de um lado da equação
\item Simplifique e encontre o valor de $x$
\end{enumerate}

\subsection{Questão 4}
\subsubsection*{Enunciado}
Resolva a equação: $5(x - 3) + 5 = 0$

\subsubsection*{Passo a Passo}
\begin{align*}
5(x - 3) + 5 &= 0\\
5(x - 3)  &= -5 \\
(x - 3) &= \frac{-5}{5} \\
(x - 3) &= -1 \\
x &= -1 + 3 \\
x &= 2 
\end{align*}

\subsubsection*{Resposta}
$\boxed{2}$

\subsection{Questão 5}
\subsubsection*{Enunciado}
A soma de quatro números pares consecutivos é 76. Determine esses números.

\subsubsection*{Resolução}
\begin{enumerate}
\item Seja $x$ o primeiro número par
\item Então: $x + (x+2) + (x+4) + (x+6) = 76$
\item $4x + 12 = 76$
\item $4x = 64$
\item $x = 16$
\item Números: 16, 18, 20 e 22
\end{enumerate}

\subsubsection*{Resposta}
$\boxed{16,\ 18,\ 20\ \text{e}\ 22}$

\section{Problemas Algébricos}

\subsection{Conceito Teórico}
Problemas algébricos envolvem traduzir situações do cotidiano para a linguagem matemática usando equações. A estratégia geral é:
\begin{itemize}
\item Definir a incógnita
\item Estabelecer a relação entre as quantidades
\item Montar a equação
\item Resolver
\item Verificar se a solução faz sentido no contexto
\end{itemize}

\subsection{Questão 6}
\subsubsection*{Enunciado}
Ana tem o triplo da idade de Beatriz. Daqui a 12 anos, Ana terá o dobro da idade de Beatriz. Qual a idade atual de Beatriz?

\subsubsection*{Resolução}
\begin{enumerate}
\item Hoje:
\begin{itemize}
\item Beatriz = $x$ anos
\item Ana = $3x$ anos
\end{itemize}
\item Daqui a 12 anos:
\begin{itemize}
\item Beatriz = $x + 12$
\item Ana = $3x + 12$
\end{itemize}
\item Equação: $3x + 12 = 2(x + 12)$
\item Resolvendo:
\begin{align*}
3x + 12 &= 2x + 24 \\
3x - 2x &= 24 - 12 \\
x &= 12
\end{align*}
\end{enumerate}

\subsubsection*{Resposta}
$\boxed{12\ \text{anos}}$

\section{Regra de Três}

\subsection{Conceito Teórico}
A regra de três resolve problemas de proporcionalidade:
\begin{itemize}
\item \textbf{Direta}: Quando as grandezas aumentam ou diminuem juntas
\item \textbf{Inversa}: Quando uma grandeza aumenta enquanto a outra diminui
\end{itemize}

\subsection{Questão 7}
\subsubsection*{Enunciado}
Uma torneira enche um tanque em 6 horas. Se fossem utilizadas 4 torneiras iguais, quanto tempo levaria para encher o mesmo tanque?

\subsubsection*{Resolução}
\begin{enumerate}
\item Relação inversamente proporcional
\item $1 \times 6 = 4 \times x$
\item $6 = 4x$
\item $x = \frac{6}{4} = 1,5$ horas (1 hora e 30 minutos)
\end{enumerate}

\subsubsection*{Resposta}
$\boxed{1\ \text{hora e}\ 30\ \text{minutos}}$

\subsection{Questão 8}
\subsubsection*{Enunciado}
Em uma fábrica, 8 máquinas produzem 1.200 peças em 5 dias. Quantas peças seriam produzidas por 12 máquinas em 7 dias?

\subsubsection*{Resolução}
\begin{enumerate}
\item Relação direta com máquinas e dias
\item $\frac{8}{12} \times \frac{5}{7} = \frac{1200}{y}$
\item Simplificando: $\frac{40}{84} = \frac{1200}{y}$
\item $40y = 100800$
\item $y = 2520$ peças
\end{enumerate}

\subsubsection*{Resposta}
$\boxed{2.520\ \text{peças}}$

\section{Proporcionalidade}

\subsection{Conceito Teórico}
Grandezas são diretamente proporcionais quando a razão entre elas é constante. Em mapas e escalas:
\[ \text{Escala} = \frac{\text{Medida no desenho}}{\text{Medida real}} \]

\subsection{Questão 9}
\subsubsection*{Enunciado}
Um mapa foi desenhado na escala 1:25.000. Se no mapa a distância entre duas cidades é de 8 cm, qual a distância real em quilômetros?

\subsubsection*{Resolução}
\begin{enumerate}
\item Escala 1:25.000 significa que 1 cm no mapa = 25.000 cm reais
\item $8 \times 25.000 = 200.000$ cm
\item Convertendo para km: $\frac{200.000}{100.000} = 2$ km
\end{enumerate}

\subsubsection*{Resposta}
$\boxed{2\ \text{km}}$

\section{Geometria Espacial}

\subsection{Conceito Teórico}
O volume de um paralelepípedo retângulo é calculado por:
\[ V = \text{comprimento} \times \text{largura} \times \text{altura} \]
No caso do cubo temos: 
\[V = a \cdot a \cdot a   \]
\[V = a^{3} \]

\begin{itemize}
    \item a é a aresta do cubo.
    \item  Lembre-se que 1 m³ $ = 1.000$  litros.
\end{itemize}


\subsection{Questão 10}
\subsubsection*{Enunciado}
Uma piscina tem formato de paralelepípedo com dimensões 8m × 4m × 2m. Qual o volume de água necessário para enchê-la completamente? (em litros)

\subsubsection*{Resolução}
\begin{enumerate}
\item Volume = comprimento × largura × profundidade
\item $V = 8 \times 4 \times 2 = 64$ m³
\item Conversão: 1 m³ = 1.000 L
\item $64 \times 1.000 = 64.000$ L
\end{enumerate}

\subsubsection*{Resposta}
$\boxed{64.000\ \text{litros}}$

\section{Conclusão}
Esta apostila apresentou uma revisão completa de conceitos fundamentais de matemática através de problemas resolvidos detalhadamente. Para consolidar o aprendizado. \\

Um Boa Sorte para todos!!!

\begin{itemize}
\item Resolva novamente os problemas sem consultar as soluções
\item Crie variações dos problemas apresentados
\item Pratique com exercícios adicionais
\end{itemize}

\end{document}
