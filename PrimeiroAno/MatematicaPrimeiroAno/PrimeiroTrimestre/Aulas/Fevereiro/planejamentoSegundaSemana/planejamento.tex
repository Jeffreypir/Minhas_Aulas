
\documentclass[a4paper,12pt]{article}
\usepackage[brazil]{babel}
\usepackage[a4paper,top=1.5cm,bottom=2.5cm,left=2cm,right=2cm]{geometry}
\usepackage{amsmath}

\title{Planejamento Semanal - Grandezas e Sistema Internacional de Unidades}
\author{}
\date{}

\begin{document}

\maketitle

\section*{Objetivos Gerais}
\begin{itemize}
    \item Compreender o conceito de grandezas e conversão de unidades.
    \item Identificar e utilizar corretamente as unidades de medida.
    \item Aplicar o Sistema Internacional de Unidades (SI) na resolução de problemas.
    \item Desenvolver habilidades para conversão entre diferentes sistemas de medida.
\end{itemize}

\section*{Planejamento das Aulas}

\subsection*{Aula 1 - Introdução às Grandezas e Conversão de Unidades}
\textbf{Conteúdo:}
\begin{itemize}
    \item Definição de grandezas físicas: São quantidades que podem ser medidas, como comprimento, massa, tempo e temperatura.
    \item Grandezas escalares e vetoriais: As grandezas escalares possuem apenas magnitude (exemplo: massa, temperatura). As grandezas vetoriais possuem magnitude e direção (exemplo: velocidade, força).
    \item Conversão de unidades dentro do mesmo sistema de medidas: Uso de fatores de conversão.
\end{itemize}
\textbf{Exemplo de Aplicação:} Conversão de 1,5 km para metros. Como 1 km equivale a 1000 metros, temos $1,5 \times 1000 = 1500$ m.

\textbf{Metodologia:} Aula expositiva com exemplos práticos do cotidiano.

\textbf{Atividade:} Exercícios de conversão de medidas simples.

\subsection*{Aula 2 - Unidades de Medida de uma Grandeza}
\textbf{Conteúdo:}
\begin{itemize}
    \item Unidades fundamentais e derivadas: Unidades fundamentais são aquelas independentes, como metro (m) e segundo (s). Unidades derivadas são compostas por mais de uma unidade fundamental, como metro por segundo (m/s).
    \item Prefixos métricos e suas aplicações: Kilo (k), Mili (m), Micro ($\mu$), etc.
    \item Sistema métrico decimal e suas vantagens: Facilidade na conversão entre unidades.
\end{itemize}
\textbf{Exemplo de Aplicação:} Se uma caixa pesa 2 kg, quantos gramas ela tem? Sabemos que 1 kg = 1000 g, então $2
\times 1000 = 2000$ g. \\
\textbf{Metodologia:} Resolução de problemas em grupo.

\textbf{Atividade:} Exercícios envolvendo mudanças de unidades.

\subsection*{Aula 3 - Sistema Internacional de Unidades (SI)}
\textbf{Conteúdo:}
\begin{itemize}
    \item Definição do SI e sua importância: Sistema adotado mundialmente para padronização de medidas.
    \item Unidades básicas do SI: Metro (m), quilograma (kg), segundo (s), ampere (A), kelvin (K), mol (mol) e candela (cd).
    \item Aplicações do SI na ciência e tecnologia: Uso do SI na engenharia, física e química.
\end{itemize}
\textbf{Exemplo de Aplicação:} A velocidade de um carro é geralmente medida em km/h. Para converter 72 km/h para m/s, usamos a relação $1 km/h = \frac{5}{18} m/s$, então $72 \times \frac{5}{18} = 20$ m/s.

\textbf{Metodologia:} Discussão sobre padrões internacionais de medidas.

\textbf{Atividade:} Exercícios de conversão de unidades utilizando o SI.

\subsection*{Aula 4 - Aplicação e Exercícios de Revisão}
\textbf{Conteúdo:}
\begin{itemize}
    \item Revisão geral dos conteúdos estudados.
    \item Aplicação prática das conversões de unidades.
    \item Problemas contextualizados.
\end{itemize}
\textbf{Exemplo de Aplicação:} Um atleta correu 400 metros em 50 segundos. Qual foi sua velocidade média? $v = \frac{400}{50} = 8$ m/s.

\textbf{Metodologia:} Aula prática com exercícios de aplicação.

\textbf{Atividade:} Lista de exercícios com desafios matemáticos envolvendo grandezas.

\section*{Questões com Respostas}

\begin{enumerate}
    \item Converta 1500 mm para metros. \\
    \textbf{Resposta:} $1500\text{ mm} = 1,5\text{ m}$.
    
    \item Um automóvel percorre 90 km em 2 horas. Qual é a sua velocidade média em m/s? \\
    \textbf{Resposta:} $v = \frac{90\times1000}{2\times3600} = 12,5\text{ m/s}$.
    
    \item Qual é a unidade de medida da força no SI? \\
    \textbf{Resposta:} Newton (N), que equivale a $1\text{ kg} \cdot \text{m/s}^2$.
    
    \item Uma caixa tem volume de 2,5 L. Qual é o volume em metros cúbicos? \\
    \textbf{Resposta:} $2,5 L = 2,5 \times 10^{-3} m^3$.
\end{enumerate}

\end{document}

