\documentclass[12pt]{article}
\usepackage[utf8]{inputenc}
\usepackage[brazil]{babel}
\usepackage{geometry}
\geometry{a4paper, left=2cm, right=2cm, top=2cm, bottom=2cm}
\usepackage{amsmath} % Para fórmulas matemáticas
\usepackage{multicol} % Para dividir a página em colunas
\usepackage{enumitem} % Para listas personalizadas
\usepackage{geometry} % Para ajustar margens
\geometry{a4paper, left=2cm, right=2cm, top=2cm, bottom=2cm}
\usepackage{xcolor} % Para usar cores
\usepackage{titlesec} % Para personalizar títulos

% Definir a cor das seções como azul
\titleformat{\section}
  {\normalfont\Large\bfseries\color{blue}} % Formato do título
  {\thesection} % Número da seção
  {1em} % Espaço entre número e título
  {} % Código antes do título

\renewcommand{\thesubsection}{\textcolor{red}{\arabic{section}.\arabic{subsection}}}
\titleformat{\subsection}{\color{red}\normalfont\bfseries}{\thesubsection}{1em}{}

\title{\textcolor{blue}{Unidades de Medida, Sistema Internacional (SI) e Conversão de Unidades}}
    \author{Professor: Jefferson }
\date{}

% Remove a numeração de todas as páginas
\pagestyle{empty}


\begin{document}

\maketitle
\thispagestyle{empty}

\begin{center}
\large{Nome: \underline{\hspace{8cm}} \quad Série-Turma: \underline{\hspace{3cm}}}
\end{center}


\begin{multicols}{2}
\section*{Introdução às Unidades de Medida}

\subsection*{O que são Unidades de Medida?}
Unidades de medida são padrões utilizados para quantificar grandezas físicas, como comprimento, massa, tempo, temperatura, entre outras. Elas permitem que façamos medições precisas e consistentes.

\subsection*{Grandezas Fundamentais e Derivadas}
\begin{itemize}
    \item \textbf{Grandezas Fundamentais}: São as grandezas básicas, como comprimento (metro), massa (quilograma) e tempo (segundo).
    \item \textbf{Grandezas Derivadas}: São combinações das grandezas fundamentais, como área (metro quadrado) e velocidade (metro por segundo).
\end{itemize}

\subsection*{Exemplo}
\begin{itemize}
    \item Comprimento: 5 metros (m).
    \item Massa: 2 quilogramas (kg).
    \item Tempo: 10 segundos (s).
\end{itemize}

\subsection*{Atividade}

 1. Liste três grandezas físicas e suas respectivas unidades de medida.  
\vspace{1cm}

2. Explique a diferença entre grandezas fundamentais e derivadas.  
\vspace{1cm}

\section*{Sistema Internacional de Unidades (SI)}

\subsection*{O que é o SI?}
O Sistema Internacional de Unidades (SI) é o padrão global de unidades de medida, adotado pela maioria dos países. Ele define as unidades básicas e derivadas para todas as grandezas físicas.

\subsection*{Unidades Básicas do SI}
\begin{itemize}
    \item Comprimento: Metro (m).
    \item Massa: Quilograma (kg).
    \item Tempo: Segundo (s).
    \item Corrente elétrica: Ampere (A).
    \item Temperatura: Kelvin (K).
    \item Quantidade de matéria: Mol (mol).
    \item Intensidade luminosa: Candela (cd).
\end{itemize}

\subsection*{Exemplo}
\begin{itemize}
    \item 1 km = 1000 m.
    \item 1 kg = 1000 g.
    \item 1 h = 3600 s.
\end{itemize}

\subsection*{Atividade}

1. Converta 2,5 km para metros.  
\vspace{1cm} \newline 
2. Quantos segundos existem em 1,5 horas?  
\vspace{1cm}\newline
3. Explique por que o SI é importante para a ciência e a tecnologia.  
\vspace{1cm}


\section*{Conversão de Unidades}

\subsection*{Por que Converter Unidades?}
A conversão de unidades é necessária para padronizar medições e facilitar a comunicação entre diferentes sistemas de medida.

\subsection*{Fatores de Conversão}
\begin{itemize}
    \item 1 km = 1000 m.
    \item 1 m = 100 cm.
    \item 1 kg = 1000 g.
    \item 1 h = 60 min = 3600 s.
\end{itemize}

\subsection*{Exemplo}
\begin{itemize}
    \item Converter 3 km para metros: \( 3 \, \text{km} \times 1000 = 3000 \, \text{m} \).
    \item Converter 500 g para quilogramas: \( 500 \, \text{g} \div 1000 = 0,5 \, \text{kg} \).
\end{itemize}

\subsection*{Atividade}

1. Converta 4500 g para quilogramas.  
\vspace{1cm}\newline
2. Converta 2,5 horas para segundos.  
\vspace{1cm}\newline
3. Explique como você faria para converter 1,2 m para centímetros.  
\vspace{1cm}


\section*{Aplicações Práticas do SI e Conversões}

\subsection*{Aplicações do SI no Cotidiano}
O SI é utilizado em diversas áreas, como engenharia, medicina, comércio e ciência. Por exemplo:
\begin{itemize}
    \item Medição de distâncias em mapas (km).
    \item Dosagem de medicamentos (mg ou mL).
    \item Controle de tempo em competições esportivas (s).
\end{itemize}

\subsection*{Exemplo}
\begin{itemize}
    \item Um carro percorre 120 km em 2 horas. Qual é a sua velocidade média em m/s?
    \begin{align*}
        \text{Velocidade} &= \frac{\text{Distância}}{\text{Tempo}} \\
        &= \frac{120 \, \text{km}}{2 \, \text{h}} = 60 \, \text{km/h} \\
        &= 60 \times \frac{1000 \, \text{m}}{3600 \, \text{s}} \approx 16,67 \, \text{m/s}.
    \end{align*}
\end{itemize}

\subsection*{Atividade}
1. Um avião voa a 900 km/h. Converta essa velocidade para m/s.  
\vspace{1cm}\newline
2. Uma caixa contém 2,5 kg de arroz. Quantos gramas de arroz há na caixa?  
\vspace{1cm}\newline
3. Pesquise uma aplicação do SI em sua comunidade e descreva como ele é utilizado.  
\vspace{1cm}

\end{multicols}

\end{document}
