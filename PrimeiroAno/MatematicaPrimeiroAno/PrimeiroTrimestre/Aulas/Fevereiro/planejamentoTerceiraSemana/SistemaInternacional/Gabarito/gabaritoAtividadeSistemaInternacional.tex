\documentclass[12pt]{article}
\usepackage[utf8]{inputenc}
\usepackage[brazil]{babel}
\usepackage{geometry}
\geometry{a4paper, left=2cm, right=2cm, top=2cm, bottom=2cm}
\usepackage{amsmath} % Para fórmulas matemáticas
\usepackage{multicol} % Para dividir a página em colunas
\usepackage{enumitem} % Para listas personalizadas
\usepackage{xcolor} % Para usar cores
\usepackage{titlesec} % Para personalizar títulos

% Definir a cor das seções como azul
\titleformat{\section}
  {\normalfont\Large\bfseries\color{blue}} % Formato do título
  {\thesection} % Número da seção
  {1em} % Espaço entre número e título
  {} % Código antes do título

\title{\textcolor{blue}{Atividade: Unidades de Medida, Sistema Internacional (SI) e Conversão de Unidades}}
\author{Professor: Jefferson}
\date{}

% Remove a numeração de todas as páginas
\pagestyle{empty}

\begin{document}

\maketitle
\thispagestyle{empty}

\begin{center}
\large{Nome: \underline{\hspace{8cm}} \quad Série-Turma: \underline{\hspace{3cm}}}
\end{center}

\vspace{1cm}

\begin{multicols}{2}

\section*{Questões}

1. (ENEM) Um medicamento deve ser administrado na proporção de 1,5 mg por quilograma de massa corporal. Se uma pessoa
    pesa 70 kg, qual é a quantidade de medicamento que ela deve tomar ?\newline
a) 105 mg  
b) 70 mg  
c) 1,5 mg  
d) 0,015 mg  
e) 150 mg  
\vspace{0.5cm}

2. (SAEPE) Um avião voa a uma velocidade de 900 km/h. Qual é a sua velocidade em m/s ?\newline  
a) 250 m/s  
b) 300 m/s  
c) 150 m/s  
d) 100 m/s  
e) 200 m/s  
\vspace{0.5cm}

3. (ENEM) Uma piscina tem 12 metros de comprimento, 6 metros de largura e 1,5 metros de profundidade. Qual é o volume de água, em litros, necessário para encher completamente essa piscina ?\newline (Considere que 1 m³ = 1000 litros.)  
a) 108 litros  
b) 1080 litros  
c) 10.800 litros  
d) 108.000 litros  
e) 1.080.000 litros  
\vspace{0.5cm}

4. (SAEPE) Um carro percorre 180 km em 2 horas. Qual é a sua velocidade média em m/s ?\newline  
a) 20 m/s  
b) 25 m/s  
c) 30 m/s  
d) 35 m/s  
e) 40 m/s  
\vspace{0.5cm}

5. (ENEM) Uma caixa contém 2,5 kg de arroz. Quantos gramas de arroz há na caixa ?\newline  
a) 250 g  
b) 500 g  
c) 1000 g  
d) 1500 g  
e) 2500 g  
\vspace{0.5cm}

6. (SAEPE) Um recipiente tem capacidade para 5 litros de água. Quantos mililitros (mL) de água cabem nesse recipiente ?\newline  
a) 50 mL  
b) 500 mL  
c) 5000 mL  
d) 50.000 mL  
e) 500.000 mL  
\vspace{0.5cm}

7. (ENEM) Um estudante mediu o comprimento de uma mesa e encontrou 1,2 metros. Qual é o comprimento da mesa em centímetros ?\newline  
a) 12 cm  
b) 120 cm  
c) 1200 cm  
d) 0,12 cm  
e) 1,2 cm  
\vspace{0.5cm}

8. (SAEPE) Uma torneira despeja 2 litros de água por minuto. Quantos litros de água serão despejados em 1 hora ?\newline  
a) 60 litros  
b) 120 litros  
c) 240 litros  
d) 360 litros  
e) 480 litros  
\vspace{0.5cm}

9. (ENEM) Um terreno retangular tem 50 metros de comprimento e 30 metros de largura. Qual é a área do terreno em hectares ?\newline (Considere que 1 hectare = 10.000 m².)  
a) 0,15 hectares  
b) 1,5 hectares  
c) 15 hectares  
d) 150 hectares  
e) 1500 hectares  
\vspace{0.5cm}

10. (SAEPE) Um recipiente tem capacidade para 3,5 litros de água. Quantos mililitros (mL) de água cabem nesse recipiente ?\newline  
a) 35 mL  
b) 350 mL  
c) 3500 mL  
d) 35.000 mL  
e) 350.000 mL  
\vspace{0.5cm}

\end{multicols}

\newpage % Página para o gabarito

\section*{Gabarito}

\begin{enumerate}
    \item a) 105 mg  
    \item a) 250 m/s  
    \item d) 108.000 litros  
    \item b) 25 m/s  
    \item e) 2500 g  
    \item c) 5000 mL  
    \item b) 120 cm  
    \item b) 120 litros  
    \item a) 0,15 hectares  
    \item c) 3500 mL  
\end{enumerate}

\end{document}
