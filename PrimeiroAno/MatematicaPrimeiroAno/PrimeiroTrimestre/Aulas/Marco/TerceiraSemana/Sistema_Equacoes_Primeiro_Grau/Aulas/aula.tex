\documentclass[11pt]{article}
\usepackage[utf8]{inputenc}
\usepackage[T1]{fontenc}
\usepackage{newtxtext,newtxmath} % Fonte Times New Melhor renderiza
\usepackage{amsmath}
\usepackage{multicol}
\usepackage{geometry}
\usepackage{tikz}
\usetikzlibrary{arrows.meta}
\usepackage{enumitem} % Para listas personalizadas
\usepackage{xcolor} % Para usar cores
\usepackage{titlesec} % Para personalizar títulos

% Ajusta o espaçamento antes e depois das seções
\titleformat{\section}[block]{\normalfont\Large\bfseries}{\thesection}{1em}{}
\titlespacing*{\section}{8pt}{8pt}{8pt}

\titleformat{\subsection}[block]{\normalfont\large\bfseries}{\thesubsection}{1em}{}
\titlespacing*{\subsection}{6pt}{6pt}{6pt}

\titleformat{\subsubsection}[block]{\normalfont\normalsize\bfseries}{\thesubsubsection}{1em}{}
\titlespacing*{\subsubsection}{6pt}{6pt}{6pt}

\geometry{a4paper, left=1cm, right=1cm, top=0.5cm, bottom=1.2cm}

\setlength{\columnseprule}{0.4pt}  % Linha dividindo as colunas 
\setlength{\baselineskip}{1.0\baselineskip} % Espaçamento simples

% Definir a cor das seções como azul
\titleformat{\section}
  {\normalfont\Large\bfseries\color{blue}} % Formato do título
  {\thesection} % Número da seção
  {1em} % Espaço entre número e título
  {} % Código antes do título

\renewcommand{\thesubsection}{\textcolor{red}{\arabic{section}.\arabic{subsection}}}
\titleformat{\subsection}{\color{red}\normalfont\bfseries}{\thesubsection}{1em}{}
\title{\textcolor{blue}{Sistemas de Equações do 1º Grau - Entendendo e Aplicando}}
\author{Professor: Jefferson}
\date{}

\begin{document}

\maketitle
\vspace{-1cm}  % Ajuste o valor conforme necessário

\begin{center}
\large{Nome: \underline{\hspace{8cm}} \quad Série-Turma: \underline{\hspace{3cm}}}
\end{center}

\begin{multicols}{2}

\section*{O que é um Sistema de Equações do 1º Grau?}

Um sistema de equações do 1º grau é um conjunto de duas ou mais equações lineares com duas ou mais variáveis. O objetivo é encontrar valores para as variáveis que satisfaçam todas as equações simultaneamente.

A forma geral de um sistema de equações do 1º grau com duas variáveis é:

\[
\begin{cases}
a_1x + b_1y = c_1 \\
a_2x + b_2y = c_2
\end{cases}
\]


\textbf{Onde:}
    \begin{itemize}
    \item  {\color{blue} \(a_1\), \(b_1\), \(c_1\), \(a_2\), \(b_2\) e \(c_2\)} são coeficientes conhecidos.
    \item {\color{red} \(x\) e \(y\)} são as variáveis que queremos encontrar. 
\end{itemize}
 
\section*{Métodos de Resolução}

Existem vários métodos para resolver sistemas de equações do 1º grau. Vamos estudar dois deles: o método da substituição e o método da adição.

\subsection*{Método da Substituição}

Este método consiste em isolar uma das variáveis em uma das equações e substituir essa expressão na outra equação.

\textbf{Exemplo:}

Resolva o sistema:

\[
\begin{cases}
x + y = 5 \\
2x - y = 1
\end{cases}
\]

\textbf{Solução:}

1. Isolamos \(y\) na primeira equação:

\[
y = 5 - x
\]

2. Substituímos \(y\) na segunda equação:

\[
2x - (5 - x) = 1
\]

3. Resolvemos a equação resultante:

\[
    \begin{aligned}
        3x - 5 &= 1 \\
        3x &= 6 \\
        x &= 2
    \end{aligned}
\]

4. Substituímos \(x = 2\) na expressão de \(y\):

\[
y = 5 - 2 = 3
\]

Portanto, a solução do sistema é \(x = 2\) e \(y = 3\).

\subsection*{Método da Adição}

Este método consiste em somar ou subtrair as equações do sistema de modo a eliminar uma das variáveis.

\textbf{Exemplo:}

Resolva o sistema:

\[
\begin{cases}
3x + 2y = 8 \\
2x - 2y = 2
\end{cases}
\]

\textbf{Solução:}

1. Somar as equações para eliminar \(y\).

\[
   \begin{aligned}
   3x + 2y &= 8 \quad \text{(Equação 1)} \\
   2x - 2y &= 2 \quad \text{(Equação 2)} \\
   \hline
   5x + 0 &= 10 \\
   5x &= 10 \\
   x &= \frac{10}{5} \\
   x &= 2
   \end{aligned}
   \]

2. Substituímos \(x = 2\) na primeira equação para encontrar \(y\):

\[
    \begin{aligned}
         3(2) + 2y &= 8 \\
         6 + 2y    &= 8 \\
         2y + 6    &= 8 \\
          y + 3    &= 4 \\
          y       &= 4 -3 \\
          y       &= 1 
    \end{aligned}
\]

Portanto, a solução do sistema é \(x = 2\) e \(y = 1\).

\section*{Atividade de Fixação}

Agora que já vimos como resolver sistemas de equações do 1º grau, vamos praticar com algumas questões.

\subsection*{Questão 1. Resolva o sistema:}
\[
\begin{cases}
x + y = 7 \\
x - y = 1
\end{cases}
\]

\subsection*{Questão 2. Resolva o sistema:}
\[
\begin{cases}
2x + 3y = 12 \\
4x - y = 10
\end{cases}
\]

\subsection*{Questão 3. Resolva o sistema:}
\[
\begin{cases}
5x - 2y = 4 \\
3x + y = 9
\end{cases}
\]

\subsection*{Questão 4. Resolva o sistema:}
\[
\begin{cases}
\frac{x}{2} + y = 5 \\
x - \frac{y}{3} = 4
\end{cases}
\]

\section*{Atividade Contextualizada}

Resolva as questões abaixo montando e resolvendo sistemas de equações do 1º grau adequados a cada situação.

\subsection*{Questão 1. (Compras no Supermercado)}
João comprou 3 maçãs e 2 bananas por R\$ 5,00. Maria comprou 2 maçãs e 4 bananas por R\$ 6,00. Qual é o preço de uma maçã e de uma banana?

\subsection*{Questão 2. (Idades)}
A soma das idades de Pedro e Ana é 25 anos. A diferença entre as idades é 5 anos. Qual é a idade de cada um?

\subsection*{Questão 3. (Investimentos)}
Um investidor aplicou R\$ 10.000,00 em dois fundos de investimento. No primeiro fundo, ele ganhou 5\% ao ano, e no segundo, ganhou 8\% ao ano. No final de um ano, ele teve um lucro total de R\$ 700,00. Quanto ele investiu em cada fundo?

\subsection*{Questão 4. (Viagem)}
Dois carros partem de duas cidades distantes 300 km uma da outra. O primeiro carro viaja a 60 km/h, e o segundo a 80 km/h. Em quanto tempo eles se encontrarão?

\subsection*{Questão 5. (Produção)}
Uma fábrica produz dois tipos de produtos, A e B. Para produzir uma unidade de A, são necessários 2 kg de matéria-prima, e para produzir uma unidade de B, são necessários 3 kg de matéria-prima. Em um dia, a fábrica usou 120 kg de matéria-prima e produziu 50 unidades no total. Quantas unidades de cada produto foram produzidas?

\subsection*{Questão 6. (Cinema)}
Em um cinema, o ingresso para adultos custa R\$ 20,00 e para crianças custa R\$ 10,00. Em um dia, foram vendidos 100 ingressos, e a arrecadação total foi de R\$ 1.500,00. Quantos ingressos para adultos e para crianças foram vendidos?

\subsection*{Questão 7. (Geometria)}
Um retângulo tem perímetro de 40 cm. Sabendo que o comprimento é o dobro da largura, determine as dimensões do retângulo.

\subsection*{Questão 8. (Economia Doméstica)}
Uma família gasta R\$ 800,00 por mês com alimentação e transporte. Sabe-se que o gasto com transporte é R\$ 200,00 a mais que o gasto com alimentação. Quanto a família gasta com cada item?

\subsection*{Questão 9. (Esportes)}
Em uma partida de basquete, um jogador marcou 25 pontos entre cestas de 2 e 3 pontos. Se ele acertou 10 cestas no total, quantas foram de 2 pontos e quantas foram de 3 pontos?

\subsection*{Questão 10. (Viagem de Trem)}
Dois trens partem de cidades distantes 600 km uma da outra. O Trem A viaja a 80 km/h, e o Trem B a 70 km/h. Em quanto tempo após a partida eles se encontrarão?

\subsection*{Questão 11. (Compras de Livros)}
Joana comprou 2 livros e 3 cadernos por R\$ 50,00. Pedro comprou 4 livros e 1 caderno por R\$ 60,00. Qual é o preço de um livro e de um caderno?

\subsection*{Questão 12. (Idades de Irmãos)}
A soma das idades de dois irmãos é 30 anos. Sabendo que um irmão é 6 anos mais velho que o outro, qual é a idade de cada um?

\subsection*{Questão 13. (Distribuição de Lucros)}
Uma empresa dividiu um lucro de R\$ 10.000,00 entre dois funcionários. O primeiro recebeu R\$ 2.000,00 a mais que o segundo. Quanto cada funcionário recebeu?

\subsection*{Questão 14. (Viagem de Ônibus)}
Dois ônibus partem de cidades distantes 400 km uma da outra. O primeiro ônibus viaja a 70 km/h, e o segundo a 90 km/h. Em quanto tempo eles se encontrarão?

\subsection*{Questão 15. (Produção de Camisetas)}
Uma confecção produz camisetas de dois tamanhos: P e M. Para produzir uma camiseta P, são necessários 1,5 m de tecido, e para uma camiseta M, 2 m de tecido. Em um dia, foram usados 200 m de tecido para produzir 120 camisetas. Quantas camisetas de cada tamanho foram produzidas?

\subsection*{Questão 16. (Venda de Frutas)}
Um feirante vendeu 50 kg de maçãs e laranjas por R\$ 300,00. Se o preço do kg da maçã é R\$ 8,00 e o da laranja é R\$ 4,00, quantos kg de cada fruta ele vendeu?

\subsection*{Questão 17. (Geometria: Triângulo)}
Um triângulo tem perímetro de 30 cm. Sabendo que um lado é o dobro do outro e que o terceiro lado é 6 cm, determine as medidas dos lados.

\subsection*{Questão 18. (Economia Doméstica: Contas)}
Uma família gasta R\$ 1.200,00 por mês com aluguel e energia elétrica. Sabe-se que o gasto com aluguel é R\$ 400,00 a mais que o gasto com energia. Quanto a família gasta com cada item?

\subsection*{Questão 19. (Esportes: Futebol)}
Em um jogo de futebol, um time marcou 20 gols no campeonato. Sabendo que o número de vitórias é o dobro do número de empates e que cada vitória vale 3 pontos e cada empate vale 1 ponto, quantas vitórias e quantos empates o time teve?

\subsection*{Questão 20. (Viagem de Avião)}
Dois aviões partem de cidades distantes 1.200 km uma da outra. O primeiro avião viaja a 500 km/h, e o segundo a 700 km/h. Em quanto tempo após a partida eles se encontrarão?

\end{multicols}

\end{document}
