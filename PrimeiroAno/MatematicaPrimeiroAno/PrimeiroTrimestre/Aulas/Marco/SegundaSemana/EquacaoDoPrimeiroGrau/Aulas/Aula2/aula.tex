\usepackage{extsizes} % Permite 14pt, 17pt, 20pt
\documentclass[11pt]{extarticle} % Mais próximo do 15ptusepackage[utf8]{inputenc}
\usepackage[T1]{fontenc}
\usepackage{newtxtext,newtxmath}
\usepackage{amsmath}
\usepackage{multicol}
\usepackage{geometry}
\usepackage{tikz}
\usetikzlibrary{arrows.meta}
\usepackage{enumitem}
\usepackage{xcolor}
\usepackage{titlesec}

\geometry{a4paper, left=1cm, right=1cm, top=0.5cm, bottom=1.2cm}

\titleformat{\section}[block]{\normalfont\Large\bfseries}{\thesection}{1em}{}
\titlespacing*{\section}{8pt}{8pt}{8pt}

\titleformat{\subsection}[block]{\normalfont\large\bfseries}{\thesubsection}{1em}{}
\titlespacing*{\subsection}{6pt}{6pt}{6pt}

\titleformat{\subsubsection}[block]{\normalfont\normalsize\bfseries}{\thesubsubsection}{1em}{}
\titlespacing*{\subsubsection}{6pt}{6pt}{6pt}

\setlength{\columnseprule}{0.4pt}
\setlength{\baselineskip}{1.0\baselineskip}

\titleformat{\section}{\normalfont\Large\bfseries\color{blue}}{\thesection}{1em}{}
\titleformat{\subsection}{\color{red}\normalfont\bfseries}{\thesubsection}{1em}{}

\title{\textcolor{blue}{Notação Cientifica, Grandezas e Equações do 1º Grau e suas aplicações}}
\author{Professor: Jefferson}
\date{}

\begin{document}

\maketitle
\vspace{-1cm}

\begin{center}
\large{Nome: \underline{\hspace{8cm}} \quad Série-Turma: \underline{\hspace{3cm}}}
\end{center}

\begin{multicols}{2}

\section*{Exercícios de Equações do 1° Grau}

\subsection*{Questão 1}
Resolva a equação: \( 2x + 5 = 17 \)

\subsection*{Questão 2}
Determine o valor de \( x \): \( 3x - 7 = 14 \)

\subsection*{Questão 3}
Resolva: \( 4(x + 3) = 20 \)

\subsection*{Questão 4}
Encontre a solução: \( 5x - 2 = 3x + 10 \)

\subsection*{Questão 5}
Resolva: \( \frac{x}{3} + 4 = 7 \)

\subsection*{Questão 6}
Determine \( x \): \( 2,5x + 1,2 = 6,7 \)

\subsection*{Questão 7}
Resolva: \( 3(2x - 1) = 5(x + 2) \)

\subsection*{Questão 8}
Encontre o valor de \( x \): \( \frac{2x + 1}{5} = 3 \)

\subsection*{Questão 9}
Resolva: \( 0,4x - 1,6 = 0,8 \)

\subsection*{Questão 10}
Determine a solução: \( \frac{x - 3}{2} = \frac{x + 1}{4} \)

\subsection*{Questão 11}
Resolva: \( 6 - 2(x + 3) = 4x \)

\subsection*{Questão 12}
Encontre \( x \): \( \frac{3}{x} = \frac{9}{12} \) (para \( x \neq 0 \))

\subsection*{Questão 13}
Resolva: \( 2x + \frac{x}{2} = 10 \)

\subsection*{Questão 14}
Determine o valor de \( x \): \( 5 - 3(x - 1) = 2 - x \)

\subsection*{Questão 15}
Resolva: \( \frac{2x - 3}{5} - \frac{x + 1}{2} = 0 \)

\columnbreak

\section*{Atividade - Aplicações}

\subsection*{Questão 1 (Grandezas Proporcionais)}
Uma torneira enche um reservatório em 45 minutos. Quantos reservatórios idênticos podem ser enchidos em 6 horas?

\subsection*{Questão 2 (Perímetro)}
Um terreno retangular tem perímetro de 180 m. Se a razão entre comprimento e largura é 5:4, determine suas dimensões.

\subsection*{Questão 3 (Área)}
A área de um triângulo é 48 cm². Se a base é 4 cm maior que a altura, encontre ambas as medidas.

\subsection*{Questão 4 (Conversão de Unidades)}
Converta 2,5 km² para m² e expresse o resultado em notação científica.

\subsection*{Questão 5 (Proporção)}
Para construir 300 m de estrada em 15 dias, são necessários 8 operários. Quantos operários são necessários para construir 500 m em 20 dias?

\subsection*{Questão 6 (Notação Científica)}
A distância Terra-Sol é aproximadamente 149,6 milhões de km. Expresse esta distância em metros usando notação científica.

\subsection*{Questão 7 (Velocidade Média)}
Um carro percorre 360 km em 4 horas e 30 minutos. Qual foi sua velocidade média em m/s?

\subsection*{Questão 8 (Área e Conversão)}
Um campo de futebol tem 110 m × 75 m. Expresse sua área em hectares (1 ha = 10.000 m²).

\subsection*{Questão 9 (Escalas)}
Em um mapa na escala 1:25.000, duas cidades distam 8 cm. Qual é a distância real em km?

\subsection*{Questão 10 (Volume)}
Uma caixa d'água cúbica tem volume de 64.000 litros. Qual é a medida de sua aresta em metros?

\subsection*{Questão 11 (Densidade)}
Um objeto tem massa de 450 g e volume de 180 cm³. Qual é sua densidade em kg/m³?

\subsection*{Questão 12 (Tempo)}
Quantos segundos existem em 2 dias, 5 horas e 30 minutos? Expresse o resultado em notação científica.

\subsection*{Questão 13 (Capacidade)}
Um reservatório cilíndrico tem diâmetro de 4 m e altura de 3 m. Quantos litros de água ele comporta?

\subsection*{Questão 14 (Massa)}
A massa da Terra é aproximadamente 5,97 × 10²⁴ kg. Expresse esta massa em toneladas (1 t = 10³ kg).

\subsection*{Questão 15 (Vazão)}
Uma torneira despeja 12 litros por minuto. Quanto tempo levará para encher um reservatório de 4,5 m³?

\section*{Desafios}

\subsection*{Desafio 1 (Problema Integrado)}
Um agricultor tem um terreno retangular de 120 m × 80 m. Ele deseja:
\begin{itemize}
\item Dividir o terreno em 4 partes iguais
\item Cercar cada parte com 3 fios de arame
\item Sabendo que cada rolo de arame tem 50 m, quantos rolos serão necessários?
\end{itemize}

\subsection*{Desafio 2 (Conversão Complexa)}
A velocidade da luz é 3 × 10⁸ m/s. Expresse esta velocidade em:
\begin{itemize}
\item km/h
\item milhas por hora (1 milha ≈ 1,609 km)
\end{itemize}

\subsection*{Desafio 3 (Proporção Composta)}
Na construção de um prédio, 15 operários trabalhando 8 horas por dia levam 60 dias. Quantos operários seriam necessários para terminar em 40 dias trabalhando 10 horas diárias?

\end{multicols}

\begin{center}
\textbf{Dicas e Fórmulas Úteis:} 
\begin{itemize}
\item Conversão de unidades: 1 km = 1000 m; 1 m³ = 1000 L
\item Notação científica: $a \times 10^n$ onde $1 \leq a < 10$
\item Velocidade média: $v = \frac{\Delta s}{\Delta t}$
\item Densidade: $d = \frac{m}{V}$
\item Para problemas de proporção: mantenha as relações diretas/inversas
\end{itemize}
\end{center}

\end{document}
