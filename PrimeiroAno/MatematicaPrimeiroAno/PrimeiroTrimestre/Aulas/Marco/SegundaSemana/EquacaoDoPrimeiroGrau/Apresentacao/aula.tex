\documentclass[12pt]{beamer}
\usepackage[utf8]{inputenc}
\usepackage[brazil]{babel}
\usepackage{amsmath} % Para fórmulas matemáticas
\usepackage{xcolor} % Para usar cores

\usetheme{Berkeley}

% Definir cores para títulos e subtítulos
\setbeamercolor{frametitle}{fg=white}
\setbeamercolor{framesubtitle}{fg=green}
% Adicionar nota de rodapé em todos os slides
\setbeamertemplate{footline}{
    \vspace{0.1cm} % Espaço acima da nota
    \footnotesize % Tamanho do texto
%%%%%%%%%    \hspace{2cm} Mestre em Modelagem Matemática e Computacional - UFPB
}

\title{\textcolor{white}{Equações do 1º Grau - Entendendo e Aplicando}}
\author{Professor: Jefferson}
\date{}

\begin{document}

\frame{\titlepage}

% Slide do sumário
\begin{frame}
    \frametitle{Sumário}
    \tableofcontents
\end{frame}

\begin{frame}
    \section{Introdução às Equações do 1º Grau}
    \frametitle{Introdução às Equações do 1º Grau}
    \framesubtitle{O que é uma Equação do 1º Grau?}
    \begin{itemize}
        \item Uma equação do 1º grau é uma expressão matemática que envolve uma variável \( x \) elevada ao expoente 1.
        \item Forma geral: $ \color{red} ax + b = 0 $, onde \( \textcolor{red}{a} \) e \( \textcolor{red}{b} \) são coeficientes conhecidos.
        \item Objetivo: Encontrar o valor de \( \textcolor{red}{x} \) que satisfaz a equação.
    \end{itemize}
\end{frame}

\begin{frame}
    \section{Resolução de Equações do 1º Grau}
    \frametitle{Resolução de Equações do 1º Grau}
    \framesubtitle{Equação Simples}
    \textcolor{green}{Passos Básicos}
    \begin{enumerate}
        \item Isolar o termo com \( x \).
        \item Realizar operações inversas para resolver a equação.
    \end{enumerate}

    \vspace{1.5cm}
    \textcolor{green}{Exemplo:}
    Resolva a equação:
    \[
    3x + 6 = 0
    \]
   \end{frame}

  \begin{frame} 
    \frametitle{Resolução:}
      Resolva a equação:
    \[
    3x + 6 = 0
    \]
      \vspace{5cm}

  \end{frame}

  \begin{frame}
      \frametitle{Exemplo 2: Equação com Frações}
      Resolva a equação:
      \[
          \frac{x}{4} - 3 = 0
      \]

      \vspace{5cm}
  \end{frame}

\begin{frame}
    \section{Tipos de Equações do 1º Grau}
    \frametitle{Tipos de Equações do 1º Grau}
    \framesubtitle{Equações com Parênteses}
    Resolva a equação:
    \[
    2(x - 3) = 8
    \]
    \vspace{5cm}

\end{frame}

\begin{frame}
    \frametitle{Equações com Coeficientes Negativos}
    Resolva a equação:
    \[
    -4x + 8 = -12
    \]
    \vspace{5cm}

\end{frame}

\begin{frame}
    \section{Atividades de Fixação}
    \frametitle{Atividades de Fixação}
    \framesubtitle{Questão 1}
    Resolva a equação:
    \[
    5x - 7 = 18
    \]

    \vspace{5cm}
\end{frame}

\begin{frame}
    \frametitle{Atividade de Fixação }
    \framesubtitle{Questão 2}
    Resolva a equação:
    \[
    \frac{x + 1}{2} = 3
    \]
    \vspace{5cm}
\end{frame}

\begin{frame}
    \frametitle{Atividade de Fixação }
    \framesubtitle{Questão 3}
    Resolva a equação:
    \[
    -2x + 4 = 10
    \]
    \vspace{5cm}
\end{frame}

\begin{frame}
    \frametitle{Atividade de Fixação }
    \framesubtitle{Questão 4}
    Resolva a equação:
    \[
    \frac{3x}{5} - 4 = 6
    \]
    \vspace{5cm}
\end{frame}

\begin{frame}
    \section{Resolução das Atividades}
    \frametitle{Resolução das Atividades}
    \framesubtitle{Questão 1}
    Resolva a equação:
    \[
    5x - 7 = 18
    \]
    \textbf{Solução:}
    \begin{align*}
        5x &= 25 \quad (\text{Somar 7 em ambos os lados}) \\
        x &= 5 \quad (\text{Dividir por 5})
    \end{align*}
    \textbf{Resposta:} \( x = 5 \).
\end{frame}

\begin{frame}
    \frametitle{Resolução das Atividades }
    \framesubtitle{Questão 2}
    Resolva a equação:
    \[
    \frac{x + 1}{2} = 3
    \]
    \textbf{Solução:}
    \begin{align*}
        x + 1 &= 6 \quad (\text{Multiplicar por 2}) \\
        x &= 5 \quad (\text{Subtrair 1})
    \end{align*}
    \textbf{Resposta:} \( x = 5 \).
\end{frame}

\begin{frame}
    \frametitle{Resolução das Atividades }
    \framesubtitle{Questão 3}
    Resolva a equação:
    \[
    -2x + 4 = 10
    \]
    \textbf{Solução:}
    \begin{align*}
        -2x &= 6 \quad (\text{Subtrair 4 de ambos os lados}) \\
        x &= -3 \quad (\text{Dividir por -2})
    \end{align*}
    \textbf{Resposta:} \( x = -3 \).
\end{frame}

\begin{frame}
    \frametitle{Resolução das Atividades }
    \framesubtitle{Questão 4}
    Resolva a equação:
    \[
    \frac{3x}{5} - 4 = 6
    \]
    \textbf{Solução:}
    \begin{align*}
        \frac{3x}{5} &= 10 \quad (\text{Somar 4 em ambos os lados}) \\
        3x &= 50 \quad (\text{Multiplicar por 5}) \\
        x &= \frac{50}{3} \quad (\text{Dividir por 3})
    \end{align*}
    \textbf{Resposta:} \( x = \frac{50}{3} \).
\end{frame}

\section{Atividades de Fixação: Contexto}

\begin{frame}
    \frametitle{Questão 1: Análise de Planos}
    \framesubtitle{Comparação de Custos}
    Uma empresa de transporte oferece dois planos para entregas rápidas:
    \begin{itemize}
        \item \textbf{Plano A:} Taxa fixa de R\$ 15,00 + R\$ 2,50 por quilômetro rodado.
        \item \textbf{Plano B:} Taxa fixa de R\$ 30,00 + R\$ 1,80 por quilômetro rodado.
    \end{itemize}
    \textbf{Pergunta:} Para qual distância (em km) os dois planos terão o mesmo custo?
    \textbf{Dica:} Iguale os custos totais: \(15 + 2.50x = 30 + 1.80x\).
\end{frame}

\begin{frame}
    \frametitle{Questão 2: Comparação de Descontos}
    \framesubtitle{Escolha da Melhor Opção}
    Uma camisa custa R\$ 120,00 com duas opções:
    \begin{itemize}
        \item \textbf{Opção 1:} 20\% de desconto no total.
        \item \textbf{Opção 2:} Desconto fixo de R\$ 30,00.
    \end{itemize}
    \textbf{Pergunta:} A partir de quantas camisas compradas a \textbf{Opção 1} se torna mais vantajosa que a \textbf{Opção 2}?
    \textbf{Dica:} Para \(n\) camisas: \(0.80 \times 120n = 120n - 30\).
\end{frame}

\begin{frame}
    \frametitle{Questão 3: Proporcionalidade Ambiental}
    \framesubtitle{Reciclagem de Papel}
    Um estudo mostra que, para cada 5 kg de papel reciclado, evita-se o corte de uma árvore. Se uma escola recolheu 120 kg de papel em um mês, quantas árvores foram preservadas?
    \textbf{Dica:} Relacione a quantidade de papel coletado com a proporção dada.
\end{frame}

\begin{frame}
    \frametitle{Questão 5: Geometria Aplicada}
    \framesubtitle{Perímetro de um Triângulo}
    Um terreno triangular tem sua base medindo o triplo do seu lado. Sabendo que o perímetro do terreno é 50 metros e os dois lados do triângulo são iguais, qual é a medida da base e dos lados desse triângulo?
    \textbf{Dica:} O perímetro é a soma dos lados.
\end{frame}

\begin{frame}
    \frametitle{Questão 6: Economia Doméstica}
    \framesubtitle{Redução do Consumo de Água}
    Uma família reduzirá seu consumo de água para atingir a meta de gastar no máximo R\$ 150,00 por mês. Atualmente, eles pagam R\$ 0,50 por m³ de água e consomem 400 m³ mensais. Quantos m³ precisam reduzir para atingir a meta?
    \textbf{Dica:} Equação: \(0.50(400 - x) = 150\).
\end{frame}

\begin{frame}
    \frametitle{Questão 7: Escolha de Pacotes}
    \framesubtitle{Comparação de Planos de Internet}
    \begin{itemize}
        \item \textbf{Básico:} 100 MB por R\$ 80/mês.
        \item \textbf{Premium:} 200 MB por R\$ 140/mês.
    \end{itemize}
    \textbf{Pergunta:} Quantos meses são necessários para que o Pacote Premium seja mais econômico por MB que o Básico?
    \textbf{Dica:} Calcule o custo por MB de cada pacote.
\end{frame}

\begin{frame}
    \frametitle{Questão 8: Movimento Uniforme}
    \framesubtitle{Encontro de Dois Trens}
    Dois trens partem de cidades distantes 600 km uma da outra. O Trem A viaja a 80 km/h, e o Trem B a 70 km/h. Em quanto tempo após a partida eles se encontrarão?
    \textbf{Dica:} A distância total é a soma das distâncias percorridas por cada trem: \(80t + 70t = 600\).
\end{frame}

\begin{frame}
    \frametitle{Questão 9: Sustentabilidade}
    \framesubtitle{Comparação de Lâmpadas}
    Uma lâmpada LED consome 10 W/h e dura 25.000 horas. Uma lâmpada incandescente consome 60 W/h e dura 1.000 horas. Considerando que 1 kWh custa R\$ 0,80, após quantas horas de uso o custo total (compra + energia) da LED se torna menor que o da incandescente, sabendo que a LED custa R\$ 40,00 e a incandescente R\$ 5,00?
    \textbf{Dica:} Modele o custo total de cada lâmpada em função do tempo \(t\).
\end{frame}

\begin{frame}
    \frametitle{Questão 10: Grandezas}
    \framesubtitle{Produção de Peças}
    Uma máquina produz 120 peças em 5 horas. Quantas peças ela produzirá em 8 horas, mantendo a mesma taxa de produção?
    \textbf{Dica:} A quantidade de peças produzidas é diretamente proporcional ao tempo de produção.
\end{frame}

\begin{frame}
    \frametitle{Questão 11: Grandezas}
    \framesubtitle{Tempo de Viagem}
    Um carro percorre uma distância fixa a uma velocidade média de 60 km/h em 4 horas. Quanto tempo levará para percorrer a mesma distância se a velocidade média aumentar para 80 km/h?
    \textbf{Dica:} O tempo de viagem é inversamente proporcional à velocidade.
\end{frame}

\begin{frame}
    \section{Conclusão}
    \frametitle{Conclusão}
    \begin{itemize}
        \item As equações do 1º grau são fundamentais para a resolução de problemas matemáticos.
        \item A prática constante é essencial para dominar as técnicas de resolução.
        \item Continue praticando com exercícios variados!
    \end{itemize}
\end{frame}

\begin{frame}
    \frametitle{Referências}
    \framesubtitle{Livros e materiais utilizados}
    \begin{itemize}
        \item IEZZI, Gelson et al. \textbf{Matemática: Ciência e Aplicações}. 9ª ed. São Paulo: Saraiva, 2017.
        \item IEZZI, Gelson et al. \textbf{Fundamentos de Matemática Elementar}. Volume 1. São Paulo: Atual Editora, 2013.
        \item DANTE, Luiz Roberto. \textbf{Matemática: Contexto e Aplicações}. Volume 1. São Paulo: Ática, 2018.
        \item PAIVA, Manoel. \textbf{Matemática}. Volume 1. São Paulo: Moderna, 2015.
        \item BONJORNO, José Roberto; BONJORNO, Regina Azenha. \textbf{Matemática: Fazendo a Diferença}. Volume 1. São Paulo: FTD, 2018.
    \end{itemize}
\end{frame}% Slide de encerramento
\begin{frame}
\begin{center}
    \textbf{\textcolor{blue}{\Large Obrigado pela atenção!}} \\[0.5cm]
\end{center}
\end{frame}

\end{document}
