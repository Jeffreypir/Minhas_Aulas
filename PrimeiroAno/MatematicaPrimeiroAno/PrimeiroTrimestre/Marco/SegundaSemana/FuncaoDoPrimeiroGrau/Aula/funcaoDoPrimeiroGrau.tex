\documentclass[11pt]{article}
\usepackage[utf8]{inputenc}
\usepackage{amsmath}
\usepackage{multicol}
\usepackage{geometry}
\usepackage{tikz}
\usetikzlibrary{arrows.meta}
\usepackage{enumitem} % Para listas personalizadas
\usepackage{xcolor} % Para usar cores
\usepackage{titlesec} % Para personalizar títulos

\geometry{a4paper, left=1.5cm, right=1.5cm, top=1.5cm, bottom=1.5cm}

% Definir a cor das seções como azul
\titleformat{\section}
  {\normalfont\Large\bfseries\color{blue}} % Formato do título
  {\thesection} % Número da seção
  {1em} % Espaço entre número e título
  {} % Código antes do título

\renewcommand{\thesubsection}{\textcolor{red}{\arabic{section}.\arabic{subsection}}}
\titleformat{\subsection}{\color{red}\normalfont\bfseries}{\thesubsection}{1em}{}
\title{\textcolor{blue}{Função do 1º Grau e Proporcionalidade}}
\author{Professor: Jefferson}
\date{}

\begin{document}

\maketitle

\begin{center}
\large{Nome: \underline{\hspace{8cm}} \quad Série-Turma: \underline{\hspace{3cm}}}
\end{center}

\setlength{\columnsep}{20pt} % Espaço entre as colunas
\setlength{\columnseprule}{0.4pt} % Espessura da linha de separaçã



\begin{multicols}{2}

\section*{1. O que é uma Função do 1º Grau?}

\subsection*{Definição}
Uma função do 1º grau é uma relação entre duas variáveis que pode ser expressa na forma:
\[
f(x) = ax + b
\]
onde:
\begin{itemize}
    \item \(a\) e \(b\) são constantes reais, com \(a \neq 0\).
    \item \(x\) é a variável independente.
    \item \(f(x)\) é a variável dependente.
\end{itemize}

\subsection*{Relação com Proporcionalidade}
A função do 1º grau está diretamente relacionada à proporcionalidade:
\begin{itemize}
    \item Se \(b = 0\), a função é \textbf{diretamente proporcional} (\(f(x) = ax\)).
    \item Se \(b \neq 0\), a função é \textbf{linear}, mas não diretamente proporcional.
\end{itemize}

\subsection*{Exemplo 1:}
\textbf{Problema:} Dada a função \(f(x) = 2x + 3\), calcule \(f(4)\).

\textbf{Resolução:}
\[
f(4) = 2 \times 4 + 3 = 8 + 3 = 11
\]

\subsection*{Exemplo 2:}
\textbf{Problema:} Dada a função \(f(x) = -x + 5\), calcule \(f(-2)\).

\textbf{Resolução:}
\[
f(-2) = -(-2) + 5 = 2 + 5 = 7
\]

\subsection*{Atividades}

1. Dada a função \(f(x) = 3x - 4\), calcule \(f(2)\) e \(f(-1)\). 

2. Dada a função \(f(x) = -2x + 6\), calcule \(f(3)\) e \(f(0)\).

3. Escreva a função do 1º grau que representa o custo de um serviço que cobra R\$ 30,00 de taxa fixa mais R\$ 15,00 por hora trabalhada. Calcule o custo para 4 horas.

4. Identifique se as funções abaixo são diretamente proporcionais:
   - \(f(x) = 5x\)
   - \(f(x) = 3x + 2\)
   - \(f(x) = -4x\)

\section*{2. Gráfico da Função do 1º Grau}

\subsection*{Construção do Gráfico}
O gráfico de uma função do 1º grau é uma reta. Para construir o gráfico, basta determinar dois pontos que satisfaçam a equação e traçar a reta que passa por eles.

\subsection*{Exemplo 1:}
\textbf{Problema:} Construa o gráfico da função \(f(x) = 2x + 1\).

\textbf{Resolução:}
1. Escolha dois valores para \(x\):
\[
x = 0 \Rightarrow f(0) = 2 \times 0 + 1 = 1
\]
\[
x = 1 \Rightarrow f(1) = 2 \times 1 + 1 = 3
\]
2. Os pontos são \((0, 1)\) e \((1, 3)\).
3. Trace a reta que passa por esses pontos.

\begin{center}
\begin{tikzpicture}[scale=0.8]
\draw[->] (-1, 0) -- (4, 0) node[right]{\(x\)};
\draw[->] (0, -1) -- (0, 4) node[above]{\(y\)};
\draw[domain=-0.5:3.5, smooth, variable=\x, blue] plot ({\x}, {2*\x + 1});
\filldraw (0, 1) circle (2pt) node[above left]{\((0, 1)\)};
\filldraw (1, 3) circle (2pt) node[above left]{\((1, 3)\)};
\end{tikzpicture}
\end{center}

\subsection*{Exemplo 2:}
\textbf{Problema:} Construa o gráfico da função \(f(x) = -x + 4\).

\textbf{Resolução:}
1. Escolha dois valores para \(x\):
\[
x = 0 \Rightarrow f(0) = -0 + 4 = 4
\]
\[
x = 2 \Rightarrow f(2) = -2 + 4 = 2
\]
2. Os pontos são \((0, 4)\) e \((2, 2)\).
3. Trace a reta que passa por esses pontos.

\begin{center}
\begin{tikzpicture}[scale=0.8]
\draw[->] (-1, 0) -- (4, 0) node[right]{\(x\)};
\draw[->] (0, -1) -- (0, 5) node[above]{\(y\)};
\draw[domain=-0.5:3.5, smooth, variable=\x, red] plot ({\x}, {-\x + 4});
\filldraw (0, 4) circle (2pt) node[above left]{\((0, 4)\)};
\filldraw (2, 2) circle (2pt) node[above left]{\((2, 2)\)};
\end{tikzpicture}
\end{center}

\subsection*{Atividades}
1. Construa o gráfico da função \(f(x) = 3x - 2\).

2. Construa o gráfico da função \(f(x) = -2x + 5\).

3. Dada a função \(f(x) = x + 1\), identifique dois pontos e trace o gráfico.

4. Explique por que o gráfico de \(f(x) = 4x\) passa pela origem \((0, 0)\).

\section*{3. Coeficiente Angular e Linear}

\subsection*{Coeficiente Angular (\(a\))}
O coeficiente angular (\(a\)) determina a inclinação da reta:
\begin{itemize}
    \item Se \(a > 0\), a reta é crescente.
    \item Se \(a < 0\), a reta é decrescente.
\end{itemize}

\subsection*{Coeficiente Linear (\(b\))}
O coeficiente linear (\(b\)) indica onde a reta intercepta o eixo \(y\).

\subsection*{Exemplo 1:}
\textbf{Problema:} Dada a função \(f(x) = 3x - 2\), identifique o coeficiente angular e linear.

\textbf{Resolução:}
\[
a = 3 \quad (\text{coeficiente angular})
\]
\[
b = -2 \quad (\text{coeficiente linear})
\]

\subsection*{Exemplo 2:}
\textbf{Problema:} Dada a função \(f(x) = -2x + 5\), identifique o coeficiente angular e linear.

\textbf{Resolução:}
\[
a = -2 \quad (\text{coeficiente angular})
\]
\[
b = 5 \quad (\text{coeficiente linear})
\]

\subsection*{Atividades}
1. Identifique o coeficiente angular e linear da função \(f(x) = 4x - 7\).

2. Dada a função \(f(x) = -3x + 6\), determine se a reta é crescente ou decrescente.

3. Escreva a função do 1º grau que tem coeficiente angular \(a = 5\) e coeficiente linear \(b = -3\).

4. Explique o significado do coeficiente linear no gráfico de uma função do 1º grau.

\section*{4. Aplicações da Função do 1º Grau e Proporcionalidade}

\subsection*{Grandezas Diretamente Proporcionais}
Duas grandezas são diretamente proporcionais quando uma aumenta na mesma proporção que a outra. Na função do 1º grau, isso ocorre quando \(b = 0\) (\(f(x) = ax\)).

\subsection*{Exemplo 1:}
\textbf{Problema:} Um táxi cobra R\$ 2,00 por quilômetro rodado. Escreva a função que representa o custo da corrida e calcule o custo para 10 km.

\textbf{Resolução:}
1. Função do custo:
\[
C(x) = 2x
\]
2. Custo para 10 km:
\[
C(10) = 2 \times 10 = 20 \text{ reais}
\]

\subsection*{Grandezas Inversamente Proporcionais}
Duas grandezas são inversamente proporcionais quando uma aumenta na proporção inversa da outra. Isso pode ser modelado por funções do tipo \(f(x) = \frac{k}{x}\), mas não é uma função do 1º grau.

\subsection*{Exemplo 2:}
\textbf{Problema:} Se 4 operários constroem um muro em 10 dias, quantos dias levarão 8 operários para construir o mesmo muro? (Considere que o trabalho é inversamente proporcional ao número de operários.)

\textbf{Resolução:}
1. Relação inversamente proporcional:
\[
4 \times 10 = 8 \times x
\]
2. Resolvendo para \(x\):
\[
x = \frac{4 \times 10}{8} = 5 \text{ dias}
\]

\subsection*{Atividades}
1. Um carro consome 10 litros de combustível para percorrer 100 km. Escreva a função que representa o consumo e calcule o consumo para 250 km.

2. Se 6 máquinas produzem 120 peças em 4 horas, quantas peças serão produzidas por 8 máquinas em 6 horas? (Considere que a produção é diretamente proporcional ao número de máquinas e ao tempo.)

3. Se 10 pedreiros constroem uma casa em 30 dias, quantos dias levarão 15 pedreiros para construir a mesma casa? (Considere que o trabalho é inversamente proporcional ao número de pedreiros.)

4. Uma torneira enche um tanque em 12 horas com uma vazão de 5 litros por minuto. Se a vazão for aumentada para 8 litros por minuto, em quanto tempo o tanque estará cheio? (Considere que o tempo é inversamente proporcional à vazão.)

\end{multicols}

\end{document}
