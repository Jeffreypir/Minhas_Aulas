\documentclass[11pt]{article}
\usepackage[utf8]{inputenc}
\usepackage[T1]{fontenc}
\usepackage{newtxtext,newtxmath} % Fonte Times New Melhor renderiza
\usepackage{amsmath}
\usepackage{multicol}
\usepackage{geometry}
\usepackage{tikz}
\usetikzlibrary{arrows.meta}
\usepackage{enumitem} % Para listas personalizadas
\usepackage{xcolor} % Para usar cores
\usepackage{titlesec} % Para personalizar títulos

% Ajusta o espaçamento antes e depois das seções
\titleformat{\section}[block]{\normalfont\Large\bfseries}{\thesection}{1em}{}
\titlespacing*{\section}{8pt}{8pt}{8pt}

\titleformat{\subsection}[block]{\normalfont\large\bfseries}{\thesubsection}{1em}{}
\titlespacing*{\subsection}{6pt}{6pt}{6pt}

\titleformat{\subsubsection}[block]{\normalfont\normalsize\bfseries}{\thesubsubsection}{1em}{}
\titlespacing*{\subsubsection}{6pt}{6pt}{6pt}


\geometry{a4paper, left=1cm, right=1cm, top=0.5cm, bottom=1.2cm}

\setlength{\columnseprule}{0.4pt}  % Linha dividindo as colunas 
\setlength{\baselineskip}{1.0\baselineskip} % Espaçamento simples

% Definir a cor das seções como azul
\titleformat{\section}
  {\normalfont\Large\bfseries\color{blue}} % Formato do título
  {\thesection} % Número da seção
  {1em} % Espaço entre número e título
  {} % Código antes do título

\renewcommand{\thesubsection}{\textcolor{red}{\arabic{section}.\arabic{subsection}}}
\titleformat{\subsection}{\color{red}\normalfont\bfseries}{\thesubsection}{1em}{}
\title{\textcolor{blue}{Equação do 1º Grau - Entendendo e Aplicando}}
\author{Professor: Jefferson}
\date{}


\begin{document}

\maketitle
\vspace{-1cm}  % Ajuste o valor conforme necessário


\begin{multicols}{2}

\subsection*{Questão 1. (Análise de Planos)}
Para encontrar a distância em que os dois planos têm o mesmo custo, igualamos as expressões dos custos totais:

\[
15 + 2.50x = 30 + 1.80x
\]

Resolvendo para \(x\):

\[
x = \frac{15}{0.70} \approx 21.43 \text{ km}
\]

\textbf{Resposta:} Os dois planos terão o mesmo custo após aproximadamente \boxed{21.43 \text{ km}}.



\subsection*{Questão 2. (Comparação de Descontos)}
Igualando os custos totais para \(n\) camisas:

\[
0.80 \times 120n = 120n - 30
\]

Resolvendo para \(n\):

\[
n = \frac{30}{24} = 1.25
\]

Como \(n\) deve ser inteiro, arredondamos para cima:

\[
n = 2
\]

\textbf{Resposta:} A \textbf{Opção 1} se torna mais vantajosa a partir de \boxed{2 \text{ camisas}}.



\subsection*{Questão 3. (Proporcionalidade Ambiental)}
A proporção é \(5 \text{ kg} = 1 \text{ árvore}\). Para \(120 \text{ kg}\):

\[
\text{Árvores preservadas} = \frac{120}{5} = 24 \text{ árvores}
\]

\textbf{Resposta:} Foram preservadas \boxed{24 \text{ árvores}}.



\subsection*{Questão 5. (Geometria Aplicada)}
Seja \(x\) a medida de um dos lados iguais. A base mede \(3x\). O perímetro é:

\[
x + x + 3x = 50
\]

Resolvendo para \(x\):

\[
x = 10 \text{ metros}
\]

A base mede:

\[
3x = 30 \text{ metros}
\]

\textbf{Resposta:} Os lados iguais medem \boxed{10 \text{ metros}} cada, e a base mede \boxed{30 \text{ metros}}.



\subsection*{Questão 6. (Economia Doméstica)}
A família deseja gastar no máximo R\$ 150,00. Atualmente, gastam:

\[
0.50 \times 400 = 200 \text{ reais}
\]

Para reduzir o gasto para R\$ 150,00:

\[
0.50(400 - x) = 150
\]

Resolvendo para \(x\):

\[
x = 100 \text{ m³}
\]

\textbf{Resposta:} Eles precisam reduzir o consumo em \boxed{100 \text{ m³}}.



\subsection*{Questão 7. (Escolha de Pacotes)}
Custo por MB:

- \textbf{Básico:} \(\frac{80}{100} = 0.80\) reais/MB
- \textbf{Premium:} \(\frac{140}{200} = 0.70\) reais/MB

\textbf{Resposta:} O \textbf{Pacote Premium} é mais econômico por MB desde o \boxed{\text{primeiro mês}}.



\subsection*{Questão 8. (Movimento Uniforme)}
A distância total é 600 km. A soma das distâncias percorridas é:

\[
80t + 70t = 600
\]

Resolvendo para \(t\):

\[
t = 4 \text{ horas}
\]

\textbf{Resposta:} Os trens se encontrarão após \boxed{4 \text{ horas}}.



\subsection*{Questão 9. (Sustentabilidade)}
Custo total em função do tempo \(t\):

- \textbf{LED:} \(40 + 0.010 \times 0.80 \times t\)
- \textbf{Incandescente:} \(5 + 0.060 \times 0.80 \times t\)

Igualando os custos:

\[
40 + 0.008t = 5 + 0.048t
\]

Resolvendo para \(t\):

\[
t = 875 \text{ horas}
\]

\textbf{Resposta:} O custo total da \textbf{LED} se torna menor que o da \textbf{incandescente} após \boxed{875 \text{ horas}}.



\subsection*{Questão 10. (Grandezas)}
Taxa de produção:

\[
\frac{120}{5} = 24 \text{ peças/hora}
\]

Em 8 horas:

\[
24 \times 8 = 192 \text{ peças}
\]

\textbf{Resposta:} A máquina produzirá \boxed{192 \text{ peças}} em 8 horas.



\subsection*{Questão 11. (Grandezas)}
Distância percorrida:

\[
60 \times 4 = 240 \text{ km}
\]

Tempo com velocidade de 80 km/h:

\[
\frac{240}{80} = 3 \text{ horas}
\]

\textbf{Resposta:} O carro levará \boxed{3 \text{ horas}} para percorrer a mesma distância.

\end{multicols}

\end{document}

