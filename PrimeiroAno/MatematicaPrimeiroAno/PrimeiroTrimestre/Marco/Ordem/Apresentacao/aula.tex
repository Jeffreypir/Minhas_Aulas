\documentclass[12pt]{beamer}
\usepackage[utf8]{inputenc}
\usepackage[brazil]{babel}
\usepackage{amsmath} % Para fórmulas matemáticas
\usepackage{xcolor} % Para usar cores

\usetheme{Berkeley}

% Definir cores para títulos e subtítulos
\setbeamercolor{frametitle}{fg=white}
\setbeamercolor{framesubtitle}{fg=green}

\title{\textcolor{white}{Ordem de Grandeza}}
\author{Professor: Jefferson}
\date{}

\begin{document}

\frame{\titlepage}

% Slide do sumário
\begin{frame}
    \frametitle{Sumário}
    \tableofcontents
\end{frame}

\begin{frame}
    \section{O que é Ordem de Grandeza?}
    \frametitle{O que é Ordem de Grandeza?}
    \framesubtitle{Definição}
    \begin{itemize}
        \item A ordem de grandeza de um número é a potência de dez mais próxima desse valor.
        \item Para facilitar, comumente colocamos o número em notação científica.
        \item Exemplo: O número 27.000 em notação científica é \(2,7 \times 10^4\), e sua ordem de grandeza é \(10^4\).
    \end{itemize}
\end{frame}

\begin{frame}
    \section{Tipos de Ordem de Grandeza}
    \frametitle{Tipos de Ordem de Grandeza}
    \framesubtitle{Média Aritmética vs. Média Geométrica}
    \begin{itemize}
        \item \textbf{Média Aritmética}:
            \begin{itemize}
                \item Média entre 1 e 10: \(\frac{1 + 10}{2} = 5,5\).
                \item Se \(m < 5,5\), a ordem de grandeza é \(10^n\).
                \item Se \(m \geq 5,5\), a ordem de grandeza é \(10^{n+1}\).
            \end{itemize}
        \item \textbf{Média Geométrica}:
            \begin{itemize}
                \item Média entre 1 e 10: \(\sqrt{1 \times 10} \approx 3,16\).
                \item Se \(m < 3,16\), a ordem de grandeza é \(10^n\).
                \item Se \(m \geq 3,16\), a ordem de grandeza é \(10^{n+1}\).
            \end{itemize}
    \end{itemize}
\end{frame}

\begin{frame}
    \section{Como descobrir a ordem de grandeza de um número?}
    \frametitle{Como descobrir a ordem de grandeza de um número?}

        Para se obter a ordem de grandeza de determinado valor, é necessário seguir as etapas abaixo:
        \begin{enumerate}
            \item Convertê-lo em notação científica.
            \item Analisar o valor que multiplica a potência:
            \begin{itemize}
                \item Pelo princípio da média aritmética, se o valor for menor que 5,5, a grandeza será \( 10^n \); se for igual ou maior, será \( 10^{n+1} \).
                \item Pelo princípio da média geométrica, se o valor for menor que 3,16, a grandeza será \( 10^n \); se for igual ou maior, será \( 10^{n+1} \).
            \end{itemize}
        \end{enumerate}
\end{frame}

\begin{frame}
    \frametitle{Exemplos de Critérios de Arredondamento}
    \framesubtitle{Aplicação dos Critérios}
    \begin{itemize}
        \item Número: 4,0 = \(4,0 \times 10^0\).
        \begin{itemize}
            \item Média Aritmética: \(4,0 < 5,5\) → Ordem de grandeza: \(10^0\).
            \item Média Geométrica: \(4,0 > 3,16\) → Ordem de grandeza: \(10^1\).
        \end{itemize}
        \item Número: 7,0 = \(7,0 \times 10^0\).
        \begin{itemize}
            \item Média Aritmética: \(7,0 > 5,5\) → Ordem de grandeza: \(10^1\).
            \item Média Geométrica: \(7,0 > 3,16\) → Ordem de grandeza: \(10^1\).
        \end{itemize}
    \end{itemize}
\end{frame}

\begin{frame}
    \frametitle{Qual Critério Utilizar?}
    \framesubtitle{Decisão entre Média Aritmética e Geométrica}
    \begin{itemize}
        \item Valores abaixo de 3,16: Ambos os critérios concordam (\(10^n\)).
        \item Valores acima de 5,5: Ambos os critérios concordam (\(10^{n+1}\)).
        \item Valores entre 3,16 e 5,5: Depende do critério escolhido.
        \item Em vestibulares, evita-se números nessa faixa para evitar confusão.
    \end{itemize}
\end{frame}

\begin{frame}
    \section{Exemplo 2: Ordem de Grandeza de um Disperso}
    \frametitle{Exemplo 2: Ordem de Grandeza de um Disperso}

    Um coloide é um tipo de mistura heterogênea composta por disperso (menor quantidade) e dispersante (meio onde o disperso é colocado). Os dispersos devem ter tamanhos entre \(0,000000001 \, \text{m}\) e \(0,000001 \, \text{m}\). 

    Se um determinado disperso sólido possui \(0,00000087 \, \text{m}\), qual a sua ordem de grandeza?

\end{frame}
\begin{frame}
    \section{Exemplo 2: Ordem de Grandeza de um Disperso}
    \frametitle{Resolução}
    O disperso sólido: \(0,00000087 \)
     \vspace{6cm} 
\end{frame}

\begin{frame}
    \section{Tabela de Prefixos e Ordens de Grandeza}
    \frametitle{Tabela de Prefixos e Ordens de Grandeza}
    \framesubtitle{Prefixos Comuns}
    \begin{table}[h]
    \centering
    \begin{tabular}{|c|c|c|}
    \hline
    Prefixo & Unidade & Ordem de Grandeza \\
    \hline
    Femto & f & \(10^{-15}\) \\
    Pico & p & \(10^{-12}\) \\
    Nano & n & \(10^{-9}\) \\
    Micro & \(\mu\) & \(10^{-6}\) \\
    Mili & m & \(10^{-3}\) \\
    Quilo & k & \(10^{3}\) \\
    Mega & M & \(10^{6}\) \\
    Giga & G & \(10^{9}\) \\
    Tera & T & \(10^{12}\) \\
    \hline
    \end{tabular}
    \caption{Prefixos para ordens de grandeza}
    \end{table}
\end{frame}

\begin{frame}
    \section{Escalas de Ordem de Grandeza}
    \frametitle{Escalas de Ordem de Grandeza}
    \framesubtitle{Comprimento}
    \begin{itemize}
        \item \textbf{Subatômico}: \(0 \leq x < 10^{-15}\) (quarks, elétrons).
        \item \textbf{Atômico para Celular}: \(10^{-15} < x < 10^{-6}\) (prótons, vírus).
        \item \textbf{Escala Humana}: \(10^{-6} < x < 10^{6}\) (cabelo humano, Monte Everest).
        \item \textbf{Astronômico}: \(10^{6} < x < \infty\) (Sol, galáxias).
    \end{itemize}
\end{frame}

% Seção 10: Atividade
\begin{frame}
\section{Atividade: Ordem de Grandeza}
\subsection{Exercícios}
Nos exercícios abaixo encontre a ordem de grandeza utilize o método aritmético e geométrico.
\begin{enumerate}
    \item \textbf{Problema 1:} Converta o número \(4500\) g para quilogramas e determine a ordem de grandeza do valor obtido.
    \item \textbf{Problema 2:} Converta o tempo de \(2,5\) horas para segundos e determine a ordem de grandeza do valor obtido.
    \item \textbf{Problema 3:} Explique como você faria para converter \(1,2\) m para centímetros e determine a ordem de grandeza do valor obtido.
\end{enumerate}
\end{frame}



\begin{frame}
    \section{Conclusão}
    \frametitle{Conclusão}
    \begin{itemize}
        \item A ordem de grandeza é uma ferramenta útil para estimar magnitudes de números muito grandes ou muito pequenos.
        \item A notação científica facilita o cálculo da ordem de grandeza.
        \item É importante entender os critérios de arredondamento para evitar erros.
    \end{itemize}
\end{frame}


\begin{frame}
    \section{Referências}
    \frametitle{Referências}
    \framesubtitle{Livros e Materiais Utilizados}
    \begin{itemize}
        \item TIPLER, Paul A.; MOSCA, Gene. \textbf{Física para Cientistas e Engenheiros}. 6ª ed. Rio de Janeiro: LTC, 2014.
        \item HALLIDAY, David; RESNICK, Robert; WALKER, Jearl. \textbf{Fundamentos de Física}. 10ª ed. Rio de Janeiro: LTC, 2016.
    \end{itemize}
\end{frame}

% Slide de encerramento (último slide)
\begin{frame}
\begin{center}
    \textbf{\textcolor{blue}{\Large Obrigado pela atenção!}} \\[0.5cm]
    \small{Professor: Jefferson} \\
\end{center}
\end{frame}

\end{document}
