\documentclass[12pt]{beamer}
\usepackage[utf8]{inputenc}
\usepackage[brazil]{babel}
\usepackage{amsmath} % Para fórmulas matemáticas
\usepackage{xcolor} % Para usar cores

\usetheme{Berkeley}

% Definir cores para títulos e subtítulos
\setbeamercolor{frametitle}{fg=white}
\setbeamercolor{framesubtitle}{fg=green}
% Adicionar nota de rodapé em todos os slides
\setbeamertemplate{footline}{
    \vspace{0.1cm} % Espaço acima da nota
    \footnotesize % Tamanho do texto
%%%%%%%%%    \hspace{2cm} Mestre em Modelagem Matemática e Computacional - UFPB
}

\title{\textcolor{white}{Equações do 1º Grau - Entendendo e Aplicando}}
\author{Professor: Jefferson}
\date{}

\begin{document}

\frame{\titlepage}

% Slide do sumário
\begin{frame}
    \frametitle{Sumário}
    \tableofcontents
\end{frame}



% Slide 1: Título
\begin{frame}
\titlepage
\end{frame}

% Slide 2: Definição
\section{O que é um Sistema de Equações?}
\begin{frame}{O que é um Sistema de Equações?}
Um sistema de equações do 1º grau é um conjunto de duas ou mais equações lineares. Exemplo:

\[
\begin{cases}
2x + y = 10 \\
x - y = 2
\end{cases}
\]

\textbf{Objetivo:} Encontrar valores para \(x\) e \(y\) que satisfaçam ambas as equações.
\end{frame}

% Slide 3: Método da Substituição
\section{Método da Substituição}
\begin{frame}{Método da Substituição}
\textbf{Exemplo:}
\[
\begin{cases}
x + y = 5 \\
2x - y = 1
\end{cases}
\]

\textbf{Passo a Passo:}

1. Isolar \(y\) na 1ª equação:
\[
y = 5 - x
\]

2. Substituir na 2ª equação:
\[
2x - (5 - x) = 1
\]

3. Resolver para \(x\):
\[
3x - 5 = 1 \Rightarrow x = 2
\]

4. Encontrar \(y\):
\[
y = 5 - 2 = 3
\]
\end{frame}

% Slide 4: Método da Adição
\section{Método da Adição}
\begin{frame}{Método da Adição}
\textbf{Exemplo:}
\[
\begin{cases}
3x + 2y = 8 \\
2x - 2y = 2
\end{cases}
\]

\textbf{Passo a Passo:}

1. Somar as equações:
\[
\begin{aligned}
3x + 2y &= 8 \\
2x - 2y &= 2 \\
\hline
5x &= 10 \Rightarrow x = 2
\end{aligned}
\]

2. Substituir \(x = 2\):
\[
3(2) + 2y = 8 \Rightarrow y = 1
\]
\end{frame}

% Slide 5: Atividades de Fixação
\section{Atividades de Fixação}
\begin{frame}{Atividades de Fixação}
Resolva os sistemas:

1. 
\[
\begin{cases}
x + y = 7 \\
x - y = 1
\end{cases}
\]

2. 
\[
\begin{cases}
2x + 3y = 12 \\
4x - y = 10
\end{cases}
\]

3. 
\[
\begin{cases}
5x - 2y = 4 \\
3x + y = 9
\end{cases}
\]
\end{frame}

% Slide 6: Questões Contextualizadas
\section{Questões Contextualizadas}

\begin{frame}
\frametitle{Questão 1. (Compras no Supermercado)}
João comprou 3 maçãs e 2 bananas por R\$ 5,00. Maria comprou 2 maçãs e 4 bananas por R\$ 6,00. Qual é o preço de uma maçã e de uma banana?
\end{frame}

\begin{frame}
\frametitle{Questão 2. (Idades)}
A soma das idades de Pedro e Ana é 25 anos. A diferença entre as idades é 5 anos. Qual é a idade de cada um?
\end{frame}

\begin{frame}
\frametitle{Questão 3. (Investimentos)}
Um investidor aplicou R\$ 10.000,00 em dois fundos de investimento. No primeiro fundo, ele ganhou 5\% ao ano, e no segundo, ganhou 8\% ao ano. No final de um ano, ele teve um lucro total de R\$ 700,00. Quanto ele investiu em cada fundo?
\end{frame}

\begin{frame}
\frametitle{Questão 4. (Viagem)}
Dois carros partem de duas cidades distantes 300 km uma da outra. O primeiro carro viaja a 60 km/h, e o segundo a 80 km/h. Em quanto tempo eles se encontrarão?
\end{frame}

\begin{frame}
\frametitle{Questão 5. (Produção)}
Uma fábrica produz dois tipos de produtos, A e B. Para produzir uma unidade de A, são necessários 2 kg de matéria-prima, e para produzir uma unidade de B, são necessários 3 kg de matéria-prima. Em um dia, a fábrica usou 120 kg de matéria-prima e produziu 50 unidades no total. Quantas unidades de cada produto foram produzidas?
\end{frame}

\begin{frame}
\frametitle{Questão 6. (Cinema)}
Em um cinema, o ingresso para adultos custa R\$ 20,00 e para crianças custa R\$ 10,00. Em um dia, foram vendidos 100 ingressos, e a arrecadação total foi de R\$ 1.500,00. Quantos ingressos para adultos e para crianças foram vendidos?
\end{frame}

\begin{frame}
\frametitle{Questão 7. (Geometria)}
Um retângulo tem perímetro de 40 cm. Sabendo que o comprimento é o dobro da largura, determine as dimensões do retângulo.
\end{frame}

\begin{frame}
\frametitle{Questão 8. (Economia Doméstica)}
Uma família gasta R\$ 800,00 por mês com alimentação e transporte. Sabe-se que o gasto com transporte é R\$ 200,00 a mais que o gasto com alimentação. Quanto a família gasta com cada item?
\end{frame}

\begin{frame}
\frametitle{Questão 9. (Esportes)}
Em uma partida de basquete, um jogador marcou 25 pontos entre cestas de 2 e 3 pontos. Se ele acertou 10 cestas no total, quantas foram de 2 pontos e quantas foram de 3 pontos?
\end{frame}

\begin{frame}
\frametitle{Questão 10. (Viagem de Trem)}
Dois trens partem de cidades distantes 600 km uma da outra. O Trem A viaja a 80 km/h, e o Trem B a 70 km/h. Em quanto tempo após a partida eles se encontrarão?
\end{frame}

\begin{frame}
\frametitle{Questão 11. (Compras de Livros)}
Joana comprou 2 livros e 3 cadernos por R\$ 50,00. Pedro comprou 4 livros e 1 caderno por R\$ 60,00. Qual é o preço de um livro e de um caderno?
\end{frame}

\begin{frame}
\frametitle{Questão 12. (Idades de Irmãos)}
A soma das idades de dois irmãos é 30 anos. Sabendo que um irmão é 6 anos mais velho que o outro, qual é a idade de cada um?
\end{frame}

\begin{frame}
\frametitle{Questão 13. (Distribuição de Lucros)}
Uma empresa dividiu um lucro de R\$ 10.000,00 entre dois funcionários. O primeiro recebeu R\$ 2.000,00 a mais que o segundo. Quanto cada funcionário recebeu?
\end{frame}

\begin{frame}
\frametitle{Questão 14. (Viagem de Ônibus)}
Dois ônibus partem de cidades distantes 400 km uma da outra. O primeiro ônibus viaja a 70 km/h, e o segundo a 90 km/h. Em quanto tempo eles se encontrarão?
\end{frame}

\begin{frame}
\frametitle{Questão 15. (Produção de Camisetas)}
Uma confecção produz camisetas de dois tamanhos: P e M. Para produzir uma camiseta P, são necessários 1,5 m de tecido, e para uma camiseta M, 2 m de tecido. Em um dia, foram usados 200 m de tecido para produzir 120 camisetas. Quantas camisetas de cada tamanho foram produzidas?
\end{frame}

\begin{frame}
\frametitle{Questão 16. (Venda de Frutas)}
Um feirante vendeu 50 kg de maçãs e laranjas por R\$ 300,00. Se o preço do kg da maçã é R\$ 8,00 e o da laranja é R\$ 4,00, quantos kg de cada fruta ele vendeu?
\end{frame}

\begin{frame}
\frametitle{Questão 17. (Geometria: Triângulo)}
Um triângulo tem perímetro de 30 cm. Sabendo que um lado é o dobro do outro e que o terceiro lado é 6 cm, determine as medidas dos lados.
\end{frame}

\begin{frame}
\frametitle{Questão 18. (Economia Doméstica: Contas)}
Uma família gasta R\$ 1.200,00 por mês com aluguel e energia elétrica. Sabe-se que o gasto com aluguel é R\$ 400,00 a mais que o gasto com energia. Quanto a família gasta com cada item?
\end{frame}

\begin{frame}
\frametitle{Questão 19. (Esportes: Futebol)}
Em um jogo de futebol, um time marcou 20 gols no campeonato. Sabendo que o número de vitórias é o dobro do número de empates e que cada vitória vale 3 pontos e cada empate vale 1 ponto, quantas vitórias e quantos empates o time teve?
\end{frame}

\begin{frame}
\frametitle{Questão 20. (Viagem de Avião)}
Dois aviões partem de cidades distantes 1.200 km uma da outra. O primeiro avião viaja a 500 km/h, e o segundo a 700 km/h. Em quanto tempo após a partida eles se encontrarão?
\end{frame}

% Slide 7: Conclusão
\begin{frame}{Conclusão}
\begin{itemize}
\item Sistemas de equações resolvem problemas práticos.
\item Métodos principais: substituição e adição.
\item Pratique com as atividades propostas!
\end{itemize}

\centering
\textbf{Obrigado pela atenção!}
\end{frame}

\end{document}
