\documentclass[11pt]{article}
\usepackage[utf8]{inputenc}
\usepackage[T1]{fontenc}
\usepackage{newtxtext,newtxmath}
\usepackage{amsmath}
\usepackage{multicol}
\usepackage{geometry}
\usepackage{enumitem}
\usepackage{xcolor}

\geometry{a4paper, left=2cm, right=2cm, top=2cm, bottom=2cm}
\setlength{\columnsep}{1.5cm}

\title{Respostas Explicadas - Revisão de Matemática}
\author{Professor: Jefferson}
\date{}

\begin{document}

\maketitle

\begin{multicols}{2}

\section*{Notação Científica e Razão}

\begin{enumerate}[leftmargin=*]

\item \textbf{População da China}: $\boxed{1,\!412 \times 10^9 \text{ habitantes}}$ \\
\textit{Explicação}: Para escrever 1,412 bilhão em notação científica:
\begin{itemize}
\item 1 bilhão = 1.000.000.000 = $10^9$
\item Movemos a vírgula para depois do primeiro dígito: 1,412
\item Contamos quantas casas a vírgula andou: 9 casas
\item Resultado: 1,412 × $10^9$
\end{itemize}

\item \textbf{Razão entre massas}: $\boxed{1,\!5 \times 10^9}$ \\
\textit{Explicação}: 
\begin{align*}
\frac{\text{Massa elefante}}{\text{Massa formiga}} &= \frac{6 \times 10^3 \text{ kg}}{4 \times 10^{-6} \text{ kg}} \\
&= \frac{6}{4} \times 10^{3-(-6)} \\
&= 1,\!5 \times 10^9
\end{align*}
Significa que o elefante é 1,5 bilhão de vezes mais pesado que a formiga.

\item \textbf{Distância Terra-Sol}: $\boxed{149.600.000 \text{ km}}$ \\
\textit{Explicação}: Basta multiplicar:
\begin{align*}
1,\!496 \times 10^8 &= 1,\!496 \times 100.000.000 \\
&= 149.600.000 \text{ km}
\end{align*}

\item \textbf{Diâmetro do vírus}: $\boxed{0,\!00000025 \text{ m}}$ \\
\textit{Explicação}: O expoente -7 indica que a vírgula deve mover 7 casas para a esquerda:
\begin{align*}
2,\!5 \times 10^{-7} &= 0,\!00000025 \text{ m} \\
&= 250 \text{ nanômetros}
\end{align*}

\end{enumerate}

\section*{Geometria}

\begin{enumerate}[leftmargin=*,resume]

\item \textbf{Largura do terreno}: $\boxed{7,\!5 \text{ m}}$ \\
\textit{Explicação}: Se comprimento = 3 × largura:
\begin{align*}
\text{Perímetro} &= 2(\text{L} + \text{C}) = 60 \\
2(x + 3x) &= 60 \\
8x &= 60 \\
x &= 7,\!5 \text{ m}
\end{align*}

\item \textbf{Comprimento da piscina}: $\boxed{12 \text{ m}}$ \\
\textit{Explicação}: Resolvendo a equação da área:
\begin{align*}
\text{Área} &= \text{L} \times \text{C} = 72 \\
x(x+6) &= 72 \\
x^2 + 6x - 72 &= 0 \\
x &= 6 \text{ m (largura)} \\
\text{Comprimento} &= 6 + 6 = 12 \text{ m}
\end{align*}

\item \textbf{Lado do triângulo}: $\boxed{12 \text{ cm}}$ \\
\textit{Explicação}: Triângulo equilátero tem 3 lados iguais:
\begin{align*}
\text{Perímetro} &= 3 \times \text{lado} = 36 \text{ cm} \\
\text{Lado} &= 36 \div 3 = 12 \text{ cm}
\end{align*}

\item \textbf{Raio do círculo}: $\boxed{5 \text{ m}}$ \\
\textit{Explicação}: Usando a fórmula da área:
\begin{align*}
A &= \pi r^2 \\
78,\!5 &= 3,\!14 \times r^2 \\
r^2 &= 78,\!5 \div 3,\!14 \\
r^2 &= 25 \\
r &= 5 \text{ m}
\end{align*}

\end{enumerate}

\section*{Álgebra}

\begin{enumerate}[leftmargin=*,resume]

\item \textbf{Preço do livro}: $\boxed{\text{R}\$\,70,\!00}$ \\
\textit{Explicação}: Cálculo do preço original:
\begin{align*}
\text{Total com desconto} &= 240 \text{ reais} \\
\text{Desconto total} &= 4 \times 10 = 40 \text{ reais} \\
\text{Total sem desconto} &= 240 + 40 = 280 \text{ reais} \\
\text{Preço por livro} &= 280 \div 4 = 70 \text{ reais}
\end{align*}

\item \textbf{Idade de Carla}: $\boxed{14 \text{ anos}}$ \\
\textit{Explicação}: Resolvendo a equação:
\begin{align*}
\text{Hoje:} & \quad \text{Carla} = x, \text{Ana} = 2x \\
\text{Em 6 anos:} & \quad (x+6) + (2x+6) = 54 \\
& \quad 3x + 12 = 54 \\
& \quad 3x = 42 \\
& \quad x = 14 \text{ anos}
\end{align*}

\end{enumerate}

\section*{Regra de Três}

\begin{enumerate}[leftmargin=*,resume]

\item \textbf{Folhas produzidas}: $\boxed{3.000 \text{ folhas}}$ \\
\textit{Explicação}: Cálculo da produtividade:
\begin{align*}
\text{Produtividade} &= \frac{600}{3 \times 2} = 100 \text{ folhas/impressora-hora} \\
\text{Total} &= 5 \times 6 \times 100 = 3.000 \text{ folhas}
\end{align*}

\item \textbf{Pedreiros necessários}: $\boxed{16}$ \\
\textit{Explicação}: Relação inversamente proporcional:
\begin{align*}
8 \text{ pedreiros} \times 12 \text{ dias} &= x \times 6 \text{ dias} \\
x &= \frac{8 \times 12}{6} = 16 \text{ pedreiros}
\end{align*}

\item \textbf{Custo do tecido}: $\boxed{\text{R}\$\,60,\!00}$ \\
\textit{Explicação}: Proporção direta:
\begin{align*}
2 \text{ m} &= 24 \text{ reais} \\
1 \text{ m} &= 12 \text{ reais} \\
5 \text{ m} &= 5 \times 12 = 60 \text{ reais}
\end{align*}

\item \textbf{Torneiras necessárias}: $\boxed{6}$ \\
\textit{Explicação}: Relação inversa:
\begin{align*}
4 \times 9 &= x \times 6 \\
x &= \frac{36}{6} = 6 \text{ torneiras}
\end{align*}

\end{enumerate}

\section*{Problemas Diversos}

\begin{enumerate}[leftmargin=*,resume]

\item \textbf{Volume do cubo}: $\boxed{64 \text{ cm}^3}$ \\
\textit{Explicação}: Fórmula do volume:
\begin{align*}
V &= \text{aresta}^3 \\
&= 4^3 = 64 \text{ cm}^3
\end{align*}

\item \textbf{Volume do cilindro}: $\boxed{226,\!08 \text{ cm}^3}$ \\
\textit{Explicação}: Cálculo do volume:
\begin{align*}
V &= \pi r^2 h \\
&= 3,\!14 \times 3^2 \times 8 \\
&= 226,\!08 \text{ cm}^3
\end{align*}

\item \textbf{Conversão para litros}: $\boxed{2.000 \text{ L}}$ \\
\textit{Explicação}: Fator de conversão:
\begin{align*}
1 \text{ m}^3 &= 1.000 \text{ L} \\
2 \text{ m}^3 &= 2.000 \text{ L}
\end{align*}

\item \textbf{Densidade}: $\boxed{3 \text{ g/cm}^3}$ \\
\textit{Explicação}: Cálculo da densidade:
\begin{align*}
d &= \frac{m}{V} \\
&= \frac{45 \text{ g}}{15 \text{ cm}^3} \\
&= 3 \text{ g/cm}^3
\end{align*}

\item \textbf{Capacidade da piscina}: $\boxed{48.000 \text{ L}}$ \\
\textit{Explicação}: Cálculo do volume:
\begin{align*}
V &= 8 \times 5 \times 1,\!2 = 48 \text{ m}^3 \\
&= 48 \times 1.000 = 48.000 \text{ L}
\end{align*}

\item \textbf{Volume do bloco}: $\boxed{270 \text{ cm}^3}$ \\
\textit{Explicação}: Usando a densidade:
\begin{align*}
V &= \frac{m}{d} \\
&= \frac{810 \text{ g}}{3 \text{ g/cm}^3} \\
&= 270 \text{ cm}^3
\end{align*}

\end{enumerate}

\end{multicols}

\end{document}
