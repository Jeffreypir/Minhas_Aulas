\documentclass[11pt]{article}
\usepackage[utf8]{inputenc}
\usepackage[T1]{fontenc}
\usepackage{newtxtext,newtxmath}
\usepackage{amsmath}
\usepackage{multicol}
\usepackage{geometry}
\usepackage{tikz}
\usetikzlibrary{arrows.meta}
\usepackage{enumitem}
\usepackage{xcolor}
\usepackage{titlesec}

% Configurações de layout
\geometry{a4paper, left=1cm, right=1cm, top=0.5cm, bottom=1.2cm}
\setlength{\columnseprule}{0.4pt}
\setlength{\baselineskip}{1.0\baselineskip}

% Cores para títulos
\titleformat{\section}{\normalfont\Large\bfseries\color{blue}}{\thesection}{1em}{}
\titleformat{\

\title{\textcolor{blue}{Resolução de Equações do 1º Grau no Excel - Guia Prático}}
\author{Professor: Jefferson}
\date{}

\begin{document}

\maketitle
\vspace{-1cm}

\begin{center}
\large{Nome: \underline{\hspace{8cm}} \quad Turma: \underline{\hspace{3cm}}}
\end{center}

\begin{multicols}{2}

\section*{Introdução Interativa}
As equações do 1º grau modelam situações reais como:
\begin{itemize}
    \item Cálculo de distâncias
    \item Orçamentos domésticos
    \item Planos de telefonia
\end{itemize}

\section*{1. Método Direto (Fórmula)}
\
Resolva no Excel:
\[ 2x + 8 = 0 \]

\begin{enumerate}
    \item Preencha:

    \begin{tabular}{|l|c|}
    \hline
    \textbf{Célula} & \textbf{Valor} \\ \hline
    A1 (Descrição) & "Coeficiente a" \\ \hline
    B1 (Valor) & 2 \\ \hline
    A2 & "Coeficiente b" \\ \hline
    B2 & 8 \\ \hline
    \end{tabular}
    
    \item Insira em B3: \texttt{=-B2/B1}
    \item Resultado deve aparecer: \boxed{-4}
\end{enumerate}

\section*{2. Método Atingir Meta}
\
\begin{enumerate}
    \item Digite em:
    \begin{itemize}
        \item A1: "Valor de x" (deixe vazio)
        \item B1: \texttt{=3*A1 - 6} (para 3x - 6 = 0)
    \end{itemize}
    
    \item Acesse: \textcolor{blue}{Dados} $\rightarrow$ \textcolor{blue}{Análise de Hipóteses} $\rightarrow$ \textcolor{blue}{Atingir Meta}
    
    \item Preencha:
    \begin{tabular}{|l|l|}
    \hline
    \textbf{Opção} & \textbf{Valor} \\ \hline
    Definir célula & B1 \\ \hline
    Para valor & 0 \\ \hline
    Alterando célula & A1 \\ \hline
    \end{tabular}
\end{enumerate}

\section*{3. Método Gráfico Visual}
\
\begin{tabular}{|c|c|}
\hline
\textbf{x} & \textbf{y=2x-4} \\ \hline
-2 & \texttt{=2*A2-4} \\ \hline
0 & \texttt{=2*A3-4} \\ \hline
2 & \texttt{=2*A4-4} \\ \hline
\end{tabular}

\
\begin{enumerate}
    \item Selecione os dados
    \item \textcolor{blue}{Inserir} $\rightarrow$ \textcolor{blue}{Gráfico de Dispersão}
    \item Observe onde a linha cruza o eixo x
\end{enumerate}

\begin{center}
\begin{tikzpicture}[scale=0.5]
    \draw[->] (-3,0) -- (3,0) node[right]{$x$};
    \draw[->] (0,-5) -- (0,1) node[above]{$y$};
    \draw[blue, thick] (-2,-8) -- (0,-4) -- (2,0) -- (3,2);
    \filldraw (2,0) circle (3pt) node[above right]{Solução (2,0)};
\end{tikzpicture}
\end{center}

\section*{Atividades Práticas}

\par
1. Resolva no Excel:

\[ 5x + 10 = 0 \]

2. Use o método gráfico para:
\[ x - 3 = 0 \]
3. Uma operadora cobra R\$ 30,00 fixos + R\$ 0,50 por minuto. Com R\$ 50,00, quantos minutos são possíveis?
\[ 0.5x + 30 = 50 \]

4. Plote e resolva:
\[ \frac{x}{2} + 4 = 0 \]

\
5. Compare dois planos:
\begin{itemize}
    \item Plano A: R\$ 40 + R\$ 1,20/min
    \item Plano B: R\$ 60 + R\$ 0,80/min
\end{itemize}
Determine quando os planos se igualam.

6. Resolva graficamente:
\[ 2(x - 3) + 4 = 0 \]

\section*{Situações-Problema}

7. Resolva utilizando o método direto:
\[ -3x + 9 = 0 \]

8. Use a função Atingir Meta para encontrar a solução de:
\[ \frac{x}{4} - 2 = 0 \]

\
9. Um táxi cobra R\$ 4,50 de bandeirada mais R\$ 2,30 por km. Qual a distância máxima que pode ser percorrida com R\$ 30,00?
\[ 2.3x + 4.5 = 30 \]

10. Plote e resolva graficamente a equação:
\[ -x + 5 = 2x - 1 \]

\
11. Compare três operadoras de internet:
\begin{itemize}
    \item Operadora X: R\$ 80 fixos + R\$ 0,10 por MB excedente
    \item Operadora Y: R\$ 50 fixos + R\$ 0,25 por MB excedente
    \item Operadora Z: R\$ 120 fixos (ilimitado)
\end{itemize}
Determine em qual consumo as Operadoras X e Y se tornam equivalentes.

12. Modele e resolva:
Um reservatório perde 2 litros por hora. Após 6 horas, restam 48 litros. Qual era o volume inicial?
\[ V - 2 \times 6 = 48 \]

\
\
Um investimento inicial de R\$ 500,00 rende R\$ 20,00 por mês. Em quanto tempo o montante atingirá R\$ 800,00?
\[ 500 + 20x = 800 \]

\
13. Uma empresa de fretes cobra R\$ 150,00 por viagem mais R\$ 0,80 por km. Para um orçamento de R\$ 300,00, qual é o raio máximo de entrega?
\[ 0.8x + 150 = 300 \]
\

14. Sua família gasta R\$ 0,80 por kWh de energia. Este mês a conta foi R\$ 120,00 para 150 kWh. Quantos kWh podem ser usados para não ultrapassar R\$ 100,00?

\

15. Para o festival da escola, a turma gastou R\$ 200,00 na decoração e R\$ 5,00 por convite. Se arrecadaram R\$ 10,00 por convite, quantos precisam vender para ter lucro?

\

\end{multicols}

\end{document}
