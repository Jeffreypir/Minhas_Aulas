\documentclass[11pt]{article}
\usepackage[utf8]{inputenc}
\usepackage[T1]{fontenc}
\usepackage{newtxtext,newtxmath}
\usepackage{amsmath}
\usepackage{multicol}
\usepackage{geometry}
\usepackage{enumitem}
\usepackage{xcolor}
\usepackage{titlesec}

% Configurações de layout
\geometry{a4paper, left=1cm, right=1cm, top=0.5cm, bottom=1.2cm}
\setlength{\columnseprule}{0.4pt}
\setlength{\baselineskip}{1.0\baselineskip}

% Cores para títulos
\titleformat{\section}{\normalfont\Large\color{blue}}{\thesection}{1em}{}
\titleformat{\subsection}{\normalfont\large\color{red}}{\thesubsection}{1em}{}

\title{\textcolor{blue}{Questionário: \textit{Nosedive} (Black Mirror)}}
\author{Professor: Jefferson}
\date{}

\begin{document}

\maketitle
\vspace{-1cm}

\begin{multicols}{2}

\section*{Texto de Apoio}

O episódio \textit{Nosedive}, da série \textit{Black Mirror}, retrata uma sociedade em que a vida das pessoas é fortemente influenciada por um sistema de avaliações sociais. Cada interação cotidiana resulta em uma pontuação dada pelos outros, determinando o status social e o acesso a oportunidades. A protagonista, Lacie Pound, busca desesperadamente aumentar sua pontuação para obter benefícios exclusivos, como a possibilidade de morar em uma residência de alto padrão.

A narrativa ilustra como a busca por aprovação pode levar a comportamentos artificiais e relações superficiais. A necessidade constante de agradar os outros torna-se uma obsessão, afetando a autenticidade e a saúde emocional dos indivíduos. Pequenos deslizes resultam em quedas abruptas na pontuação, levando a consequências severas, como a perda de direitos e status.

Além disso, o episódio levanta questões sobre o impacto das redes sociais na autoestima e no comportamento humano. Em um mundo onde a reputação online tem mais valor do que a personalidade real, a pressão para manter uma imagem idealizada se torna sufocante. Esse cenário faz uma analogia com a realidade contemporânea, onde métricas digitais, como curtidas e seguidores, influenciam a forma como as pessoas se veem e são vistas pelos outros.

\section*{Parte 1: Compreensão do Episódio}

\subsection*{Questões Objetivas}

1. Qual é a profissão inicial de Lacie Pound?\\
2. Qual o sistema de avaliação social mostrado no episódio usa?\\
3. Qual evento social Lacie deseja participar para subir sua avaliação?\\
4. O que acontece quando a pontuação de uma pessoa cai drasticamente?\\
5. Qual o impacto do sistema de avaliação no transporte público e na habitação?\\
6. Como os comportamentos das pessoas são influenciados pelo medo de receber notas baixas?\\

\subsection*{Questões de Curta Resposta}

7. Descreva em 1-2 frases por que o episódio se chama "Nosedive" (Mergulho Livre).\\
8. Qual é o objeto físico que representa a avaliação social no episódio?\\
9. Como a amizade entre Lacie e Naomi reflete a superficialidade das relações nesse sistema?\\
10. O que acontece no final do episódio e qual a simbologia dessa cena?\\

\section*{Parte 2: Análise Crítica}
\subsection*{Questões Dissertativas}

11. Explique como o sistema de avaliação afeta o comportamento das pessoas no episódio.\\
12. O episódio mostra que as pessoas se moldam para agradar os outros. Como isso se compara ao que acontece nas redes sociais atuais?\\
13. Se o sistema de avaliação de \textit{Nosedive} existisse na nossa realidade, quais seriam as vantagens e desvantagens?\\
14. O episódio critica a forma como a sociedade valoriza aparências e status. Você concorda que isso acontece no mundo real? Justifique sua resposta.\\
15. Qual personagem do episódio você considera mais autêntico? Por quê?\\

\section*{Parte 3: Conexão Pessoal}

16. Se você vivesse nessa sociedade, que estratégia usaria para manter uma alta avaliação sem perder sua autenticidade?\\
17. Você já se sentiu pressionado a agir de determinada forma para ganhar aprovação nas redes sociais? Como isso afetou seu comportamento?\\
18. Como o episódio pode influenciar a forma como usamos as redes sociais hoje?\\
19. Se você pudesse modificar um elemento do episódio para torná-lo ainda mais impactante, o que mudaria?\\

\end{multicols}

\end{document}

