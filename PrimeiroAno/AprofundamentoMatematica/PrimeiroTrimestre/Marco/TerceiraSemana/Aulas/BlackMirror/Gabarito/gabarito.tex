\documentclass[11pt]{article}
\usepackage[utf8]{inputenc}
\usepackage[T1]{fontenc}
\usepackage{newtxtext,newtxmath}
\usepackage{amsmath}
\usepackage{multicol}
\usepackage{geometry}
\usepackage{enumitem}
\usepackage{xcolor}
\usepackage{titlesec}

% Configurações de layout
\geometry{a4paper, left=1cm, right=1cm, top=0.5cm, bottom=1.2cm}
\setlength{\columnseprule}{0.4pt}
\setlength{\baselineskip}{1.0\baselineskip}

% Cores para títulos
\titleformat{\section}{\normalfont\Large\color{blue}}{\thesection}{1em}{}
\titleformat{\subsection}{\normalfont\large\color{red}}{\thesubsection}{1em}{}

\title{\textcolor{blue}{Question\'ario: \textit{Nosedive} (Black Mirror) - Respostas}}
\author{Professor: Jefferson}
\date{}

\begin{document}

\maketitle
\vspace{-1cm}

\begin{multicols}{2}

\section*{Parte 1: Compreens\~ao do Epis\'odio}

\subsection*{Quest\~oes Objetivas}

1. Qual \'e a profiss\~ao inicial de Lacie Pound?\\
\textbf{Ela trabalha em um escritório corporativo.}

2. Qual o sistema de avaliação social mostrado no episódio usa?\\
\textbf{b) Avaliação por estrelas de 1 a 5.}

3. Qual evento social Lacie deseja participar para subir sua avaliação?\\
\textbf{O casamento de sua amiga de infância, Naomi.}

4. O que acontece quando a pontuação de uma pessoa cai drasticamente?\\
\textbf{Ela perde privilégios sociais, como acesso a transportes, serviços e moradias de melhor qualidade.}

5. Qual o impacto do sistema de avaliação no transporte público e na habitação?\\
\textbf{Pessoas com baixa pontuação não podem acessar certos meios de transporte e moradias de alto padrão.}

6. Como os comportamentos das pessoas são influenciados pelo medo de receber notas baixas?\\
\textbf{Elas evitam conflitos, agem de maneira superficial e tentam agradar constantemente os outros.}

\subsection*{Quest\~oes de Curta Resposta}

7. Descreva em 1-2 frases por que o episódio se chama "Nosedive" (Mergulho Livre).\\
\textbf{Porque a pontuação de Lacie despenca abruptamente ao longo do episódio, refletindo sua queda na sociedade.}

8. Qual é o objeto físico que representa a avaliação social no episódio?\\
\textbf{Um smartphone com um sistema de classificação por estrelas.}

9. Como a amizade entre Lacie e Naomi reflete a superficialidade das relações nesse sistema?\\
\textbf{Naomi convida Lacie apenas por causa de sua pontuação, não por uma amizade genuína.}

10. O que acontece no final do episódio e qual a simbologia dessa cena?\\
\textbf{Lacie é presa e, na cela, se liberta ao não precisar mais se preocupar com sua pontuação, simbolizando sua libertação do sistema.}

\section*{Parte 2: An\'alise Cr\'itica}

11. Explique como o sistema de avaliação afeta o comportamento das pessoas no episódio.\\
\textbf{As pessoas vivem para agradar os outros, escondendo suas verdadeiras emoções e sentimentos para manter uma boa pontuação.}

12. O episódio mostra que as pessoas se moldam para agradar os outros. Como isso se compara ao que acontece nas redes sociais atuais?\\
\textbf{Nas redes sociais, muitos buscam aprovação por meio de curtidas e seguidores, moldando sua imagem digital para serem aceitos socialmente.}

13. Se o sistema de avaliação de \textit{Nosedive} existisse na nossa realidade, quais seriam as vantagens e desvantagens?\\
\textbf{Vantagens: incentivo à educação e boas maneiras. Desvantagens: perda de autenticidade e exclusão social injusta.}

14. O episódio critica a forma como a sociedade valoriza aparências e status. Você concorda que isso acontece no mundo real? Justifique sua resposta.\\
\textbf{Sim, pois muitas pessoas valorizam status e aparências nas redes sociais, priorizando a aceitação social ao invés da autenticidade.}

15. Qual personagem do episódio você considera mais autêntico? Por quê?\\
\textbf{A caminhoneira Susan, pois ela vive sem se preocupar com a pontuação e age de maneira genuína.}

\section*{Parte 3: Conex\~ao Pessoal}

16. Se você vivesse nessa sociedade, que estratégia usaria para manter uma alta avaliação sem perder sua autenticidade?\\
\textbf{Ser educado e respeitoso, mas sem comprometer meus valores ou fingir ser algo que não sou.}

17. Você já se sentiu pressionado a agir de determinada forma para ganhar aprovação nas redes sociais? Como isso afetou seu comportamento?\\
\textbf{Sim, em alguns momentos senti a necessidade de seguir tendências para ser aceito, o que gerou ansiedade.}

18. Como o episódio pode influenciar a forma como usamos as redes sociais hoje?\\
\textbf{Mostra os perigos da obsessão por validação digital e incentiva a autenticidade e o equilíbrio.}

19. Se você pudesse modificar um elemento do episódio para torná-lo ainda mais impactante, o que mudaria?\\
\textbf{Tornaria o final mais aberto, mostrando o futuro de Lacie após sua libertação do sistema.}

\end{multicols}

\end{document}
