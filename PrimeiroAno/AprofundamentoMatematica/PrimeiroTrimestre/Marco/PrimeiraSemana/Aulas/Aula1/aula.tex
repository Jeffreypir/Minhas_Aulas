\documentclass[11pt]{article}
\usepackage[utf8]{inputenc}
\usepackage{amsmath}
\usepackage{multicol}
\usepackage{geometry}
\usepackage{tikz}
\usetikzlibrary{arrows.meta}
\usepackage{enumitem} % Para listas personalizadas
\usepackage{xcolor} % Para usar cores
\usepackage{titlesec} % Para personalizar títulos

\geometry{a4paper, left=1.5cm, right=1.5cm, top=1.5cm, bottom=1.5cm}

% Definir a cor das seções como azul
\titleformat{\section}
  {\normalfont\Large\bfseries\color{blue}} % Formato do título
  {\thesection} % Número da seção
  {1em} % Espaço entre número e título
  {} % Código antes do título

\renewcommand{\thesubsection}{\textcolor{red}{\arabic{section}.\arabic{subsection}}}
\titleformat{\subsection}{\color{red}\normalfont\bfseries}{\thesubsection}{1em}{}
\title{\textcolor{blue}{Notação Científica e Grandezas Aplicadas à Informática}}
\author{Professor: Jefferson}
\date{}

\begin{document}

\maketitle

\begin{center}
\large{Nome: \underline{\hspace{8cm}} \quad Série-Turma: \underline{\hspace{3cm}}
\end{center}

\setlength{\columnsep}{20pt} % Espaço entre as colunas
\setlength{\columnseprule}{0.4pt} % Espessura da linha de separação

\begin{multicols}{2}

\section*{1. Grandezas na Informática}

\subsection*{Unidades de Medida}
\begin{itemize}
    \item \textbf{Bit (b)}: Menor unidade de informação (0 ou 1).
    \item \textbf{Byte (B)}: 8 bits.
    \item \textbf{Quilobyte (KB)}: \(1.024\) bytes.
    \item \textbf{Megabyte (MB)}: \(1.024\) KB.
    \item \textbf{Gigabyte (GB)}: \(1.024\) MB.
    \item \textbf{Terabyte (TB)}: \(1.024\) GB.
    \item \textbf{Petabyte (PB)}: \(1.024\) TB.
\end{itemize}

\subsection*{Exemplo 1:}
\textbf{Problema:} Converta 5 GB para bytes.

\textbf{Resolução:}
\[
5 \, \text{GB} = 5 \times 1.024 \times 1.024 \times 1.024 \, \text{bytes} \approx 5 \times 10^9 \, \text{bytes}.
\]

\subsection*{Exemplo 2:}
\textbf{Problema:} Um arquivo de 2 MB é igual a quantos KB?

\textbf{Resolução:}
\[
2 \, \text{MB} = 2 \times 1.024 \, \text{KB} \approx  2.048 \, \text{KB}.
\]

\subsection*{Atividades}
1. Converta 3 TB para bytes. \\
2. Um pendrive tem capacidade de 64 GB. Quantos bytes ele pode armazenar? \\
3. Um filme em alta definição tem 4,7 GB. Quantos MB ele ocupa? \\
4. Explique a diferença entre bit e byte.

\section*{2. Notação Científica na Informática}

\subsection*{Aplicações Práticas}
\begin{itemize}
    \item \textbf{Tamanho de Arquivos}:
    \begin{itemize}
        \item 1 TB = \(1 \times 10^{12}\) bytes.
        \item 1 GB = \(1 \times 10^9\) bytes.
    \end{itemize}
    \item \textbf{Velocidade de Transmissão}:
    \begin{itemize}
        \item 1 Gbps (Gigabit por segundo) = \(1 \times 10^9\) bits por segundo.
    \end{itemize}
    \item \textbf{Capacidade de Armazenamento}:
    \begin{itemize}
        \item Um HD de 2 TB = \(2 \times 10^{12}\) bytes.
    \end{itemize}
\end{itemize}

\subsection*{Exemplo 1:}
\textbf{Problema:} Converta 0,000000001 segundos para notação científica.

\textbf{Resolução:}
\[
0,000000001 \, \text{s} = 1 \times 10^{-9} \, \text{s}.
\]

\subsection*{Exemplo 2:}
\textbf{Problema:} Qual é a velocidade em bits por segundo de uma conexão de 10 Mbps?

\textbf{Resolução:}
\[
10 \, \text{Mbps} = 10 \times 10^6 \, \text{bits por segundo}.
\]

\subsection*{Atividades}
1. Converta 500 GB para bytes usando notação científica. \\
2. Qual é a velocidade em bits por segundo de uma conexão de 100 Mbps? \\
3. Um SSD tem capacidade de 1 TB. Quantos bytes ele pode armazenar? \\
4. Explique por que a notação científica é útil na informática.

\section*{3. Exemplos Práticos}

\subsection*{Exemplo 1:}
\textbf{Problema:} Um serviço de streaming transmite vídeos a 5 Mbps. Quantos bits são transmitidos em 1 minuto?

\textbf{Resolução:}
\[
5 \, \text{Mbps} = 5 \times 10^6 \, \text{bits por segundo}.
\]
\[
1 \, \text{minuto} = 60 \, \text{segundos}.
\]
\[
\text{Bits transmitidos} = 5 \times 10^6 \times 60 = 3 \times 10^8 \, \text{bits}.
\]

\subsection*{Exemplo 2:}
\textbf{Problema:} Um HD externo tem capacidade de 4 TB. Quantos arquivos de 500 MB cabem no HD?

\textbf{Resolução:}
\[
4 \, \text{TB} = 4 \times 1.024 \times 1.024 \, \text{MB} \approx  4.194.304 \, \text{MB}.
\]
\[
\text{Número de arquivos} = \frac{4.194.304}{500} \approx 8.389 \, \text{arquivos}.
\]

\subsection*{Atividades}
1. Um filme de 2 GB é baixado a uma velocidade de 10 Mbps. Quanto tempo levará para o download? \\
2. Um pendrive de 32 GB é usado para armazenar fotos de 4 MB cada. Quantas fotos cabem no pendrive? \\
3. Um SSD de 500 GB tem 80\% de sua capacidade ocupada. Quantos bytes estão livres? \\
4. Explique como a notação científica pode ser usada para comparar tamanhos de arquivos.

\section*{4. Conclusão}

\subsection*{Resumo}
\begin{itemize}
    \item As grandezas na informática são essenciais para medir tamanhos de arquivos, velocidades de transmissão e capacidades de armazenamento.
    \item A notação científica facilita a representação de números muito grandes ou muito pequenos.
    \item Compreender essas grandezas é fundamental para o uso eficiente de recursos tecnológicos.
\end{itemize}

\subsection*{Atividade Final}
1. Crie uma tabela comparando as unidades de medida (bit, byte, KB, MB, GB, TB) e suas respectivas ordens de grandeza. \\
2. Resolva os problemas propostos nas atividades anteriores e explique suas resoluções.

\end{multicols}

\end{document}
