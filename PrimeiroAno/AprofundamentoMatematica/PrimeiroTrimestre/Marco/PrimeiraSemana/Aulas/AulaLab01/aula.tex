\documentclass[11pt]{article}
\usepackage[utf8]{inputenc}
\usepackage{amsmath}
\usepackage{multicol}
\usepackage{geometry}
\usepackage{tikz}
\usetikzlibrary{arrows.meta}
\usepackage{enumitem}
\usepackage{xcolor}
\usepackage{titlesec}

\geometry{a4paper, left=1cm, right=1cm, top=1.5cm, bottom=1.5cm}

% Definir a cor das seções como azul
\titleformat{\section}
  {\normalfont\Large\bfseries\color{blue}}
  {\thesection}
  {1em}
  {}

\renewcommand{\thesubsection}{\textcolor{red}{\arabic{section}.\arabic{subsection}}}
\titleformat{\subsection}{\color{red}\normalfont\bfseries}{\thesubsection}{1em}{}
\title{\textcolor{blue}{Aula Prática no Laboratório: \\ Pensamento Computacional com Excel}}
\author{Professor: Jefferson}
\date{}

\begin{document}

\maketitle

\begin{multicols}{2}

\section*{Introdução ao Excel}

O Excel é uma ferramenta poderosa que pode ser usada para aplicar o pensamento computacional. Ele permite organizar dados, realizar cálculos e automatizar tarefas. Vamos explorar como usar o Excel para resolver problemas de forma eficiente.

\section*{Exemplo Prático: Gerenciamento de Finanças Pessoais}

Vamos aplicar o pensamento computacional para gerenciar finanças pessoais usando o Excel.

\subsection*{1. Decomposição}
Divida o problema em tarefas menores:
\begin{itemize}
    \item Listar todas as fontes de renda.
    \item Listar todas as despesas.
    \item Calcular o saldo mensal.
    \item Identificar áreas para economizar.
\end{itemize}

\subsection*{2. Reconhecimento de Padrões}
Identifique padrões:
\begin{itemize}
    \item Despesas recorrentes (aluguel, contas, etc.).
    \item Rendas fixas e variáveis.
\end{itemize}

\subsection*{3. Abstração}
Foque no essencial:
\begin{itemize}
    \item Não é necessário detalhar cada pequena despesa, apenas categorias principais.
    \item Concentre-se nas despesas que têm maior impacto no orçamento.
\end{itemize}

\subsection*{4. Algoritmos}
Crie um passo a passo:
\begin{enumerate}
    \item Crie uma planilha no Excel.
    \item Liste todas as fontes de renda em uma coluna.
    \item Liste todas as despesas em outra coluna.
    \item Use fórmulas para calcular o saldo mensal.
    \item Analise os dados para identificar áreas de economia.
\end{enumerate}

\section*{Atividade Prática}

\subsection*{Atividade 1: Criando uma Planilha de Orçamento}
Crie uma planilha no Excel para gerenciar suas finanças pessoais. Siga os passos abaixo:
\begin{enumerate}
    \item Abra o Excel e crie uma nova planilha.
    \item Na coluna A, liste todas as suas fontes de renda.
    \item Na coluna B, insira os valores correspondentes a cada fonte de renda.
    \item Na coluna C, liste todas as suas despesas.
    \item Na coluna D, insira os valores correspondentes a cada despesa.
    \item Use a fórmula \texttt{=SOMA(B2:B10)} para calcular o total de renda.
    \item Use a fórmula \texttt{=SOMA(D2:D10)} para calcular o total de despesas.
    \item Use a fórmula \texttt{=B11-D11} para calcular o saldo mensal.
\end{enumerate}

\subsection*{Atividade 2: Analisando Dados}
Analise os dados da sua planilha de orçamento e responda:
\begin{itemize}
    \item Qual é a sua principal fonte de renda?
    \item Qual é a sua maior despesa?
    \item Em quais áreas você pode economizar?
\end{itemize}

\end{multicols}

\end{document}
