\documentclass[12pt]{article}
\usepackage[utf8]{inputenc}
\usepackage{amsmath}
\usepackage{multicol}
\usepackage{geometry}
\usepackage{tikz}
\usetikzlibrary{arrows.meta}
\usepackage{enumitem}
\usepackage{xcolor}
\usepackage{titlesec}

\geometry{a4paper, left=1cm, right=1cm, top=1.5cm, bottom=1.5cm}

\titleformat{\section}
  {\normalfont\Large\bfseries\color{blue}}
  {\thesection}
  {1em}
  {}

\renewcommand{\thesubsection}{\textcolor{red}{\arabic{section}.\arabic{subsection}}}
\titleformat{\subsection}{\color{red}\normalfont\bfseries}{\thesubsection}{1em}{}

\title{\textcolor{blue}{Uso do Excel para Resolver Problemas de Grandezas Proporcionais}}
\author{Professor: Jefferson}
\date{}

\begin{document}

\maketitle

\begin{center}
\large{Nome: \underline{\hspace{8cm}} \quad Série-Turma: \underline{\hspace{3cm}}}
\end{center}

\begin{multicols}{2}

\section*{Introdução ao Uso do Excel}

O Microsoft Excel é uma ferramenta poderosa para resolver problemas matemáticos de maneira eficiente. Ele permite criar tabelas, aplicar fórmulas e visualizar gráficos para entender melhor a relação entre as grandezas.

\subsection*{Fórmulas Básicas no Excel}
As fórmulas do Excel seguem uma estrutura simples, sempre iniciando com ``=``. Algumas fórmulas úteis incluem:

\begin{itemize}
    \item Soma: \texttt{=SOMA(A1:A5)}
    \item Multiplicação: \texttt{=A1*A2}
    \item Regra de Três Simples: \texttt{=(B1*C1)/A1}
    \item Regra de Três Composta: \texttt{=(B1*C1*D1)/(A1*E1)}
\end{itemize}


\subsection*{Exemplos de Operações no Excel}

\begin{itemize}
    \item Soma: Para somar os valores das células A1 e A2, a fórmula é:
    \[
    \mathtt{= A1 + A2}
    \]
    Exemplo: Se A1 = 10 e A2 = 5, o resultado será 15.

    \item Subtração: Para subtrair o valor de A2 de A1, a fórmula é:
    \[
    \mathtt{= A1 - A2}
    \]
    Exemplo: Se A1 = 10 e A2 = 5, o resultado será 5.

    \item Multiplicação: Para multiplicar os valores de A1 e A2, a fórmula é:
    \[
    \mathtt{= A1 \times A2}
    \]
    Exemplo: Se A1 = 10 e A2 = 5, o resultado será 50.

    \item Divisão: Para dividir o valor de A1 por A2, a fórmula é:
    \[
    \mathtt{= \frac{A1}{A2}}
    \]
    Exemplo: Se A1 = 10 e A2 = 5, o resultado será 2.
\end{itemize}

\section*{Grandezas Diretas e Inversas}

\subsection*{Grandezas Diretamente Proporcionais}
Quando duas grandezas aumentam ou diminuem juntas, dizemos que são diretamente proporcionais. No Excel, podemos calcular o valor desconhecido com a fórmula:
\[
X = \frac{B1 \times C1}{A1}
\]
Exemplo:
Se 5 máquinas produzem 1200 peças quantas peças produziram 12 máquinas?
\[
\mathtt{= \frac{(1200 \times 12)}{5}} \Rightarrow 2880 \, \text{dias}.
\]

\subsection*{Grandezas Inversamente Proporcionais}
Quando uma grandeza aumenta e a outra diminui, são inversamente proporcionais. A fórmula no Excel é:
\[
X = \frac{A1 \times B1}{C1}
\]
Exemplo:
Se um carro a 60 km/h faz um trajeto em 3 horas, quanto tempo levará a 80 km/h?
\[
\mathtt{= \frac{(60 \times 3)}{80}} \Rightarrow 2,25 \, \text{horas}.
\]

\section*{Atividade: Aplicação no Excel}

Crie uma planilha no Excel e resolva os seguintes problemas utilizando fórmulas:

\subsection*{Questão 1}
Um caminhão transporta 100 sacos de cimento em 5 viagens. Quantos sacos levará em 8 viagens?

\subsection*{Questão 2}
Se 4 torneiras enchem um tanque em 10 horas, quantas horas levarão 6 torneiras?

\subsection*{Questão 3}
Uma equipe de 8 trabalhadores constrói um muro em 15 dias. Quantos dias levará uma equipe de 10 trabalhadores?

\end{multicols}

\end{document}

