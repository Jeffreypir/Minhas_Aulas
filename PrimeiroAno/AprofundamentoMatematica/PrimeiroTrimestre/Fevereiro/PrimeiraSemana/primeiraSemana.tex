\documentclass{article}
\usepackage[utf8]{inputenc}
\usepackage{amsmath}
\usepackage{amssymb}
\usepackage{geometry}
\geometry{a4paper, margin=1.5cm}

\title{Planejamento Semanal de Aulas de Matemática}
\author{}
\date{}

\begin{document}

\maketitle

\section{Objetivos}
\begin{itemize}
    \item Explorar os conjuntos numéricos (naturais, inteiros, racionais e reais).
    \item Desenvolver habilidades de manipulação algébrica.
\end{itemize}

\section{Aula 1: Explorando os Conjuntos Numéricos}

\subsection{Duração:}
1 hora e 30 minutos.

\subsection{Conteúdo:}

\begin{enumerate}
    \item \textbf{Introdução aos Conjuntos Numéricos:}
    \begin{itemize}
        \item \textbf{Naturais ($\mathbb{N}$):} Números positivos usados para contar (1, 2, 3, \dots).
        \item \textbf{Inteiros ($\mathbb{Z}$):} Inclui os naturais e seus opostos (\dots, 2, 1, 0, 1, 2, \dots).
        \item \textbf{Racionais ($\mathbb{Q}$):} Números que podem ser expressos como frações ($\frac{a}{b}$, onde $a$ e $b$ são inteiros e $b \neq 0$).
        \item \textbf{Reais ($\mathbb{R}$):} Inclui todos os racionais e irracionais (números que não podem ser expressos como frações, como $\sqrt{2}$ e $\pi$).
    \end{itemize}

    \item \textbf{Exemplos:}
    \begin{itemize}
        \item \textbf{Naturais:} Contagem de objetos, como 5 maçãs.
        \item \textbf{Inteiros:} Temperaturas abaixo de zero, como $5^\circ$C.
        \item \textbf{Racionais:} Frações como $\frac{1}{2}$, que representa metade de uma pizza.
        \item \textbf{Reais:} Medidas de comprimento, como 1,4142... metros ($\sqrt{2}$).
    \end{itemize}

    \item \textbf{Atividades:}
    \begin{itemize}
        \item \textbf{Classificação de Números:} Fornecer uma lista de números e pedir aos alunos para classificálos em naturais, inteiros, racionais ou reais.
        \item \textbf{Discussão em Grupo:} Dividir a turma em grupos e pedir que discutam e apresentem exemplos práticos de cada conjunto numérico.
    \end{itemize}
\end{enumerate}

\subsection{Recursos Necessários:}
\begin{itemize}
    \item Quadro branco e marcadores.
    \item Lista de números para classificação.
    \item Material de apoio (folhas com exemplos e explicações).
\end{itemize}

\subsection{Questões:}

1. Classifique os números abaixo em naturais ($\mathbb{N}$), inteiros ($\mathbb{Z}$), racionais ($\mathbb{Q}$) ou reais
($\mathbb{R}$): \\
    a) $5$  
    b) $3$  
    c) $\frac{2}{3}$  
    d) $\sqrt{2}$  
    e) $0$  
    f) $1,5$  
    g) $\pi$  
    h) $\frac{7}{2}$   \\

   Respostas:
    a) $\mathbb{N}, \mathbb{Z}, \mathbb{Q}, \mathbb{R}$  
    b) $\mathbb{Z}, \mathbb{Q}, \mathbb{R}$  
    c) $\mathbb{Q}, \mathbb{R}$  
    d) $\mathbb{R}$  
    e) $\mathbb{Z}, \mathbb{Q}, \mathbb{R}$  
    f) $\mathbb{Q}, \mathbb{R}$  
    g) $\mathbb{R}$  
    h) $\mathbb{Q}, \mathbb{R}$   \\

2. Escreva três exemplos de números naturais.   \\
   Resposta: $1, 2, 3$ (ou qualquer número positivo inteiro).

3. Escreva três exemplos de números inteiros que não são naturais.  
   Resposta: $1, 2, 3$ (ou qualquer número inteiro negativo). \\

4. Explique por que $\frac{4}{2}$ é um número racional. \\ 
   Resposta: Porque pode ser expresso como uma fração de dois inteiros ($\frac{4}{2} = 2$).

5. Dê um exemplo de número irracional e explique por que ele não é racional.   \\
   Resposta: $\sqrt{3}$ é irracional porque não pode ser expresso como uma fração de dois inteiros.

6. Classifique o número $0,75$ em $\mathbb{N}, \mathbb{Z}, \mathbb{Q}, \mathbb{R}$.   \\
   Resposta: $\mathbb{Q}, \mathbb{R}$.

7. Por que o número $5$ pertence aos conjuntos $\mathbb{Z}, \mathbb{Q}, \mathbb{R}$? \\ 
   Resposta: Porque é um número inteiro, que também é racional e real.

8. Escreva um número que pertença apenas ao conjunto $\mathbb{R}$.   \\
   Resposta: $\sqrt{5}$ (ou qualquer número irracional).

\section{Aula 2: Manipulação Algébrica}

\subsection{Duração:}
1 hora e 30 minutos.

\subsection{Conteúdo:}

\begin{enumerate}
    \item \textbf{Introdução à Manipulação Algébrica:}
    \begin{itemize}
        \item \textbf{Expressões Algébricas:} Combinação de números, variáveis e operações (ex: $3x + 2y$).
        \item \textbf{Simplificação de Expressões:} Combinar termos semelhantes e aplicar propriedades distributivas.
        \item \textbf{Resolução de Equações:} Isolar a variável para encontrar seu valor.
    \end{itemize}

    \item \textbf{Exemplos:}
    \begin{itemize}
        \item \textbf{Simplificação:} Simplificar a expressão $3x + 2x  5$ para $5x  5$.
        \item \textbf{Resolução de Equações:} Resolver a equação $2x + 3 = 7$, encontrando $x = 2$.
    \end{itemize}

    \item \textbf{Atividades:}
    \begin{itemize}
        \item \textbf{Prática de Simplificação:} Fornecer uma lista de expressões para simplificar.
        \item \textbf{Resolução de Equações:} Propor equações simples para os alunos resolverem.
        \item \textbf{Jogo de Equações:} Criar um jogo onde os alunos devem resolver equações para avançar em um tabuleiro.
    \end{itemize}
\end{enumerate}

\subsection{Recursos Necessários:}
\begin{itemize}
    \item Quadro branco e marcadores.
    \item Lista de expressões e equações.
    \item Material para o jogo de equações (tabuleiro, dados, cartas com equações).
\end{itemize}

\subsection{Questões:}

1. Simplifique as expressões abaixo:

    a) $3x + 2x  5$  
    b) $4y  y + 7$  
    c) $2a + 3b  a + 5b$  
    d) $5x  3x + 2x  x$  

   Respostas:
    a) $5x  5$  
    b) $3y + 7$  
    c) $a + 8b$  
    d) $3x$  \\ 

2. Resolva as equações abaixo:

    a) $2x + 3 = 7$  
    b) $5y  4 = 11$  
    c) $3a + 2 = 8$  
    d) $4b  6 = 10$  

   Respostas:
    a) $x = 2$  
    b) $y = 3$  
    c) $a = 2$  
    d) $b = 4$   \\

3. Aplique a propriedade distributiva e simplifique:

    a) $2(x + 3)$  
    b) $3(2y  4)$  
    c) $4(a + b  2)$  
    d) $5(2x  3y + 1)$  

   Respostas:
    a) $2x + 6$  
    b) $6y  12$  
    c) $4a + 4b  8$  
    d) $10x  15y + 5$   \\

4. Combine os termos semelhantes:

    a) $3x + 2y  x + 4y$  
    b) $5a  2b + 3a + b$  
    c) $7m + 3n  2m  n$  
    d) $4p + 5q  p  3q$  

   Respostas:
    a) $2x + 6y$  
    b) $8a  b$  
    c) $5m + 2n$  
    d) $3p + 2q$   \\

5. Resolva a equação $3(x + 2) = 15$.  

   Resposta: $x = 3$. \\

6. Simplifique a expressão $2x + 3y  x + 4y  5$.  
   Resposta: $x + 7y  5$. \\

7. Resolva a equação $4(y  3) = 20$.  

   Resposta: $y = 8$. \\

8. Simplifique a expressão $5a  2b + 3a + 4b  6$.  

   Resposta: $8a + 2b  6$. \\

\section{Avaliação:}
\begin{itemize}
    \item \textbf{Participação nas Discussões:} Avaliar a participação e compreensão dos alunos durante as discussões em grupo.
    \item \textbf{Atividades Práticas:} Verificar a precisão e o raciocínio nas atividades de classificação, simplificação e resolução de equações.
    \item \textbf{Feedback:} Fornecer feedback individual e coletivo, destacando pontos fortes e áreas para melhoria.
\end{itemize}

\section{Materiais Adicionais:}
\begin{itemize}
    \item Apostilas com resumos dos conceitos abordados.
    \item Links para vídeos educacionais sobre conjuntos numéricos e manipulação algébrica.
    \item Exercícios extras para prática em casa.
\end{itemize}

\end{document}
