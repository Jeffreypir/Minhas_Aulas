\documentclass[11pt]{article}
\usepackage[utf8]{inputenc}
\usepackage[T1]{fontenc}
\usepackage{amsmath}
\usepackage{multicol}
\usepackage{geometry}
\usepackage{tikz}
\usetikzlibrary{shapes.geometric, arrows.meta, calc}
\usepackage{enumitem}
\usepackage{xcolor}
\usepackage{titlesec}
\usepackage{tcolorbox}

% Configurações de layout
\geometry{a4paper, left=1cm, right=1cm, top=0.5cm, bottom=1.2cm}
\setlength{\columnseprule}{0.4pt}
\setlength{\baselineskip}{1.0\baselineskip}

% Cores personalizadas
\definecolor{titleblue}{RGB}{0,80,150}
\definecolor{sectionred}{RGB}{180,0,0}
\definecolor{darkgreen}{RGB}{0,100,0}
\definecolor{explanationbg}{RGB}{240,248,255}

% Formatação de títulos
\titleformat{\section}{\normalfont\Large\bfseries\color{titleblue}}{\thesection}{1em}{}
\titleformat{\subsection}{\normalfont\large\bfseries\color{sectionred}}{\thesubsection}{1em}{}
\titleformat{\subsubsection}{\normalfont\normalsize\bfseries\color{darkgreen}}{\thesubsubsection}{1em}{}

\title{\textcolor{titleblue}{Atividade Avaliativa: Perímetro}}
\author{Professor: Jefferson}
\date{}

\begin{document}

\maketitle
\vspace{-1cm}

\begin{center}
    \large{\textbf{Observação:} Respostas no caderno com letra legível. \quad Série: 1 Ano. Valor: 1,0 }
\end{center}

\begin{multicols}{2}

\section*{Atividade}
\begin{enumerate}

\item \textbf{Retângulo}\\
Um retângulo tem perímetro 34 cm. Se o comprimento é 5 cm maior que a largura, encontre as dimensões.
\begin{tcolorbox}[colback=explanationbg,colframe=titleblue,title=Dica:]
Chame a largura de $x$ e o comprimento de $x+5$. \\ Perímetro $= 2x + 2(x+5) = 34$
\end{tcolorbox}

\item \textbf{Quadrado}\\
O perímetro de um quadrado é igual a 3 vezes a medida de seu lado mais 20 cm. Qual é seu lado?
\begin{tcolorbox}[colback=explanationbg,colframe=titleblue,title=Dica:]
Perímetro do quadrado $= 4x$ \\ Equação: $4x = 3x + 20$
\end{tcolorbox}

\item \textbf{Triângulo Equilátero}\\
Um triângulo equilátero tem perímetro 12 cm a menos que um quadrado de lado 10 cm. Qual é seu lado?
\begin{tcolorbox}[colback=explanationbg,colframe=titleblue,title=Dica:]
Perímetro do quadrado $= 40$ cm \\ $3x = 40 - 12$
\end{tcolorbox}

\item \textbf{Retângulo (Desafio)}\\
A largura de um retângulo é a metade do comprimento. Se o perímetro é 60 m, encontre a área.
\begin{tcolorbox}[colback=explanationbg,colframe=titleblue,title=Dica:]
Largura $= x$, Comprimento $= 2x$ \\ Perímetro $= 2(x + 2x) = 60$
\end{tcolorbox}

\item \textbf{Hexágono Regular}\\
Um hexágono regular tem perímetro igual ao de um triângulo equilátero de lado 12 cm. Qual é seu lado?
\begin{tcolorbox}[colback=explanationbg,colframe=titleblue,title=Dica:]
Perímetro do triângulo $= 36$ cm \\ $6x = 36$
\end{tcolorbox}

\item \textbf{Figura Composta}\\
Uma figura formada por 3 quadrados iguais lado a lado tem perímetro 56 cm. Qual é o lado de cada quadrado?
\begin{tcolorbox}[colback=explanationbg,colframe=titleblue,title=Dica:]
Cada quadrado tem lado $x$. Perímetro total $= 8x$ (14 lados expostos - 2 lados internos)
\end{tcolorbox}

\item \textbf{Retângulo (Fração)}\\
A largura de um retângulo é $\frac{2}{3}$ do comprimento. Se o perímetro é 50 cm, encontre as dimensões.
\begin{tcolorbox}[colback=explanationbg,colframe=titleblue,title=Dica:]
Largura $= \frac{2}{3}x$, Comprimento $= x$ \\ $2\left(x + \frac{2}{3}x\right) = 50$
\end{tcolorbox}

\item \textbf{Comparação de Perímetros}\\
Um retângulo tem comprimento igual ao lado de um quadrado de perímetro 28 cm. Se o retângulo tem largura 3 cm e perímetro 26 cm, verifique a afirmação.
\begin{tcolorbox}[colback=explanationbg,colframe=titleblue,title=Dica:]
Encontre primeiro o lado do quadrado ($7$ cm), depois verifique o perímetro do retângulo com comprimento $7$ cm e largura $3$ cm.
\end{tcolorbox}

\item \textbf{Polígono Regular}\\
Um pentágono regular tem perímetro 15 cm maior que o dobro de seu lado. Qual é seu lado?
\begin{tcolorbox}[colback=explanationbg,colframe=titleblue,title=Dica:]
$5x = 2x + 15$
\end{tcolorbox}

\item \textbf{Problema Contextualizado}\\
Uma cerca retangular para cachorros tem perímetro 30 m. Se o comprimento é o triplo da largura, quantos metros de cerca serão necessários para aumentar cada lado em 1 m?
\begin{tcolorbox}[colback=explanationbg,colframe=titleblue,title=Dica:]
Largura original = $x$, Comprimento = $3x$ \\ Novo perímetro = $2[(x+1)+(3x+1)]$
\end{tcolorbox}

\section*{Desafio:}

\item \textbf{Sistema de Equações}\\
Dois retângulos têm o mesmo perímetro. O primeiro tem largura 8 cm e comprimento $x+2$ cm. O segundo tem largura $x$ cm e comprimento 10 cm. Determine $x$ e o perímetro comum.
\begin{tcolorbox}[colback=explanationbg,colframe=titleblue,title=Dica:]
Iguale os perímetros: \\ $2(8 + x + 2) = 2(x + 10)$
\end{tcolorbox}

\end{enumerate}
\end{multicols}

\end{document}
