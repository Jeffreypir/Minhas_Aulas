\documentclass[11pt]{article}
\usepackage[utf8]{inputenc}
\usepackage[T1]{fontenc}
\usepackage{amsmath}
\usepackage{multicol}
\usepackage{geometry}
\usepackage{tikz}
\usetikzlibrary{shapes.geometric, arrows.meta, calc}
\usepackage{enumitem}
\usepackage{xcolor}
\usepackage{titlesec}
\usepackage{tcolorbox}

% Configurações de layout
\geometry{a4paper, left=1cm, right=1cm, top=0.5cm, bottom=1.2cm}
\setlength{\columnseprule}{0.4pt}
\setlength{\baselineskip}{1.0\baselineskip}

% Cores personalizadas
\definecolor{titleblue}{RGB}{0,80,150}
\definecolor{sectionred}{RGB}{180,0,0}
\definecolor{darkgreen}{RGB}{0,100,0}
\definecolor{explanationbg}{RGB}{240,248,255}

% Formatação de títulos
\titleformat{\section}{\normalfont\Large\bfseries\color{titleblue}}{\thesection}{1em}{}
\titleformat{\subsection}{\normalfont\large\bfseries\color{sectionred}}{\thesubsection}{1em}{}
\titleformat{\subsubsection}{\normalfont\normalsize\bfseries\color{darkgreen}}{\thesubsubsection}{1em}{}

\title{\textcolor{titleblue}{Atividade Avaliativa 2: Aprofundamento em Matemática \\ Assunto: Área}}
\author{Professor: Jefferson}
\date{}

\begin{document}

\maketitle
\vspace{-1cm}

\begin{center}
    \large{\textbf{Observação:} Respostas no caderno com letra legível. \quad Série: 1 Ano. Valor: 1,0 }
\end{center}

\begin{multicols}{2}

\section*{Atividade}
\begin{enumerate}

\item \textbf{Retângulo}\\
Um retângulo tem área 48 cm². Se o comprimento é 8 cm, qual é sua largura?
\begin{tcolorbox}[colback=explanationbg,colframe=titleblue,title=Dica:]
Área $=$ comprimento × largura \\ $48 = 8 \times x$
\end{tcolorbox}

\item \textbf{Triângulo}\\
Um triângulo tem base 12 cm e altura 5 cm. Qual é sua área?
\begin{tcolorbox}[colback=explanationbg,colframe=titleblue,title=Dica:]
Área do triângulo $=$ $\frac{base \times altura}{2}$
\end{tcolorbox}

\item \textbf{Retângulo (Desafio)}\\
A largura de um retângulo é um terço do comprimento. Se a área é 108 m², encontre as dimensões.
\begin{tcolorbox}[colback=explanationbg,colframe=titleblue,title=Dica:]
Largura $= \frac{1}{3}x$, Comprimento $= x$ \\ Área $= x \times \frac{x}{3} = 108$
\end{tcolorbox}

\item \textbf{Losango}\\
As diagonais de um losango medem 6 cm e 8 cm. Qual é sua área?
\begin{tcolorbox}[colback=explanationbg,colframe=titleblue,title=Dica:]
Área do losango $=$ $\frac{d_1 \times d_2}{2}$
\end{tcolorbox}

\item \textbf{Figura Composta}\\
Uma figura formada por 2 quadrados iguais lado a lado tem área total 98 cm². Qual é o perímetro da figura?
\begin{tcolorbox}[colback=explanationbg,colframe=titleblue,title=Dica:]
Área de cada quadrado $= 49$ cm² \\ Lado $=$ $\sqrt{49}$ \\ Perímetro total $=$ 6 lados
\end{tcolorbox}

\item \textbf{Trapézio}\\
Um trapézio tem bases 5 cm e 7 cm e altura 4 cm. Qual é sua área?
\begin{tcolorbox}[colback=explanationbg,colframe=titleblue,title=Dica:]
Área do trapézio $=$ $\frac{(B + b) \times h}{2}$
\end{tcolorbox}

\item \textbf{Comparação de Áreas}\\
Um quadrado tem área igual a um retângulo com lados 9 cm e 4 cm. Qual é o lado do quadrado?
\begin{tcolorbox}[colback=explanationbg,colframe=titleblue,title=Dica:]
Área do retângulo $=$ 36 cm² \\ Lado do quadrado $=$ $\sqrt{36}$
\end{tcolorbox}

\item \textbf{Círculo}\\
Um círculo tem área 78,5 cm². Qual é seu raio? (Use $\pi = 3,14$)
\begin{tcolorbox}[colback=explanationbg,colframe=titleblue,title=Dica:]
Área $=$ $\pi r^2$ \\ $78,5 = 3,14 \times r^2$
\end{tcolorbox}

\item \textbf{Problema Contextualizado}\\
Um terreno retangular de 20 m × 30 m tem 50% de sua área ocupada por uma casa. Quantos m² de área livre restam?
\begin{tcolorbox}[colback=explanationbg,colframe=titleblue,title=Dica:]
Área total $=$ 600 m² \\ Área livre $=$ 50% de 600
\end{tcolorbox}

\section*{Desafio:}

\item \textbf{Sistema de Equações}\\
Dois retângulos têm a mesma área. O primeiro tem largura 5 cm e comprimento $x+3$ cm. O segundo tem largura $x$ cm e comprimento 7,5 cm. Determine $x$ e a área comum.
\begin{tcolorbox}[colback=explanationbg,colframe=titleblue,title=Dica:]
Iguale as áreas: \\ $5 \times (x + 3) = x \times 7,5$
\end{tcolorbox}

\end{enumerate}
\end{multicols}

\end{document}
