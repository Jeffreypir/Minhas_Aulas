\documentclass{article}
\usepackage[utf8]{inputenc}
\usepackage[portuguese]{babel}
\usepackage{amsmath}

\title{Respostas Comentadas de Aprofundamento em Matemática}
\author{}
\date{}

\begin{document}

\maketitle

\section*{Questão 1: Tempo de Download do Filme}

\textbf{Dados:}
\begin{itemize}
    \item Tamanho do arquivo: $3\,\text{GB} = 24.576\,\text{Mb}$ (convertido usando $3 \times 1024 \times 8$)
    \item Velocidade de download: $12\,\text{Mbps}$
\end{itemize}

\textbf{Fórmula:}
\[
\text{Tempo} = \frac{\text{Tamanho do arquivo}}{\text{Velocidade}} = \frac{24.576\,\text{Mb}}{12\,\text{Mbps}}
\]

\textbf{Cálculo:}
\[
\frac{24.576}{12} = 2.048\,\text{segundos} \approx 34,13\,\text{minutos}
\]

\textbf{Resposta:} O download levará aproximadamente \boxed{34,13\,\text{minutos}}.

\section*{Questão 2: Número de Fotos no Pendrive}

\textbf{Dados:}
\begin{itemize}
    \item Capacidade do pendrive: $64\,\text{GB} = 65.536\,\text{MB}$ (convertido usando $64 \times 1024$)
    \item Tamanho de cada foto: $8\,\text{MB}$
\end{itemize}

\textbf{Fórmula:}
\[
\text{Número de fotos} = \frac{\text{Capacidade total}}{\text{Tamanho por foto}} = \frac{65.536\,\text{MB}}{8\,\text{MB}}
\]

\textbf{Cálculo:}
\[
\frac{65.536}{8} = 8.192\,\text{fotos}
\]

\textbf{Resposta:} Cabem \boxed{8.192\,\text{fotos}} no pendrive.

\section*{Questão 3: Espaço Livre no SSD}

\textbf{Dados:}
\begin{itemize}
    \item Capacidade total: $1\,\text{TB} = 1.000\,\text{GB}$
    \item Espaço ocupado: $72,8\%$
\end{itemize}

\textbf{Fórmula:}
\[
\text{Espaço livre} = \text{Capacidade total} \times (1 - \text{Porcentagem ocupada})
\]

\textbf{Cálculo:}
\[
1.000\,\text{GB} \times (1 - 0,728) = 1.000 \times 0,272 = 272\,\text{GB}
\]

\textbf{Resposta:} Restam \boxed{272\,\text{GB}} de espaço livre.

\section*{Questão 4: Conversão de MB para Bytes}

\textbf{Dado:}
\[
1\,\text{MB} = 1.048.576\,\text{bytes}
\]

\textbf{Cálculo:}
\[
5\,\text{MB} = 5 \times 1.048.576 = 5.242.880\,\text{bytes}
\]

\textbf{Notação científica:}
\[
5.242.880 \approx 5,24 \times 10^6\,\text{bytes}
\]

\textbf{Resposta:} O valor é \boxed{5,24 \times 10^6\,\text{bytes}}.

\section*{Questão 5: Conversão de MB para GB}

\textbf{Dado:}
\[
1\,\text{GB} = 1.024\,\text{MB}
\]

\textbf{Cálculo:}
\[
3.072\,\text{MB} = \frac{3.072}{1.024} = 3\,\text{GB}
\]

\textbf{Resposta:} Equivale a \boxed{3\,\text{GB}}.

\section*{Questão 6: Fórmula de Soma no Excel}

\textbf{Resposta:} A fórmula correta é \boxed{\texttt{=SOMA(A1:A5)}}.

\section*{Questão 7: Célula Válida no Excel}

\textbf{Resposta:} A célula válida é \boxed{\texttt{D3}}.

\section*{Questão 8: Fórmula de Média no Excel}

\textbf{Resposta:} A fórmula correta é \boxed{\texttt{=MÉDIA(A1:A5)}}.

\section*{Questão 9: Função para Maior Valor}

\textbf{Resposta:} A função correta é \boxed{\texttt{=MÁXIMO()}}.

\section*{Questão 10: Notação Científica em Informática}

\textbf{Resposta:} Serve para \boxed{\text{representar números grandes ou pequenos de forma simplificada}}.

\end{document}
