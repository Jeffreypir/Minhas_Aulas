\documentclass{article}
\usepackage[utf8]{inputenc}
\usepackage[portuguese]{babel}
\usepackage{graphicx}
\usepackage{hyperref}

\title{Projeto de Pesquisa: Simulador de Poluição Marinha com Arduino}
\author{Jefferson \\ Maria Jamile}
\date{}

\begin{document}

\maketitle

\section{Objetivo Educativo}
Demonstrar como a poluição afeta o ecossistema marinho e como tecnologias podem ajudar no monitoramento ambiental.

\section{Materiais Necessários}
\begin{itemize}
    \item Arduino Uno
    \item Sensor de cor TCS34725
    \item LEDs RGB (pelo menos 3)
    \item Resistores (220Ω para os LEDs)
    \item Recipiente transparente com água (simulando o mar)
    \item Materiais coloridos para simular poluição (plástico, tinta, papel)
    \item Jumpers e protoboard
    \item Cabo USB para conexão com computador
\end{itemize}

\section{Passo a Passo para Montagem}

\subsection{1. Conexão do Sensor de Cor TCS34725}
\begin{enumerate}
    \item Conecte o sensor ao Arduino:
    \begin{itemize}
        \item VCC do sensor → 3.3V do Arduino
        \item GND do sensor → GND do Arduino
        \item SDA do sensor → A4 (SDA) do Arduino
        \item SCL do sensor → A5 (SCL) do Arduino
    \end{itemize}
\end{enumerate}

\subsection{2. Conexão dos LEDs RGB}
\begin{enumerate}
    \item Conecte cada LED RGB à protoboard com resistores de 220Ω nos cátodos (pernas R, G e B).
    \item Ligue os LEDs ao Arduino:
    \begin{itemize}
        \item Vermelho (R) → Pino 9
        \item Verde (G) → Pino 10
        \item Azul (B) → Pino 11
    \end{itemize}
    \item Conecte o GND dos LEDs ao GND do Arduino.
\end{enumerate}

\subsection{3. Montagem do Ambiente Marinho}
\begin{enumerate}
    \item Encha o recipiente com água limpa.
    \item Posicione o sensor de cor dentro do recipiente (protegido contra respingos).
    \item Prepare materiais coloridos (tinta, plástico, papel) para simular poluição.
\end{enumerate}

\subsection{4. Programação no Arduino IDE}
\begin{enumerate}
    \item Instale a biblioteca \texttt{Adafruit\_TCS34725} via Gerenciador de Bibliotecas.
    \item Carregue o seguinte código no Arduino:
\begin{verbatim}
#include <Wire.h>
#include "Adafruit_TCS34725.h"

Adafruit_TCS34725 tcs = Adafruit_TCS34725(TCS34725_INTEGRATIONTIME_50MS, 
                                          TCS34725_GAIN_4X);

const int redPin = 9;
const int greenPin = 10;
const int bluePin = 11;

void setup() {
  Serial.begin(9600);
  pinMode(redPin, OUTPUT);
  pinMode(greenPin, OUTPUT);
  pinMode(bluePin, OUTPUT);
  
  if (!tcs.begin()) {
    Serial.println("Sensor não encontrado!");
    while (1);
  }
}

void loop() {
  uint16_t r, g, b, c;
  tcs.getRawData(&r, &g, &b, &c);
  
  Serial.print("R: "); Serial.print(r);
  Serial.print(" G: "); Serial.print(g);
  Serial.print(" B: "); Serial.print(b);
  Serial.print(" C: "); Serial.println(c);
  
  // Mapeia cores para LEDs (exemplo: vermelho = poluição alta)
  if (r > g && r > b && r > 1000) {
    analogWrite(redPin, 255);    // Vermelho (poluição alta)
    analogWrite(greenPin, 0);
    analogWrite(bluePin, 0);
    Serial.println("ALERTA: Poluição detectada!");
  } else {
    analogWrite(redPin, 0);
    analogWrite(greenPin, 255);  // Verde (água limpa)
    analogWrite(bluePin, 0);
  }
  delay(500);
}
\end{verbatim}
\end{enumerate}

\subsection{5. Teste e Calibração}
\begin{enumerate}
    \item Abra o Monitor Serial (Ctrl+Shift+M) para ver os valores RGB.
    \item Adicione materiais coloridos à água e observe as mudanças:
    \begin{itemize}
        \item Água limpa → LED verde aceso.
        \item Poluição detectada → LED vermelho aceso.
    \end{itemize}
    \item Ajuste os limiares (valores de \texttt{r, g, b}) no código conforme necessário.
\end{enumerate}

\section{Conclusão}
Este projeto demonstra como sensores e microcontroladores podem ser usados para monitorar a qualidade da água. O projeto pode ser expandido   adicionando:
\begin{itemize}
    \item Um display LCD para mostrar os níveis de poluição.
    \item Um módulo Wi-Fi (ESP8266) para envio de dados em tempo real.
    \item Diferentes cores de LED para níveis intermediários de poluição.
\end{itemize}

\end{document}
