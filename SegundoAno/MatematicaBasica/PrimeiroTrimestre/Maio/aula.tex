\documentclass[11pt]{article}
\usepackage[utf8]{inputenc}
\usepackage[T1]{fontenc}
\usepackage{newtxtext,newtxmath}
\usepackage{amsmath,multicol,geometry,enumitem,xcolor,titlesec,tikz}
\usetikzlibrary{arrows.meta}
\geometry{a4paper, left=1cm, right=1cm, top=0.5cm, bottom=1.2cm}
\setlength{\columnseprule}{0.4pt}
\setlength{\baselineskip}{1.0\baselineskip}

% Formatação de títulos
\titleformat{\section}{\normalfont\Large\bfseries\color{blue}}{\thesection}{1em}{}
\titleformat{\subsection}{\normalfont\large\bfseries\color{red}}{\thesubsection}{1em}{}
\titleformat{\subsubsection}{\normalfont\normalsize\bfseries}{\thesubsubsection}{1em}{}

\title{\textcolor{blue}{Sistemas de Equações - 30 Exercícios}}
\author{Professor: Jefferson}
\date{}

\begin{document}

\maketitle
\vspace{-1cm}

\begin{center}
\large{Nome: \underline{\hspace{8cm}} \quad Série-Turma: \underline{\hspace{3cm}}}
\end{center}

\begin{multicols}{2}

\section*{Parte 1: Sistemas 2x2 (Métodos Básicos)}

\subsection*{Método da Substituição}
1. Resolva:
\[
\begin{cases}
x + y = 10 \\
2x - y = 5
\end{cases}
\]

2. Resolva:
\[
\begin{cases}
3x - y = 7 \\
x + 2y = 4
\end{cases}
\]

3. Resolva:
\[
\begin{cases}
2x + 5y = 16 \\
x = y + 1
\end{cases}
\]

\subsection*{Método da Adição}
4. Resolva:
\[
\begin{cases}
x + y = 8 \\
x - y = 2
\end{cases}
\]

5. Resolva:
\[
\begin{cases}
3x + 2y = 11 \\
4x - 2y = 10
\end{cases}
\]

6. Resolva:
\[
\begin{cases}
5x - 3y = 12 \\
2x + 3y = 24
\end{cases}
\]

\subsection*{Frações e Decimais}
7. Resolva:
\[
\begin{cases}
0.5x + y = 6 \\
x - 0.2y = 5
\end{cases}
\]

8. Resolva:
\[
\begin{cases}
\frac{x}{2} + \frac{y}{3} = 4 \\
x - y = 3
\end{cases}
\]

9. Resolva:
\[
\begin{cases}
\frac{2x}{5} - \frac{y}{2} = 1 \\
3x + 4y = 26
\end{cases}
\]

\section*{Parte 2: Aplicações Práticas}

\subsection*{Problemas Numéricos}
10. A soma de dois números é 15 e sua diferença é 3. Quais são esses números?

11. Um número é o triplo de outro. A soma deles é 48. Determine-os.

12. Dois números estão na razão 2:5. Se a soma é 28, encontre-os.

\subsection*{Idades}
13. Ana é 5 anos mais velha que Bia. Daqui a 3 anos, a soma das idades será 35. Qual a idade atual de cada uma?

14. Carlos tem o dobro da idade de Davi. Há 10 anos, ele tinha o triplo. Qual a idade atual?

15. A soma das idades de pai e filho é 45. Daqui a 15 anos, o pai terá o dobro da idade do filho. Quais as idades hoje?

\subsection*{Finanças}
16. Duas camisas e um boné custam R\$ 110. Uma camisa e dois bonés custam R\$ 100. Qual o preço de cada item?

17. 3 kg de maçã e 2 kg de banana custam R\$ 12. 1 kg de maçã e 4 kg de banana custam R\$ 14. Qual o preço por kg?

18. Um ingresso de adulto custa o dobro de um infantil. 2 adultos e 3 crianças pagam R\$ 140. Quanto custa cada ingresso?

\section*{Parte 3: Sistemas 3x3 e Desafios}

\subsection*{Sistemas com Três Variáveis}
19. Resolva:
\[
\begin{cases}
x + y + z = 6 \\
2x - y + z = 3 \\
x + 2y - z = 2
\end{cases}
\]

20. Resolva:
\[
\begin{cases}
2x - y + 3z = 9 \\
x + 3y - z = 4 \\
3x + 2y + z = 10
\end{cases}
\]

21. Resolva:
\[
\begin{cases}
x + 2y - z = 5 \\
3x - y + 2z = 8 \\
2x + 3y - 4z = 1
\end{cases}
\]

\subsection*{Problemas Gráficos}
22. Represente graficamente e resolva:
\[
\begin{cases}
y = 2x + 1 \\
y = -x + 4
\end{cases}
\]

23. Quantas soluções tem o sistema:
\[
\begin{cases}
y = 3x - 2 \\
6x - 2y = 4
\end{cases}
\]

24. Para qual valor de \( k \) o sistema é indeterminado?
\[
\begin{cases}
2x + 5y = 7 \\
4x + ky = 14
\end{cases}
\]

\subsection*{Desafios}
25. Resolva:
\[
\begin{cases}
\frac{1}{x} + \frac{1}{y} = 5 \\
\frac{2}{x} - \frac{3}{y} = -1
\end{cases}
\]
(Dica: Faça \( a = \frac{1}{x}, b = \frac{1}{y} \))

26. Encontre \( a \) e \( b \) para que o sistema tenha infinitas soluções:
\[
\begin{cases}
2x + ay = 6 \\
bx + 3y = 9
\end{cases}
\]

27. (ITA) Para que valores de \( m \) o sistema é impossível?
\[
\begin{cases}
(m+1)x + 2y = 3 \\
mx + y = 1
\end{cases}
\]

\section*{Parte 4: Problemas Contextualizados}

28. \textbf{(Movimento Uniforme)} Dois trens partem de cidades distantes 300 km. O trem A viaja a 60 km/h e o B a 40 km/h. Em quanto tempo se encontram?

29. \textbf{(Química)} Uma solução A tem 20\% de sal, e a B tem 50\%. Quantos litros de cada devem ser misturados para obter 10 L a 32\%?

30. \textbf{(Geometria)} Um retângulo tem perímetro 40 cm. Se o comprimento é 4 cm maior que a largura, calcule sua área.

\section*{Gabarito Parcial}

\begin{enumerate}[label=\arabic*., leftmargin=*]
\item \( (5, 5) \)  
\item \( (2, -1) \)  
\item \( (3, 2) \)  
\item \( (5, 3) \)  
\item \( (3, 1) \)  
\item \( (4, \frac{8}{3}) \)  
\item \( (5, 3.5) \)  
\item \( (6, 3) \)  
\item \( (5, 2.75) \)  
\item 9 e 6  
\item 12 e 36  
\item 8 e 20  
\item Ana: 12, Bia: 7  
\item Carlos: 40, Davi: 20  
\item Pai: 35, Filho: 10  
\item Camisa: R\$ 40, Boné: R\$ 30  
\item Maçã: R\$ 2/kg, Banana: R\$ 3/kg  
\item Adulto: R\$ 40, Infantil: R\$ 20  
\item \( (1, 2, 3) \)  
\item \( (2, 1, 3) \)  
\end{enumerate}

\end{multicols}
\end{document}
