
\documentclass{article}
\usepackage[utf8]{inputenc}
\usepackage{geometry}
\geometry{a4paper, margin=1.5cm}

\title{Planejamento Semanal: Razão, Proporção e Regra de Três - Semana 1}
\author{Professor(a): Jefferson}
\date{}

\begin{document}

\maketitle

\section*{Objetivo da Semana}
Introduzir os conceitos de razão e proporção, e iniciar o estudo da regra de três simples, aplicando-os em situações práticas do cotidiano.

\subsection*{Aula 1: Introdução à Razão e Proporção}
\begin{itemize}
    \item \textbf{Conteúdo:}
    \begin{itemize}
        \item Definição de razão: comparação entre duas grandezas, expressa como \( \frac{a}{b} \) ou \( a : b \).
        \item Definição de proporção: igualdade entre duas razões, expressa como \( \frac{a}{b} = \frac{c}{d} \).
        \item Propriedades fundamentais das proporções:
            \begin{itemize}
                \item Propriedade fundamental: \( a \cdot d = b \cdot c \).
                \item Propriedade da soma: \( \frac{a + b}{b} = \frac{c + d}{d} \).
            \end{itemize}
        \item Exemplos práticos:
            \begin{itemize}
                \item Escalas de mapas: se 1 cm no mapa representa 10 km na realidade, a razão é \( 1:1.000.000 \).
                \item Receitas culinárias: para fazer 12 pães, usa-se 3 xícaras de farinha. Qual a razão farinha/pães? \( \frac{3}{12} = \frac{1}{4} \).
            \end{itemize}
    \end{itemize}
    \item \textbf{Atividades:}
    \begin{itemize}
        \item Resolução de exercícios básicos:
            \begin{enumerate}
                \item Calcule a razão entre 20 e 4. \\
                      \textbf{Resposta:} \( \frac{20}{4} = 5 \).
                \item Verifique se \( \frac{2}{5} = \frac{8}{20} \) é uma proporção. \\
                      \textbf{Resposta:} Sim, pois \( 2 \cdot 20 = 5 \cdot 8 \) (40 = 40).
                \item Se 3 litros de suco custam R\$ 12, qual o custo de 7 litros? \\
                      \textbf{Resposta:} \( \frac{3}{12} = \frac{7}{x} \Rightarrow x = 28 \) (R\$ 28).
                \item Em uma escola, há 20 professores para 400 alunos. Qual a razão professor/aluno? \\
                      \textbf{Resposta:} \( \frac{20}{400} = \frac{1}{20} \).
                \item Se \( \frac{a}{b} = \frac{3}{4} \) e \( b = 12 \), qual o valor de \( a \)? \\
                      \textbf{Resposta:} \( a = 9 \).
            \end{enumerate}
        \item Discussão em grupo: trazer exemplos de razão e proporção no dia a dia.
    \end{itemize}
    \item \textbf{Recursos:} Quadro, projetor, exemplos impressos.
\end{itemize}

\subsection*{Aula 2: Aplicações de Proporção e Introdução à Regra de Três}
\begin{itemize}
    \item \textbf{Conteúdo:}
    \begin{itemize}
        \item Revisão rápida de razão e proporção.
        \item Introdução à regra de três simples:
            \begin{itemize}
                \item Diretamente proporcional: se \( a \) aumenta, \( b \) aumenta na mesma proporção.
                \item Inversamente proporcional: se \( a \) aumenta, \( b \) diminui na mesma proporção.
            \end{itemize}
        \item Exemplos práticos:
            \begin{itemize}
                \item Diretamente proporcional: se 2 pães custam R\$ 4, quanto custam 5 pães? \( \frac{2}{4} = \frac{5}{x} \Rightarrow x = 10 \) (R\$ 10).
                \item Inversamente proporcional: se 4 trabalhadores constroem um muro em 6 dias, quanto tempo levam 8 trabalhadores? \( \frac{4}{8} = \frac{x}{6} \Rightarrow x = 3 \) (3 dias).
            \end{itemize}
    \end{itemize}
    \item \textbf{Atividades:}
    \begin{itemize}
        \item Resolução de problemas práticos:
            \begin{enumerate}
                \item Se um carro percorre 240 km com 20 litros de gasolina, quantos litros precisa para percorrer 600 km? \\
                      \textbf{Resposta:} \( \frac{240}{20} = \frac{600}{x} \Rightarrow x = 50 \) litros.
                \item Se 3 torneiras enchem um tanque em 4 horas, quanto tempo levam 6 torneiras? \\
                      \textbf{Resposta:} \( \frac{3}{6} = \frac{x}{4} \Rightarrow x = 2 \) horas.
                \item Se 5 operários constroem um muro em 10 dias, quantos dias levam 10 operários? \\
                      \textbf{Resposta:} \( \frac{5}{10} = \frac{x}{10} \Rightarrow x = 5 \) dias.
                \item Se 8 metros de tecido custam R\$ 24, quanto custam 12 metros? \\
                      \textbf{Resposta:} \( \frac{8}{24} = \frac{12}{x} \Rightarrow x = 36 \) (R\$ 36).
                \item Se 6 máquinas produzem 120 peças em 2 horas, quantas peças produzem 9 máquinas em 3 horas? \\
                      \textbf{Resposta:} \( \frac{6}{9} = \frac{120}{x} \Rightarrow x = 180 \) peças.
            \end{enumerate}
        \item Atividade em duplas: criar e resolver problemas envolvendo regra de três.
    \end{itemize}
    \item \textbf{Recursos:} Lista de exercícios, calculadoras.
\end{itemize}

\end{document}


