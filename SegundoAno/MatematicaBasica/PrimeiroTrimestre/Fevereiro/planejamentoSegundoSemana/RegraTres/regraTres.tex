\documentclass[a4paper,12pt]{article}
\usepackage[brazil]{babel}
\usepackage[utf8]{inputenc}
\usepackage{amsmath}

\title{Regra de Três Simples e Composta}
\author{}
\date{}

\begin{document}

\maketitle

\section{Aula 1: Introdução à Regra de Três Simples}

A regra de três simples é uma ferramenta matemática utilizada para resolver problemas que envolvem grandezas diretamente ou inversamente proporcionais.

\subsection*{Exemplo}
Se 4 lápis custam R\$ 12,00, quanto custarão 10 lápis?

Montamos a proporção:
\[ 4 \text{ lápis } \rightarrow 12 \text{ reais} \]
\[ 10 \text{ lápis } \rightarrow x \]

Sendo diretamente proporcional:
\[ 4x = 10 \times 12 \]
\[ x = \frac{10 \times 12}{4} = 30 \]

Logo, 10 lápis custarão R\$ 30,00.

\subsection*{Atividades}

1. Se 5 metros de tecido custam R\$ 75,00, quanto custarão 8 metros?  \newline
2. Um carro percorre 300 km com 20 litros de combustível. Quantos litros gastará para percorrer 450 km? \newline
3. Se 3 trabalhadores constroem um muro em 10 dias, quantos dias levarão 5 trabalhadores para construí-lo? \newline
4. Uma máquina produz 200 peças em 5 horas. Quantas peças produzirá em 8 horas? \newline
5. Se 7 maçãs custam R\$ 14,00, qual o preço de 12 maçãs? \newline
6. Uma torneira enche um tanque em 6 horas. Quanto tempo levarão 2 torneiras iguais para enchê-lo? \newline
7. Se 4 caixas pesam 20 kg, qual será o peso de 7 caixas? \newline
8. Um ciclista percorre 40 km em 2 horas. Quantos km percorrerá em 5 horas? \newline

\textbf{Gabarito:} 1) 120,00; 2) 30 L; 3) 6 dias; 4) 320 peças; 5) 24,00; 6) 3 h; 7) 35 kg; 8) 100 km.

\section{Aula 2: Regra de Três Simples Inversa}

Quando uma grandeza aumenta e a outra diminui, dizemos que são inversamente proporcionais.

\subsection*{Exemplo}
Se 4 pintores pintam uma parede em 12 dias, quantos dias levarão 6 pintores?

\[ 4 \times 12 = 6 \times x \]
\[ x = \frac{4 \times 12}{6} = 8 \]

Logo, 6 pintores levarão 8 dias.

\subsection*{Atividades}

1. Se 3 trabalhadores fazem um serviço em 15 dias, quantos dias levarão 5 trabalhadores? \newline
2. Um carro a 80 km/h faz um percurso em 6 horas. Em quanto tempo fará a 100 km/h? \newline
3. Se 5 torneiras enchem um tanque em 4 horas, quanto tempo levarão 8 torneiras? \newline
4. Um grupo de 10 alunos divide igualmente R\$ 500,00. Quanto receberia cada um se fossem 5 alunos? \newline
5. Uma máquina enche 240 garrafas em 3 horas. Quanto tempo levaria para encher as mesmas garrafas com o dobro da velocidade? \newline
6. Se 12 operários fazem uma obra em 18 dias, quantos dias levariam 8 operários? \newline
7. Um avião viaja 900 km em 2 horas a 450 km/h. Em quanto tempo faria o mesmo percurso a 600 km/h? \newline
8. Uma equipe de 4 pessoas termina um serviço em 24 dias. Quantos dias levariam 6 pessoas? \newline

\textbf{Gabarito:} 1) 9 dias; 2) 4,8 h; 3) 2,5 h; 4) R\$ 100,00; 5) 1,5 h; 6) 27 dias; 7) 1,5 h; 8) 16 dias.

\section{Aula 3: Regra de Três Composta}

Quando há mais de duas grandezas envolvidas, utilizamos a regra de três composta.

\subsection*{Exemplo}
Se 3 máquinas produzem 900 peças em 6 horas, quantas peças produzirão 5 máquinas em 8 horas?

\[ \frac{3}{5} = \frac{900}{x} \times \frac{6}{8} \]
\[ x = \frac{900 \times 5 \times 8}{3 \times 6} = 2000 \]

Logo, 5 máquinas produzirão 2000 peças.

\subsection*{Atividades}

1. Se 4 caminhões transportam 2000 kg em 5 viagens, quantos kg transportarão 6 caminhões em 8 viagens? \newline
2. Se 10 pessoas fazem 30 camisetas em 5 dias, quantas camisetas fazem 15 pessoas em 7 dias? \newline
3. Se 5 operários constroem 200 m² em 8 dias, quantos metros construirão 8 operários em 10 dias? \newline
4. Se 6 torneiras enchem um tanque em 3 horas, em quanto tempo 9 torneiras encherão o mesmo tanque? \newline
5. Se 3 carros consomem 90 litros de combustível em 6 horas, quantos litros consumirão 5 carros em 10 horas? \newline
6. Se 7 máquinas produzem 2800 peças em 4 horas, quantas peças produzirão 9 máquinas em 6 horas? \newline
7. Se 12 trabalhadores constroem uma parede em 10 dias, quantos trabalhadores fariam a mesma parede em 6 dias? \newline
8. Se 8 estudantes fazem um trabalho em 12 horas, quantas horas levarão 10 estudantes? \newline

\textbf{Gabarito:} 1) 4800 kg; 2) 63 camisetas; 3) 500 m²; 4) 2 h; 5) 250 L; 6) 3780 peças; 7) 20 trabalhadores; 8) 9,6 h.

\section{Aula 4: Aplicações e Problemas Práticos}

Nesta aula, resolveremos problemas aplicados do dia a dia, utilizando as regras aprendidas.

\subsection*{Atividades}

1. Uma padaria vende 240 pães em 3 horas. Quantos pães venderá em 5 horas, mantendo a mesma taxa de vendas? \newline
2. Um caminhão transporta 1500 kg de carga em 4 viagens. Quantos kg transportará em 7 viagens? \newline
3. Se 5 máquinas produzem 400 peças em 6 horas, quantas peças produzirão 8 máquinas no mesmo tempo? \newline
4. Um grupo de 10 pintores pinta uma parede em 12 dias. Em quantos dias um grupo de 15 pintores faria o mesmo trabalho? \newline
5. Se um carro consome 8 litros de combustível para percorrer 120 km, quantos litros consumirá para percorrer 300 km? \newline
6. Se 4 torneiras enchem um tanque em 9 horas, quanto tempo levarão 6 torneiras para enchê-lo? \newline
7. Um ciclista percorre 50 km em 2 horas. Quantos km percorrerá em 5 horas mantendo a mesma velocidade? \newline
8. Uma confeitaria usa 3 kg de farinha para fazer 60 bolos. Quantos kg de farinha serão necessários para fazer 150 bolos? \newline
 \newline
\textbf{Gabarito:} \newline
1) 400 pães; 2) 2625 kg; 3) 640 peças; 4) 8 dias; 5) 20 L; 6) 6 h; 7) 125 km; 8) 7,5 kg. \newline

\end{document}

