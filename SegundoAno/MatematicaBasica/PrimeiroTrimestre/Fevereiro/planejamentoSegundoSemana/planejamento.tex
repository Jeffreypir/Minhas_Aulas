
\documentclass{article}
\usepackage[utf8]{inputenc}
\usepackage{geometry}
\geometry{a4paper, margin=1.5cm}

\title{Planejamento Semanal:\\ Razão, Proporção e Regra de Três}
\author{Professor(a): Jefferson }
\date{}

\begin{document}

\maketitle

\section*{Objetivo da Semana}
Aprofundar o estudo da regra de três, incluindo a regra de três composta, e aplicar os conceitos em problemas interdisciplinares e do cotidiano.

\subsection*{Aula 3: Regra de Três Composta}
\begin{itemize}
    \item \textbf{Conteúdo:}
    \begin{itemize}
        \item Definição de regra de três composta: envolve mais de duas grandezas.
        \item Passos para resolver:
            \begin{itemize}
                \item Identificar as grandezas e suas relações (diretamente ou inversamente proporcionais).
                \item Montar a proporção e resolver.
            \end{itemize}
        \item Exemplos práticos:
            \begin{itemize}
                \item Se 5 máquinas produzem 100 peças em 10 dias, quantas peças 8 máquinas produzem em 15 dias? \\
                      \textbf{Resposta:} \( \frac{5}{8} \cdot \frac{10}{15} = \frac{100}{x} \Rightarrow x = 240 \) peças.
                \item Se 4 pedreiros constroem 2 casas em 30 dias, quantos pedreiros são necessários para construir 5 casas em 20 dias? \\
                      \textbf{Resposta:} \( \frac{4}{x} \cdot \frac{2}{5} \cdot \frac{30}{20} \Rightarrow x = 15 \) pedreiros.
            \end{itemize}
    \end{itemize}
    \item \textbf{Atividades:}
    \begin{itemize}
        \item Resolução de exercícios:
            \begin{enumerate}
                \item Se 6 operários constroem um muro em 8 dias, quantos dias levam 9 operários? \\
                      \textbf{Resposta:} \( \frac{6}{9} = \frac{x}{8} \Rightarrow x = \frac{48}{9} \approx 5,33 \) dias.
                \item Se 3 torneiras enchem um tanque em 4 horas, quanto tempo levam 5 torneiras? \\
                      \textbf{Resposta:} \( \frac{3}{5} = \frac{x}{4} \Rightarrow x = 2,4 \) horas.
                \item Se 10 máquinas produzem 200 peças em 5 dias, quantas peças produzem 15 máquinas em 6 dias? \\
                      \textbf{Resposta:} \( \frac{10}{15} \cdot \frac{5}{6} = \frac{200}{x} \Rightarrow x = 360 \) peças.
                \item Se 8 trabalhadores constroem 4 casas em 10 dias, quantos dias levam 12 trabalhadores para construir 6 casas? \\
                      \textbf{Resposta:} \( \frac{8}{12} \cdot \frac{4}{6} = \frac{10}{x} \Rightarrow x = 15 \) dias.
                \item Se 5 bombas enchem um tanque em 3 horas, quanto tempo levam 7 bombas? \\
                      \textbf{Resposta:} \( \frac{5}{7} = \frac{x}{3} \Rightarrow x = \frac{15}{7} \approx 2,14 \) horas.
            \end{enumerate}
        \item Discussão de casos reais: produção industrial, taxa de trabalho, etc.
    \end{itemize}
    \item \textbf{Recursos:} Quadro, exemplos práticos, lista de exercícios.
\end{itemize}

\subsection*{Aula 4: Revisão e Aplicações Práticas}
\begin{itemize}
    \item \textbf{Conteúdo:}
    \begin{itemize}
        \item Revisão geral dos conceitos de razão, proporção e regra de três.
        \item Aplicação dos conceitos em problemas interdisciplinares:
            \begin{itemize}
                \item Física: cálculo de velocidade, tempo e distância.
                \item Química: proporções em reações químicas.
                \item Finanças: cálculo de juros e descontos.
            \end{itemize}
    \end{itemize}
    \item \textbf{Atividades:}
    \begin{itemize}
        \item Resolução de problemas contextualizados:
            \begin{enumerate}
                \item Um carro viaja a 60 km/h e leva 3 horas para percorrer uma distância. Quanto tempo levaria a 80 km/h? \\
                      \textbf{Resposta:} \( \frac{60}{80} = \frac{x}{3} \Rightarrow x = 2,25 \) horas.
                \item Uma reação química requer 2 mols de hidrogênio para 1 mol de oxigênio. Quantos mols de oxigênio são necessários para 6 mols de hidrogênio? \\
                      \textbf{Resposta:} \( \frac{2}{1} = \frac{6}{x} \Rightarrow x = 3 \) mols.
                \item Se um investimento de R\$ 1.000 rende R\$ 100 em 1 ano, quanto renderá R\$ 2.500 em 2 anos? \\
                      \textbf{Resposta:} \( \frac{1000}{100} = \frac{2500}{x} \Rightarrow x = 250 \) (R\$ 250).
                \item Se 5 operários constroem 2 casas em 10 dias, quantos dias levam 8 operários para construir 4 casas? \\
                      \textbf{Resposta:} \( \frac{5}{8} \cdot \frac{2}{4} = \frac{10}{x} \Rightarrow x = 16 \) dias.
                \item Se 3 torneiras enchem um tanque em 4 horas, quanto tempo levam 6 torneiras? \\
                      \textbf{Resposta:} \( \frac{3}{6} = \frac{x}{4} \Rightarrow x = 2 \) horas.
            \end{enumerate}
        \item Atividade em grupo: criar e resolver problemas envolvendo os conceitos estudados.
    \end{itemize}
    \item \textbf{Recursos:} Lista de problemas, calculadoras, materiais de apoio.
\end{itemize}

\section*{Avaliação}
\begin{itemize}
    \item Participação nas atividades em sala de aula.
    \item Resolução de exercícios práticos.
    \item Prova escrita ao final da segunda semana, com questões envolvendo razão, proporção e regra de três.
\end{itemize}

\end{document}

