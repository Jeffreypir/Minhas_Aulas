\documentclass[11pt]{article}
\usepackage[utf8]{inputenc}
\usepackage[T1]{fontenc}
\usepackage{newtxtext,newtxmath} % Fonte Times New Melhor renderiza
\usepackage{amsmath}
\usepackage{multicol}
\usepackage{geometry}
\usepackage{tikz}
\usetikzlibrary{arrows.meta}
\usepackage{enumitem} % Para listas personalizadas
\usepackage{xcolor} % Para usar cores
\usepackage{titlesec} % Para personalizar títulos

% Ajusta o espaçamento antes e depois das seções
\titleformat{\section}[block]{\normalfont\Large\bfseries}{\thesection}{1em}{}
\titlespacing*{\section}{8pt}{8pt}{8pt}

\titleformat{\subsection}[block]{\normalfont\large\bfseries}{\thesubsection}{1em}{}
\titlespacing*{\subsection}{6pt}{6pt}{6pt}

\titleformat{\subsubsection}[block]{\normalfont\normalsize\bfseries}{\thesubsubsection}{1em}{}
\titlespacing*{\subsubsection}{6pt}{6pt}{6pt}

\geometry{a4paper, left=1cm, right=1cm, top=0.5cm, bottom=1.2cm}

\setlength{\columnseprule}{0.4pt}  % Linha dividindo as colunas 
\setlength{\baselineskip}{1.0\baselineskip} % Espaçamento simples

% Definir a cor das seções como azul
\titleformat{\section}
  {\normalfont\Large\bfseries\color{blue}} % Formato do título
  {\thesection} % Número da seção
  {1em} % Espaço entre número e título
  {} % Código antes do título

\renewcommand{\thesubsection}{\textcolor{red}{\arabic{section}.\arabic{subsection}}}
\titleformat{\subsection}{\color{red}\normalfont\bfseries}{\thesubsection}{1em}{}
\title{\textcolor{blue}{Matemática Básica: Juros Simples e Compostos}}
\author{Professor: Jefferson}
\date{}

\begin{document}

\maketitle
\vspace{-1cm}  % Ajuste o valor conforme necessário

\begin{center}
\large{Nome: \underline{\hspace{8cm}} \quad Série-Turma: \underline{\hspace{3cm}}}
\end{center}

\begin{multicols}{2}

\section*{O que são Juros?}

Juros são valores pagos ou recebidos em operações financeiras, como empréstimos, investimentos ou financiamentos. Eles podem ser classificados em dois tipos principais: juros simples e juros compostos.

\section*{Juros Simples}

No regime de juros simples, os juros são calculados sempre sobre o valor inicial (capital), sem considerar os juros acumulados ao longo do tempo. A fórmula para calcular os juros simples é:

\[
J = C \cdot i \cdot t
\]

Onde:
\begin{itemize}
    \item  \( J \): Juros
    \item  \( C \): Capital inicial
    \item  \( i \): Taxa de juros (em decimal)
    \item  \( t \): Tempo (em períodos)
\end{itemize}

\subsection*{Exemplo 1: Cálculo de Juros Simples}

Suponha que você emprestou R\$ 1.000,00 a uma taxa de juros simples de 5\% ao mês, por 3 meses. Qual será o valor dos juros?

1. Converta a taxa de juros para decimal:
\[
i = 5\% = 0,05
\]

2. Aplique a fórmula dos juros simples:
\[
J = 1000 \cdot 0,05 \cdot 3 = 150
\]

Portanto, os juros serão de R\$ 150,00.

\subsection*{Montante no Juros Simples}

O montante \( M \) é a soma do capital inicial com os juros:
\[
M = C + J = C \cdot (1 + i \cdot t)
\]

No exemplo anterior:
\[
M = 1000 + 150 = 1150
\]

\section*{Juros Compostos}

No regime de juros compostos, os juros são calculados sobre o capital inicial e também sobre os juros acumulados em períodos anteriores. A fórmula para calcular o montante no regime de juros compostos é:

\[
M = C \cdot (1 + i)^t
\]

Onde:

\begin{itemize}
    \item \( M \): Montante
    \item \( C \): Capital inicial
    \item \( i \): Taxa de juros (em decimal)
    \item \( t \): Tempo (em períodos)
\end{itemize}
\subsection*{Exemplo 2: Cálculo de Juros Compostos}

Suponha que você investiu R\$ 1.000,00 a uma taxa de juros compostos de 5\% ao mês, por 3 meses. Qual será o montante ao final do período?

1. Converta a taxa de juros para decimal:
\[
i = 5\% = 0,05
\]

2. Aplique a fórmula dos juros compostos:
\[
M = 1000 \cdot (1 + 0,05)^3 = 1000 \cdot 1,157625 = 1157,63
\]

Portanto, o montante será de R\$ 1.157,63.

\section*{Comparação entre Juros Simples e Compostos}

A principal diferença entre os dois regimes é que, no juros simples, os juros são constantes em cada período, enquanto no juros compostos, os juros aumentam ao longo do tempo, pois são calculados sobre o montante acumulado.

\subsection*{Exemplo 3: Comparação}

Considere um capital de R\$ 1.000,00 aplicado a uma taxa de 10\% ao ano, por 3 anos. \\

\textbf{Juros Simples:}
\[
J = 1000 \cdot 0,10 \cdot 3 = 300
\]
\[
M = 1000 + 300 = 1300
\]

\textbf{Juros Compostos:}

\[
M = 1000 \cdot (1 + 0,10)^3 = 1000 \cdot 1,331 = 1331
\]

Note que, no juros compostos, o montante é maior devido ao efeito dos juros sobre juros.

\section*{Atividades de Fixação}

\subsection*{Questão 1: Juros Simples}

Um capital de R\$ 2.000,00 foi aplicado a uma taxa de juros simples de 8\% ao ano, por 5 anos. Qual será o montante ao final do período?

\subsection*{Questão 2: Juros Compostos}

Um investimento de R\$ 5.000,00 foi aplicado a uma taxa de juros compostos de 6\% ao semestre, por 4 semestres. Qual será o montante ao final do período?

\subsection*{Questão 3: Comparação}

Um capital de R\$ 3.000,00 foi aplicado a uma taxa de 12\% ao ano. Calcule o montante após 2 anos, considerando:
a) Juros simples  
b) Juros compostos  

\subsection*{Questão 4: Aplicação Prática}

Uma pessoa emprestou R\$ 10.000,00 a uma taxa de juros compostos de 2\% ao mês, por 6 meses. Qual será o valor total a ser pago ao final do período?

\subsection*{Questão 5: Tempo de Investimento}

Quanto tempo é necessário para que um capital de R\$ 1.000,00, aplicado a uma taxa de juros compostos de 5\% ao mês, dobre de valor?

\subsection*{Questão 6: Taxa de Juros}

Um investimento de R\$ 2.000,00 rendeu R\$ 400,00 em juros compostos após 2 anos. Qual foi a taxa de juros anual aplicada?

\subsection*{Questão 7: Capital Inicial}

Um montante de R\$ 6.000,00 foi obtido após aplicar um capital a uma taxa de juros compostos de 10\% ao ano, por 3 anos. Qual foi o capital inicial?

\subsection*{Questão 8: Decisão Financeira}

Uma pessoa tem duas opções de investimento:
\begin{itemize}
    \item Opção A: Juros simples de 8\% ao ano.
    \item Opção B: Juros compostos de 7\% ao ano.
\end{itemize}

Qual opção é mais vantajosa para um investimento de R\$ 4.000,00 por 5 anos?

\subsection*{Questão 9: Aplicação de Juros Compostos}

Um banco oferece um investimento com juros compostos de 1\% ao mês. Se você investir R\$ 500,00 hoje, quanto terá após 1 ano?

\subsection*{Questão 10: Juros Simples vs. Compostos}

Um capital de R\$ 1.500,00 foi aplicado a uma taxa de 9\% ao ano. Compare os montantes obtidos após 4 anos, considerando juros simples e juros compostos.

\subsection*{Questão 11: Tempo para Triplicar o Capital}
Um capital de R\$ 2.000,00 foi aplicado a uma taxa de juros compostos de 6\% ao ano. Quanto tempo será necessário para que o capital triplique de valor?

\textbf{Dica:} Use a fórmula dos juros compostos e resolva para tt.

\subsection*{Questão 12: Taxa de Juros Mensal}
Um investimento de R\$ 3.000,00 rendeu R\$ 900,00 em juros compostos após 6 meses. Qual foi a taxa de juros mensal aplicada?

\textbf{Dica:} Use a fórmula dos juros compostos e resolva para ii.

\end{multicols}

\end{document}
