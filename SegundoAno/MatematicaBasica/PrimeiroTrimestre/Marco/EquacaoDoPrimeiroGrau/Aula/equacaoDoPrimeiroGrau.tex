\documentclass[11pt]{article}
\usepackage[utf8]{inputenc}
\usepackage[T1]{fontenc}
\usepackage{newtxtext,newtxmath} % Fonte Times New Melhor renderiza
\usepackage{amsmath}
\usepackage{multicol}
\usepackage{geometry}
\usepackage{tikz}
\usetikzlibrary{arrows.meta}
\usepackage{enumitem} % Para listas personalizadas
\usepackage{xcolor} % Para usar cores
\usepackage{titlesec} % Para personalizar títulos

% Ajusta o espaçamento antes e depois das seções
\titleformat{\section}[block]{\normalfont\Large\bfseries}{\thesection}{1em}{}
\titlespacing*{\section}{8pt}{8pt}{8pt}

\titleformat{\subsection}[block]{\normalfont\large\bfseries}{\thesubsection}{1em}{}
\titlespacing*{\subsection}{6pt}{6pt}{6pt}

\titleformat{\subsubsection}[block]{\normalfont\normalsize\bfseries}{\thesubsubsection}{1em}{}
\titlespacing*{\subsubsection}{6pt}{6pt}{6pt}


\geometry{a4paper, left=1cm, right=1cm, top=0.5cm, bottom=1.2cm}

\setlength{\columnseprule}{0.4pt}  % Linha dividindo as colunas 
\setlength{\baselineskip}{1.0\baselineskip} % Espaçamento simples

% Definir a cor das seções como azul
\titleformat{\section}
  {\normalfont\Large\bfseries\color{blue}} % Formato do título
  {\thesection} % Número da seção
  {1em} % Espaço entre número e título
  {} % Código antes do título

\renewcommand{\thesubsection}{\textcolor{red}{\arabic{section}.\arabic{subsection}}}
\titleformat{\subsection}{\color{red}\normalfont\bfseries}{\thesubsection}{1em}{}
\title{\textcolor{blue}{Equação do 1º Grau - Entendendo e Aplicando}}
\author{Professor: Jefferson}
\date{}


\begin{document}

\maketitle
\vspace{-1cm}  % Ajuste o valor conforme necessário

\begin{center}
\large{Nome: \underline{\hspace{8cm}} \quad Série-Turma: \underline{\hspace{3cm}}}
\end{center}

\begin{multicols}{2}

\section*{O que é uma Equação do 1º Grau?}

Uma equação do 1º grau é uma equação algébrica que tem como característica principal a variável \( x \) com expoente 1. Ou seja, o máximo que a variável pode ser elevada é 1, como em \( ax + b = 0 \). Aqui, \( a \) e \( b \) são números conhecidos (chamados de coeficientes), e nosso objetivo é encontrar o valor de \( x \) que torna a equação verdadeira.

A forma geral de uma equação do 1º grau é:

\[
ax + b = 0
\]

Exemplo 1: Resolvendo uma equação simples

Considere a equação:

\[
3x + 6 = 0
\]

Para resolver, seguimos os seguintes passos:

1. \textbf{Isolando o termo com \(x\)}: Subtraímos 6 dos dois lados para começar a isolar a variável \(x\):

\[
3x = -6
\]

2. \textbf{Dividindo pelo coeficiente de \(x\)}: Agora, dividimos ambos os lados por 3, o coeficiente que acompanha \(x\):

\[
x = \frac{-6}{3} = -2
\]

Portanto, a solução da equação é \(x = -2\).

Exemplo 2: Equação com frações

Agora, vamos resolver uma equação que contém frações:

\[
\frac{x}{4} - 3 = 0
\]

1. \textbf{Isolando o termo com \(x\)}: Primeiro, somamos 3 nos dois lados para eliminar o -3:

\[
\frac{x}{4} = 3
\]

2. \textbf{Multiplicando para eliminar o denominador}: Agora, multiplicamos ambos os lados da equação por 4, para eliminar a fração:

\[
x = 12
\]

Portanto, a solução é \(x = 12\).

\section*{Resolução de Equações do 1º Grau}

Vamos aprender agora como resolver equações que envolvem diferentes operações, como multiplicação, frações e parênteses.

\subsection*{Tópico 1: Equações Simples}
Primeiro, vamos focar nas equações simples, como as que vimos acima. Essas equações têm apenas um termo com \(x\) e podem ser resolvidas com operações básicas.

\textbf{Atividade 1:} Resolva a equação:
\[
5x - 7 = 18
\]

\textbf{Solução:}

1. Somamos 7 aos dois lados para isolar o termo com \(x\):

\[
5x = 18 + 7 = 25
\]

2. Dividimos ambos os lados por 5:

\[
x = \frac{25}{5} = 5
\]

Logo, a solução da equação é \(x = 5\).

\subsection*{Tópico 2: Equações com Parênteses}
Agora, vamos aprender como resolver equações que envolvem parênteses. Para resolver, devemos primeiro aplicar a distributiva.

Exemplo:
\[
2(x - 3) = 8
\]

1. Aplicamos a distributiva:
\[
2x - 6 = 8
\]

2. Somamos 6 aos dois lados:
\[
2x = 8 + 6 = 14
\]

3. Dividimos ambos os lados por 2:
\[
x = \frac{14}{2} = 7
\]

Portanto, a solução é \(x = 7\).

\subsection*{Tópico 3: Equações com Frações}
Quando temos equações com frações, a estratégia é eliminar os denominadores multiplicando ambos os lados da equação pelo menor denominador comum.

Exemplo:
\[
\frac{x}{3} + 2 = 4
\]

1. Subtraímos 2 de ambos os lados para isolar o termo com \(x\):
\[
\frac{x}{3} = 4 - 2 = 2
\]

2. Multiplicamos ambos os lados por 3 para eliminar a fração:
\[
x = 2 . 3 = 6
\]

Logo, a solução é \(x = 6\).

\subsection*{Tópico 4: Equações com Coeficientes Negativos}
Agora vamos resolver equações em que o coeficiente de \(x\) é negativo. Essas equações exigem atenção ao sinal negativo, mas a resolução é semelhante às anteriores.

Exemplo:
\[
-4x + 8 = -12
\]

1. Subtraímos 8 de ambos os lados:
\[
-4x = -12 - 8 = -20
\]

2. Dividimos ambos os lados por -4:
\[
x = \frac{-20}{-4} = 5
\]

Portanto, a solução é \(x = 5\).

\section*{1. Atividade de Fixação: Expressões}

Agora que já vimos diversos tipos de equações do 1º grau, vamos praticar com algumas questões. Lembre-se de seguir os passos para isolar o \(x\) e encontrar a solução.

\subsection*{Questão 1. Resolva a equação:}
\[
3(x - 5) = 12
\]

\subsection*{Questão 2. Resolva a equação:}
\[
\frac{x + 1}{2} = 3
\]

\subsection*{Questão 3:Resolva a equação}
\[
-2x + 4 = 10
\]

\subsection*{Questão 4. Resolva a equação:}
\[
\frac{3x}{5} - 4 = 6
\]

\section*{2. Atividade de Fixação: Contexto}

Resolva as questões abaixo montando e resolvendo equações do 1º grau adequadas a cada situação.

% -----------------------------------------
\subsection*{Questão 1. (Análise de Planos)}
Uma empresa de transporte oferece dois planos para entregas rápidas:  
\begin{itemize}
    \item \textbf{Plano A:} Taxa fixa de R\$ 15,00 + R\$ 2,50 por quilômetro rodado
    \item \textbf{Plano B:} Taxa fixa de R\$ 30,00 + R\$ 1,80 por quilômetro rodado
\end{itemize}
Para qual distância (em km) os dois planos terão o mesmo custo?
\textbf{Dica:} Iguale os custos totais: \(15 + 2.50x = 30 + 1.80x\).


% -----------------------------------------
\subsection*{Questão 2. (Comparação de Descontos)}
Uma camisa custa R\$ 120,00 com duas opções:

\begin{itemize}[noitemsep]
    \item \textbf{Opção 1:} 20\% de desconto no total
    \item \textbf{Opção 2:} Desconto fixo de R\$ 30,00
\end{itemize}
A partir de quantas camisas compradas a \textbf{Opção 1} se torna mais vantajosa que a \textbf{Opção 2}?
\textbf{Dica:} Para \(n\) camisas: \(0.80 \times 120n = 120n - 30\).

% -----------------------------------------
\subsection*{Questão 3. (Proporcionalidade Ambiental)}
Um estudo mostra que, para cada 5 kg de papel reciclado, evita-se o corte de uma árvore. Se uma escola recolheu 120 kg de papel em um mês, quantas árvores foram preservadas?
\textbf{Dica:} Relacione a quantidade de papel coletado com a proporção dada.

% -----------------------------------------
\subsection*{Questão 5. (Geometria Aplicada)}
Um terreno triangular tem sua base medindo o triplo do seu lado. Sabendo que o perímetro do terreno é 50 metros e os dois lados do triângulo são iguais, qual é a medida do base e dos lados desse triâgulo? \textbf{Dica:} O perímetro é a soma dos lados.

% -----------------------------------------
\subsection*{Questão 6. (Economia Doméstica)}
Uma família reduzirá seu consumo de água para atingir a meta de gastar no máximo R\$ 150,00 por mês. Atualmente, eles pagam R\$ 0,50 por m³ de água e consomem 400 m³ mensais. Quantos m³ precisam reduzir para atingir a meta?

\textbf{Dica:} Equação: \(0.50(400 - x) = 150\).


% -----------------------------------------
\subsection*{Questão 7. (Escolha de Pacotes)}
\begin{itemize}[noitemsep]
    \item \textbf{Básico:} 100 MB por R\$ 80/mês
    \item \textbf{Premium:} 200 MB por R\$ 140/mês
\end{itemize}
Quantos meses são necessários para que o Pacote Premium seja mais econômico por MB que o Básico? \textbf{Dica:} Calcule o custo por MB de cada pacote.


% -----------------------------------------
\subsection*{Questão 8. (Movimento Uniforme)}
Dois trens partem de cidades distantes 600 km uma da outra. O Trem A viaja a 80 km/h, e o Trem B a 70 km/h. Em quanto tempo após a partida eles se encontrarão?

\textbf{Dica:} A distância total é a soma das distâncias percorridas por cada
trem: \(80t + 70t = 600\).



% -----------------------------------------
\subsection*{Questão 9. (Sustentabilidade)}
Uma lâmpada LED consome 10 W/h e dura 25.000 horas. Uma lâmpada incandescente
consome 60 W/h e dura 1.000 horas. Considerando que 1 kWh custa R\$ 0,80, após
quantas horas de uso o custo total (compra + energia) da LED se torna menor que
o da incandescente, sabendo que a LED custa R\$ 40,00 e a incandescente R\$ 5,00?
\textbf{Dica:} Modele o custo total de cada lâmpada em função do tempo $t$.

\subsection*{Questão 10. (Grandezas)}
Uma máquina produz 120 peças em 5 horas. Quantas peças ela produzirá em 8 horas, mantendo a mesma taxa de produção?

\textbf{Dica:} A quantidade de peças produzidas é diretamente proporcional ao tempo de produção.

\subsection*{Questão 11. (Grandezas)}
Um carro percorre uma distância fixa a uma velocidade média de 60 km/h em 4 horas. Quanto tempo levará para percorrer a mesma distância se a velocidade média aumentar para 80 km/h?

\textbf{Dica:} O tempo de viagem é inversamente proporcional à velocidade.

\end{multicols}

\end{document}

