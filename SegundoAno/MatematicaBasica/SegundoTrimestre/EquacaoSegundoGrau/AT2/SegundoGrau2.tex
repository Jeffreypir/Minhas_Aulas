\documentclass[11pt]{article}
\usepackage[utf8]{inputenc}
\usepackage[T1]{fontenc}
\usepackage{amsmath}
\usepackage{multicol}
\usepackage{geometry}
\usepackage{tikz}
\usetikzlibrary{shapes.geometric, arrows.meta, calc}
\usepackage{enumitem}
\usepackage{xcolor}
\usepackage{titlesec}
\usepackage{tcolorbox}

% Configurações de layout
\geometry{a4paper, left=1cm, right=1cm, top=0.5cm, bottom=1.2cm}
\setlength{\columnseprule}{0.4pt}
\setlength{\baselineskip}{1.0\baselineskip}

% Cores personalizadas
\definecolor{titleblue}{RGB}{0,80,150}
\definecolor{sectionred}{RGB}{180,0,0}
\definecolor{darkgreen}{RGB}{0,100,0}
\definecolor{explanationbg}{RGB}{240,248,255}

% Formatação de títulos
\titleformat{\section}{\normalfont\Large\bfseries\color{titleblue}}{\thesection}{1em}{}
\titleformat{\subsection}{\normalfont\large\bfseries\color{sectionred}}{\thesubsection}{1em}{}
\titleformat{\subsubsection}{\normalfont\normalsize\bfseries\color{darkgreen}}{\thesubsubsection}{1em}{}

\title{\textcolor{titleblue}{Atividade Avaliativa: Função do 2º Grau}}
\author{Professor: Jefferson}
\date{}

\begin{document}

\maketitle
\vspace{-1cm}

\begin{center}
    \large{\textbf{Observação:} Respostas no caderno com letra legível. \quad Série: 2º Ano. Valor: 1,0}
\end{center}

\begin{multicols}{2}

\section*{Atividade}
\begin{enumerate}

\item \textbf{Coeficientes}\\
Identifique $a$, $b$, $c$ em $f(x) = -x^2 + 5x - 6$.
\begin{tcolorbox}[colback=explanationbg,colframe=titleblue,title=Dica:]
Forma geral: $ax^2 + bx + c$.
\end{tcolorbox}

\item \textbf{Gráfico}\\
Esboce o gráfico de $f(x) = x^2 - 1$ indicando raízes e vértice.
\begin{tcolorbox}[colback=explanationbg,colframe=titleblue,title=Dica:]
Calcule os pontos notáveis primeiro.
\end{tcolorbox}

\item \textbf{Raízes}\\
Resolva $2x^2 - 8x + 6 = 0$.
\begin{tcolorbox}[colback=explanationbg,colframe=titleblue,title=Dica:]
Simplifique dividindo todos os termos por 2 primeiro.
\end{tcolorbox}

\item \textbf{Vértice}\\
Determine o vértice de $f(x) = -3x^2 + 6x - 2$.
\begin{tcolorbox}[colback=explanationbg,colframe=titleblue,title=Dica:]
Use as fórmulas $x_v = -b/2a$ e $y_v = -\Delta/4a$.
\end{tcolorbox}

\item \textbf{Valor Máximo}\\
Qual o valor máximo de $f(x) = -x^2 + 4x$?
\begin{tcolorbox}[colback=explanationbg,colframe=titleblue,title=Dica:]
Encontre o $y_v$ do vértice.
\end{tcolorbox}

\item \textbf{Discriminante}\\
Para $f(x) = x^2 + 2x + m$, qual $m$ para ter 1 raiz real?
\begin{tcolorbox}[colback=explanationbg,colframe=titleblue,title=Dica:]
$\Delta = 0 \Rightarrow b^2 - 4ac = 0$.
\end{tcolorbox}

\item \textbf{Forma Fatorada}\\
Escreva $f(x) = 3x^2 - 12x + 9$ na forma fatorada.
\begin{tcolorbox}[colback=explanationbg,colframe=titleblue,title=Dica:]
Fatore colocando 3 em evidência primeiro.
\end{tcolorbox}

\item \textbf{Construção}\\
Determine a função com vértice em $(1, -4)$ e que passa por $(0, -3)$.
\begin{tcolorbox}[colback=explanationbg,colframe=titleblue,title=Dica:]
Use a forma $f(x) = a(x - x_v)^2 + y_v$.
\end{tcolorbox}

\item \textbf{Estudo do Sinal}\\
Para $f(x) = -2x^2 + 8x - 6$, determine quando $f(x) < 0$.
\begin{tcolorbox}[colback=explanationbg,colframe=titleblue,title=Dica:]
Encontre as raízes e analise a concavidade.
\end{tcolorbox}

\item \textbf{Aplicação}\\
A área de um retângulo é dada por $A(x) = -x^2 + 10x$. Qual a área máxima?
\begin{tcolorbox}[colback=explanationbg,colframe=titleblue,title=Dica:]
O valor máximo ocorre no vértice.
\end{tcolorbox}

\section*{Desafio}

\item \textbf{Parâmetro}\\
Para $f(x) = (m+1)x^2 - 2x + 4$, determine $m$ para que a função tenha concavidade para baixo e duas raízes reais distintas.
\begin{tcolorbox}[colback=explanationbg,colframe=titleblue,title=Dica:]
Duas condições: $a < 0$ e $\Delta > 0$.
\end{tcolorbox}

\end{enumerate}
\end{multicols}

\end{document}
