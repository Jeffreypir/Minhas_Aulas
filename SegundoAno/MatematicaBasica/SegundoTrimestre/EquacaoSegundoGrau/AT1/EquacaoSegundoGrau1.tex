\documentclass[11pt]{article}
\usepackage[utf8]{inputenc}
\usepackage[T1]{fontenc}
\usepackage{amsmath}
\usepackage{multicol}
\usepackage{geometry}
\usepackage{tikz}
\usetikzlibrary{shapes.geometric, arrows.meta, calc}
\usepackage{enumitem}
\usepackage{xcolor}
\usepackage{titlesec}
\usepackage{tcolorbox}

% Configurações de layout
\geometry{a4paper, left=1cm, right=1cm, top=0.5cm, bottom=1.2cm}
\setlength{\columnseprule}{0.4pt}
\setlength{\baselineskip}{1.0\baselineskip}

% Cores personalizadas
\definecolor{titleblue}{RGB}{0,80,150}
\definecolor{sectionred}{RGB}{180,0,0}
\definecolor{darkgreen}{RGB}{0,100,0}
\definecolor{explanationbg}{RGB}{240,248,255}

% Formatação de títulos
\titleformat{\section}{\normalfont\Large\bfseries\color{titleblue}}{\thesection}{1em}{}
\titleformat{\subsection}{\normalfont\large\bfseries\color{sectionred}}{\thesubsection}{1em}{}
\titleformat{\subsubsection}{\normalfont\normalsize\bfseries\color{darkgreen}}{\thesubsubsection}{1em}{}

\title{\textcolor{titleblue}{Atividade Avaliativa 1: Função do 2º Grau}}
\author{Professor: Jefferson}
\date{}

\begin{document}

\maketitle
\vspace{-1cm}

\begin{center}
    \large{\textbf{Observação:} Respostas no caderno com letra legível. \quad Série: 2º Ano. Valor: 1,0}
\end{center}

\begin{multicols}{2}

\section*{Atividade}
\begin{enumerate}

\item \textbf{Forma Geral}\\
Identifique os coeficientes $a$, $b$ e $c$ na função $f(x) = 3x^2 - 5x + 2$.
\begin{tcolorbox}[colback=explanationbg,colframe=titleblue,title=Dica:]
A forma geral é $f(x) = ax^2 + bx + c$.
\end{tcolorbox}

\item \textbf{Concavidade}\\
Determine a concavidade de $f(x) = 4x^2 - 2x + 1$ e justifique.
\begin{tcolorbox}[colback=explanationbg,colframe=titleblue,title=Dica:]
Observe o sinal do coeficiente $a$.
\end{tcolorbox}

\item \textbf{Raízes Simples}\\
Encontre as raízes de $f(x) = x^2 - 9$.
\begin{tcolorbox}[colback=explanationbg,colframe=titleblue,title=Dica:]
Fatoração: $a^2 - b^2 = (a+b)(a-b)$.
\end{tcolorbox}

\item \textbf{Vértice}\\
Calcule o vértice de $f(x) = x^2 - 4x + 3$.
\begin{tcolorbox}[colback=explanationbg,colframe=titleblue,title=Dica:]
Use $x_v = -\frac{b}{2a}$ e $y_v = f(x_v)$.
\end{tcolorbox}

\item \textbf{Valor Mínimo}\\
Qual o valor mínimo de $f(x) = 2x^2 - 8x + 5$?
\begin{tcolorbox}[colback=explanationbg,colframe=titleblue,title=Dica:]
Calcule $y_v$ do vértice.
\end{tcolorbox}

\item \textbf{Discriminante}\\
Para $f(x) = x^2 + 4x + k$, qual valor de $k$ para ter 2 raízes reais?
\begin{tcolorbox}[colback=explanationbg,colframe=titleblue,title=Dica:]
$\Delta > 0 \Rightarrow b^2 - 4ac > 0$.
\end{tcolorbox}

\item \textbf{Fatoração}\\
Escreva na forma fatorada $f(x) = x^2 - 5x + 6$.
\begin{tcolorbox}[colback=explanationbg,colframe=titleblue,title=Dica:]
$f(x) = (x - x_1)(x - x_2)$ onde $x_1$ e $x_2$ são raízes.
\end{tcolorbox}

\item \textbf{Construção}\\
Determine a função com raízes 1 e -2 e que passa por (0,4).
\begin{tcolorbox}[colback=explanationbg,colframe=titleblue,title=Dica:]
Use $f(x) = a(x-1)(x+2)$ e substitua o ponto.
\end{tcolorbox}

\item \textbf{Estudo do Sinal}\\
Para $f(x) = x^2 - 4$, determine quando $f(x) \geq 0$.
\begin{tcolorbox}[colback=explanationbg,colframe=titleblue,title=Dica:]
Encontre as raízes e analise a concavidade.
\end{tcolorbox}

\item \textbf{Aplicação}\\
A trajetória de uma bola é dada por $h(t) = -5t^2 + 20t$. Qual a altura máxima?
\begin{tcolorbox}[colback=explanationbg,colframe=titleblue,title=Dica:]
Calcule o $y_v$ do vértice.
\end{tcolorbox}

\section*{Desafio}

\item \textbf{Parâmetro}\\
Para $f(x) = (k-2)x^2 + 3x - 1$, determine $k$ para que a parábola tenha concavidade para cima.
\begin{tcolorbox}[colback=explanationbg,colframe=titleblue,title=Dica:]
Condição: $a > 0$.
\end{tcolorbox}

\end{enumerate}
\end{multicols}

\end{document}
