\documentclass[11pt]{article}
\usepackage[utf8]{inputenc}
\usepackage[T1]{fontenc}
\usepackage{amsmath}
\usepackage{multicol}
\usepackage{geometry}
\usepackage{tikz}
\usetikzlibrary{shapes.geometric, arrows.meta, calc}
\usepackage{enumitem}
\usepackage{xcolor}
\usepackage{titlesec}
\usepackage{tcolorbox}

% Configurações de layout
\geometry{a4paper, left=1cm, right=1cm, top=0.5cm, bottom=1.2cm}
\setlength{\columnseprule}{0.4pt}
\setlength{\baselineskip}{1.0\baselineskip}

% Cores personalizadas
\definecolor{titleblue}{RGB}{0,80,150}
\definecolor{sectionred}{RGB}{180,0,0}
\definecolor{darkgreen}{RGB}{0,100,0}
\definecolor{explanationbg}{RGB}{240,248,255}

% Formatação de títulos
\titleformat{\section}{\normalfont\Large\bfseries\color{titleblue}}{\thesection}{1em}{}
\titleformat{\subsection}{\normalfont\large\bfseries\color{sectionred}}{\thesubsection}{1em}{}
\titleformat{\subsubsection}{\normalfont\normalsize\bfseries\color{darkgreen}}{\thesubsubsection}{1em}{}

\title{\textcolor{titleblue}{Atividade Avaliativa: Função do 2º Grau}}
\author{Professor: Jefferson}
\date{}

\begin{document}

\maketitle
\vspace{-1cm}

\begin{center}
    \large{\textbf{Observação:} Respostas no caderno com letra legível. \quad Série: 2 Ano. Valor: 1,0 }
\end{center}

\begin{multicols}{2}

\section*{Atividade}
\begin{enumerate}

\item \textbf{Forma Geral}\\
Identifique os coeficientes $a$, $b$ e $c$ na função $f(x) = 2x^2 - 3x + 1$.
\begin{tcolorbox}[colback=explanationbg,colframe=titleblue,title=Dica:]
A forma geral é $f(x) = ax^2 + bx + c$. Compare com a função dada.
\end{tcolorbox}

\item \textbf{Gráfico (Parábola)}\\
Qual é a concavidade da parábola definida por $f(x) = -x^2 + 4x - 3$? Faça um esboço do gráfico.
\begin{tcolorbox}[colback=explanationbg,colframe=titleblue,title=Dica:]
O coeficiente $a$ determina a concavidade: $a > 0$ (côncava para cima), $a < 0$ (côncava para baixo). 
\end{tcolorbox}

\item \textbf{Zeros da Função}\\
Encontre as raízes de $f(x) = x^2 - 5x + 6$.
\begin{tcolorbox}[colback=explanationbg,colframe=titleblue,title=Dica:]
Use a fórmula de Bhaskara: \\ $x = \frac{-b \pm \sqrt{\Delta}}{2a} \quad \Delta = b^2 - 4ac$.
\end{tcolorbox}

\item \textbf{Vértice da Parábola}\\
Determine as coordenadas do vértice de $f(x) = x^2 - 6x + 8$.
\begin{tcolorbox}[colback=explanationbg,colframe=titleblue,title=Dica:]
    Fórmulas: $x_v = -\frac{b}{2a}$ e $y_v = - \frac{\Delta}{4a}$.
\end{tcolorbox}

\item \textbf{Valor Máximo/Mínimo}\\
Qual é o valor máximo da função $f(x) = -2x^2 + 8x - 5$?
\begin{tcolorbox}[colback=explanationbg,colframe=titleblue,title=Dica:]
O vértice é o ponto de máximo (se $a < 0$) ou mínimo (se $a > 0$). Calcule $y_v$.
\end{tcolorbox}

\item \textbf{Análise do Discriminante}\\
Para $f(x) = 3x^2 - 4x + k$, determine $k$ para que a função tenha duas raízes reais distintas.
\begin{tcolorbox}[colback=explanationbg,colframe=titleblue,title=Dica:]
Condição: $\Delta > 0$ onde $\Delta = b^2 - 4ac$.
\end{tcolorbox}

\item \textbf{Forma Fatorada}\\
Escreva na forma fatorada a função $f(x) = 2x^2 - 8x + 6$.
\begin{tcolorbox}[colback=explanationbg,colframe=titleblue,title=Dica:]
$f(x) = a(x - x_1)(x - x_2)$, onde $x_1$ e $x_2$ são as raízes.
\end{tcolorbox}

\item \textbf{Construção da Função}\\
Determine a função quadrática cujas raízes são $2$ e $-3$ e que passa pelo ponto $(1, 8)$.
\begin{tcolorbox}[colback=explanationbg,colframe=titleblue,title=Dica:]
Use a forma $f(x) = a(x - x_1)(x - x_2)$ e substitua o ponto para encontrar $a$.
\end{tcolorbox}

\item \textbf{Estudo do Sinal}\\
Para $f(x) = -x^2 + x + 6$, determine os valores de $x$ que tornam $f(x) > 0$.
\begin{tcolorbox}[colback=explanationbg,colframe=titleblue,title=Dica:]
Encontre as raízes e analise o sinal da parábola (concavidade para baixo).
\end{tcolorbox}

\item \textbf{Aplicação: Área Máxima}\\
Um terreno retangular tem perímetro de $40$ metros. Determine as dimensões para área máxima.
\begin{tcolorbox}[colback=explanationbg,colframe=titleblue,title=Dica:]
Chame os lados de $x$ e $20 - x$. Área $A(x) = x(20 - x)$. Encontre o vértice.
\end{tcolorbox}

\section*{Desafio:}

\item \textbf{Função Quadrática com Parâmetro}\\
Para $f(x) = (m - 1)x^2 + 2x - 3$, determine $m$ para que a função seja quadrática e tenha valor mínimo.
\begin{tcolorbox}[colback=explanationbg,colframe=titleblue,title=Dica:]
1) $a \neq 0$ para ser quadrática; \\ 2) $a > 0$ para ter valor mínimo.
\end{tcolorbox}

\end{enumerate}
\end{multicols}

\end{document}
