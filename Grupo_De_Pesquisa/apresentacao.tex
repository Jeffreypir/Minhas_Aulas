\documentclass{beamer}
\usepackage[utf8]{inputenc}
\usepackage{ragged2e}

\usetheme{Berkeley}

\title{Cultura Oceânica e sua Relação com o Sertão}
\subtitle{Como o oceano influencia nossas vidas, mesmo longe do mar}
\author{Grupo de Pesquisa - Tecnologias e Sustentabilidade}
\institute{EREFEM Monsenhor José Kehrle}
\date{}

\begin{document}

\begin{frame}
\titlepage
\end{frame}

% Slide 1: Introdução (Estudante 1)
\begin{frame}{Introdução}
\justifying
\textbf{Estudante 1:} \\
"Olá! Somos estudantes do sertão nordestino, e hoje vamos falar sobre a Cultura Oceânica. Mesmo vivendo longe do mar, o oceano tem um impacto direto em nossas vidas. Ele influencia o clima, a economia e até a cultura da nossa região."
\end{frame}

% Slide 2: O Oceano e o Clima (Estudante 2)
\begin{frame}{O Oceano e o Clima}
\justifying
\textbf{Estudante 2:} \\
"O oceano é responsável por regular o clima do planeta. Ele controla as correntes marítimas e o ciclo da água, que trazem as chuvas tão importantes para o sertão. Sem o oceano, nosso semiárido seria ainda mais seco e difícil de viver."
\end{frame}

% Slide 3: O Oceano e o Oxigênio (Estudante 3)
\begin{frame}{O Oceano e o Oxigênio}
\justifying
\textbf{Estudante 3:} \\
"Você sabia que mais da metade do oxigênio que respiramos vem do oceano? Isso mesmo! As algas marinhas e o fitoplâncton produzem oxigênio através da fotossíntese. Então, mesmo no sertão, dependemos do mar para respirar."
\end{frame}

% Slide 4: O Oceano na Nossa Mesa (Estudante 4)
\begin{frame}{O Oceano na Nossa Mesa}
\justifying
\textbf{Estudante 4:} \\
"Além disso, muitos dos alimentos que consumimos têm uma ligação direta com o oceano. Peixes, camarões e até o sal que usamos na cozinha vêm do mar. O oceano está presente na nossa mesa, mesmo estando longe do litoral."
\end{frame}

% Slide 5: O Oceano na Nossa Cultura (Estudante 5)
\begin{frame}{O Oceano na Nossa Cultura}
\justifying
\textbf{Estudante 5:} \\
"O oceano também está presente na nossa cultura. Muitas histórias, músicas e tradições do sertão falam sobre o mar. Cantadores de repente e poetas nordestinos muitas vezes usam o oceano como inspiração, mostrando que ele faz parte da nossa identidade, mesmo à distância."
\end{frame}

% Slide 6: Desafios do Oceano (Estudante 6)
\begin{frame}{Desafios do Oceano}
\justifying
\textbf{Estudante 6:} \\
"Mas, infelizmente, o oceano está enfrentando grandes desafios. A poluição por plásticos, as mudanças climáticas e a pesca predatória estão prejudicando os ecossistemas marinhos. E isso não afeta só o litoral; aqui no sertão, sentimos os impactos no clima e na economia."
\end{frame}

% Slide 7: Nossa Responsabilidade (Estudante 7)
\begin{frame}{Nossa Responsabilidade}
\justifying
\textbf{Estudante 7:} \\
"A Cultura Oceânica nos ensina que todos somos responsáveis pelo oceano, mesmo quem vive longe dele. Precisamos cuidar do mar para garantir um futuro sustentável para o planeta. E isso começa com pequenas ações no nosso dia a dia."
\end{frame}

% Slide 8: Ações no Sertão (Estudante 8)
\begin{frame}{Ações no Sertão}
\justifying
\textbf{Estudante 8:} \\
"No sertão, sabemos que cada ação conta. Reduzir o uso de plásticos, economizar água e apoiar iniciativas sustentáveis são formas de proteger o oceano. Afinal, o que fazemos aqui reflete lá no mar."
\end{frame}

% Slide 9: O Oceano é Patrimônio de Todos (Estudante 9)
\begin{frame}{O Oceano é Patrimônio de Todos}
\justifying
\textbf{Estudante 9:} \\
"O oceano é um patrimônio de todos. Ele conecta continentes, culturas e pessoas. Cuidar dele é cuidar da vida no planeta, seja no litoral ou no sertão. Precisamos agir agora para garantir um futuro melhor para as próximas gerações."
\end{frame}

% Slide 10: Conclusão (Estudante 10)
\begin{frame}{Conclusão}
\justifying
\textbf{Estudante 10:} \\
"Vamos juntos construir uma relação mais sustentável com o oceano. Ele é essencial para a vida na Terra, e depende de nós protegê-lo. Obrigado por assistir e lembre-se: o oceano começa aqui, no sertão, com cada uma das nossas ações!"
\end{frame}

\end{document}



