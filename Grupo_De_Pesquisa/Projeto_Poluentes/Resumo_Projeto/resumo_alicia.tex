\documentclass[12pt]{article}
\usepackage[utf8]{inputenc}
\usepackage[brazil]{babel}
\usepackage{graphicx}
\usepackage{amsmath}
\usepackage{geometry}
\usepackage{indentfirst}
\usepackage{setspace}
\usepackage{titlesec}
\usepackage{hyperref}

\geometry{a4paper, margin=2.5cm}
\onehalfspacing
\titleformat{\section}{\normalfont\bfseries\Large}{\thesection}{1em}{}

\begin{document}

% Capa
\begin{titlepage}
    \centering
    {\large EREFEM Monsenhor José Kerhle }\\[2cm]
    {\Large\bfseries ANÁLISE DE ÁGUA MARINHA COM ARDUINO E SENSORES}\\[2cm]
    {\large Alicia Otacilia da Silva }\\[4cm]
    {\large Arcoverde-Pernambuco}
\end{titlepage}

% Folha de rosto
\begin{titlepage}
    \centering
    {\large Alicia Otacilia da Silva}\\[2cm]
    {\Large\bfseries ANÁLISE DE ÁGUA MARINHA COM ARDUINO E SENSORES}\\[1.5cm]
    Orientador: Jefferson Bezerra dos Santos \\[3cm]
    {\large Arcoverde -- Pernambuco}\\
    {\large 2025}
\end{titlepage}

% Resumo
\begin{abstract}
    Este projeto tem como objetivo a análise de parâmetros físico-químicos da água marinha utilizando uma plataforma Arduino acoplada a sensores específicos: TDS Meter (sólidos totais dissolvidos), HC-SR04 (distância e nível de água) e TCRT5000 (detecção de cor e turbidez). A proposta visa aplicar tecnologias acessíveis para monitoramento ambiental e conscientização sobre a qualidade dos oceanos.
\end{abstract}

% Sumário
\tableofcontents
\newpage

\section{Introdução}
A poluição dos oceanos é um dos principais desafios ambientais da atualidade. Este projeto tem como foco o desenvolvimento de um sistema de baixo custo utilizando Arduino e sensores para monitorar a qualidade da água marinha.

\section{Fundamentação Teórica}

\subsection{Poluição Marinha e Qualidade da Água}

A poluição dos oceanos é um dos maiores desafios ambientais do século XXI. Substâncias poluentes como resíduos sólidos, esgoto doméstico e produtos químicos industriais são constantemente despejados nos mares, comprometendo a qualidade da água e ameaçando os ecossistemas marinhos. Segundo o relatório da ONU sobre o Estado do Oceano (UNEP, 2021) uma grande preocupação é o destino dos microplásticos, aditivos químicos e outros produtos fragmentados, muitos dos quais são conhecidos por serem tóxicos e perigosos para a saúde humana, a vida selvagem e os ecossistemas. A velocidade com que a poluição oceânica está captando a atenção do público é encorajadora e é vital que aproveitemos este impulso para alcançar um oceano limpo, saudável e resiliente.


A qualidade da água pode ser avaliada por meio de diferentes parâmetros físico-químicos, como turbidez, sólidos totais dissolvidos (TDS), pH, temperatura e oxigênio dissolvido. Cada um desses indicadores fornece informações importantes sobre o nível de poluição, a salinidade e a presença de matéria orgânica ou contaminantes.

\subsection{Tecnologia Arduino}

O \textit{Arduino} é uma plataforma de prototipagem eletrônica de código aberto baseada em hardware e software fáceis de usar. Consiste em uma placa com microcontrolador (geralmente um ATmega328P) e um ambiente de desenvolvimento integrado (IDE) que permite escrever e carregar códigos na placa.

O uso do Arduino em experimentos científicos tem crescido devido ao seu baixo custo, flexibilidade e grande comunidade de usuários. Ele pode ser facilmente conectado a sensores diversos, atuadores e módulos de comunicação, sendo uma ferramenta ideal para projetos de automação e coleta de dados ambientais.

\subsection{Sensores Utilizados no Projeto}

Este projeto utiliza três sensores principais, que permitem medir parâmetros relevantes para a análise da água marinha:

\subsubsection{Sensor TDS Meter}

O sensor \textit{TDS Meter} (Total Dissolved Solids) mede a concentração de sólidos totais dissolvidos na água, como sais, minerais e metais. A leitura é feita em partes por milhão (ppm) e está diretamente relacionada à condutividade elétrica da água. Valores elevados podem indicar salinidade excessiva ou presença de contaminantes.

\subsubsection{Sensor HC-SR04}

O sensor \textit{HC-SR04} é um sensor ultrassônico utilizado para medir distâncias com precisão. No contexto deste projeto, ele é utilizado para medir o nível da água ou a distância entre a superfície líquida e o sensor. Funciona emitindo pulsos de som e calculando o tempo de retorno para determinar a distância.

\subsubsection{Sensor TCRT5000}

O \textit{TCRT5000} é um sensor óptico reflexivo que utiliza um LED infravermelho e um fototransistor. Ele é amplamente utilizado para detecção de linhas, obstáculos ou variações de cor e brilho. Neste projeto, é empregado para avaliar a turbidez da água, pois a presença de partículas em suspensão afeta a quantidade de luz refletida.

\subsection{Importância do Monitoramento com Baixo Custo}

A utilização de plataformas como o Arduino e sensores de baixo custo possibilita o desenvolvimento de soluções acessíveis para o monitoramento ambiental. Em regiões costeiras ou comunidades com poucos recursos, projetos como este podem fornecer dados importantes sobre a qualidade da água e apoiar políticas públicas de preservação ambiental.

Além disso, essa abordagem favorece a educação científica, promovendo o aprendizado prático de eletrônica, programação e ciências ambientais.


\section{Materiais e Métodos}
\subsection{Materiais}
\begin{itemize}
    \item Placa Arduino UNO
    \item Sensor HC-SR04
    \item Sensor TCRT5000
    \item Sensor TDS Meter
    \item Protoboard, jumpers, cabos USB
    \item Recipiente com água marinha ou equivalente para simulação.
\end{itemize}

\section{Resultados e Discussões}

Este projeto encontra-se em fase inicial, portanto os resultados apresentados são preliminares e baseados em testes de bancada com os sensores HC-SR04, TCRT5000 e TDS Meter. Espera-se que o sensor HC-SR04 permita medir a profundidade com boa precisão, enquanto o TCRT5000 será utilizado para detectar parâmetros específicos relacionados à turbidez da água. O TDS Meter, por sua vez, deverá fornecer informações sobre a concentração de sólidos dissolvidos totais.

A escolha destes sensores foi fundamentada na sua utilização em monitoramento ambiental e custo acessível, o que torna viável a implementação de uma rede de monitoramento contínuo. Contudo, algumas limitações como interferências ambientais e necessidade de calibração adequada são reconhecidas e serão tratadas nas próximas fases do projeto.

Os próximos passos incluem a calibração dos sensores em laboratório, seguido de testes em campo para validar a eficiência e confiabilidade dos dados coletados, permitindo, assim, uma análise crítica da qualidade da água marinha.



\section{Referências}
\begin{thebibliography}{99}

\bibitem{unep2021}
PROGRAMA DAS NAÇÕES UNIDAS PARA O MEIO AMBIENTE. \textit{Relatório da ONU sobre poluição plástica alerta sobre aumento da poluição nos oceanos}. 2021. Disponível em: \url{https://www.unep.org/pt-br/noticias-e-reportagens/comunicado-de-imprensa/relatorio-da-onu-sobre-poluicao-plastica-alerta-sobre}. Acesso em: 28 maio 2025.

\bibitem{banzi2014}
BANZI, Massimo; SHILOH, Michael. \textit{Getting Started with Arduino}. 3. ed. Sebastopol, CA: Maker Media, 2014.

\bibitem{monk2022}
MONK, Simon. \textit{Programming Arduino: Getting Started with Sketches}. 3. ed. New York: McGraw-Hill Education, 2022.

\bibitem{hcsr04}
ELEGOO. \textit{HC-SR04 Ultrasonic Sensor – Technical Specification}. Disponível em: \url{https://www.elegoo.com/products/hc-sr04-ultrasonic-sensor}. Acesso em: 28 maio 2025.

\bibitem{tds}
DFROBOT. \textit{Gravity: Analog TDS Sensor/Meter for Arduino – Datasheet}. 2021. Disponível em: \url{https://wiki.dfrobot.com/Gravity__Analog_TDS_Sensor___Meter_For_Arduino_SKU__SEN0244}. Acesso em: 28 maio 2025.

\bibitem{tcrt5000}
VISHAY INTERTECHNOLOGY. \textit{TCRT5000 Reflective Optical Sensor with Transistor Output – Datasheet}. Disponível em: \url{https://www.vishay.com/docs/83760/tcrt5000.pdf}. Acesso em: 28 maio 2025.

\bibitem{arduino}
ARDUINO. \textit{Arduino Documentation}. Disponível em: \url{https://docs.arduino.cc}. Acesso em: 28 maio 2025.

\end{thebibliography}
\end{document}

