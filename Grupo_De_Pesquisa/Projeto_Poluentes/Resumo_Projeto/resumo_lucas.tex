\documentclass[12pt]{article}
\usepackage[utf8]{inputenc}
\usepackage[brazil]{babel}
\usepackage{graphicx}
\usepackage{amsmath}
\usepackage{geometry}
\usepackage{indentfirst}
\usepackage{setspace}
\usepackage{titlesec}
\usepackage{hyperref}
\usepackage{tikz}
\usetikzlibrary{shapes, arrows, positioning, calc, circuits.ee.IEC}

\geometry{a4paper, margin=2.5cm}
\onehalfspacing
\titleformat{\section}{\normalfont\bfseries\Large}{\thesection}{1em}{}

\begin{document}

% Capa
\begin{titlepage}
    \centering
    {\large EREFEM Monsenhor José Kerhle }\\[2cm]
    {\Large\bfseries SIMULADOR DE POLUIÇÃO MARINHA COM ARDUINO E SENSOR TCS230}\\[2cm]
    {\large Lucas Matheus Borges Barbosa }\\[4cm]
    {\large Arcoverde-Pernambuco}
\end{titlepage}

% Folha de rosto
\begin{titlepage}
    \centering
    {\large Lucas Matheus Borges Barbosa}\\[2cm]
    {\Large\bfseries SIMULADOR DE POLUIÇÃO MARINHA COM ARDUINO E SENSOR TCS230}\\[1.5cm]
    Orientador: Jefferson Bezerra dos Santos \\[3cm]
    {\large Arcoverde -- Pernambuco}\\
    {\large 2025}
\end{titlepage}

% Resumo
\begin{abstract}
    Este projeto tem como objetivo desenvolver um simulador educacional que demonstre como a poluição afeta o ecossistema marinho, utilizando uma plataforma Arduino acoplada ao sensor de cor TCS230 e LEDs RGB. O sistema simula diferentes tipos de poluição em um ambiente controlado, permitindo visualizar os efeitos através de indicadores luminosos e dados quantitativos. A proposta visa conscientizar sobre a qualidade dos oceanos utilizando tecnologias acessíveis.
\end{abstract}

% Sumário
\tableofcontents
\newpage

\section{Introdução}
A poluição dos oceanos é um dos principais desafios ambientais da atualidade, afetando diretamente a biodiversidade marinha e a qualidade da água. Este projeto tem como foco o desenvolvimento de um simulador educacional utilizando Arduino e sensores para demonstrar os efeitos da poluição em ambientes marinhos, combinando aspectos de educação ambiental com tecnologia acessível.

\section{Fundamentação Teórica}

A poluição marinha é caracterizada pela introdução de substâncias ou energia no ambiente marinho que resultam em efeitos deletérios. Segundo o relatório da ONU sobre o Estado do Oceano (UNEP, 2021), os principais poluentes incluem:

\begin{itemize}
    \item Resíduos plásticos (micro e macro)
    \item Derramamentos de petróleo
    \item Esgoto doméstico e industrial
    \item Produtos químicos agrícolas
\end{itemize}

Estes poluentes alteram parâmetros físico-químicos da água como turbidez, pH e concentração de sólidos dissolvidos, podendo levar à eutrofização, redução de oxigênio e morte de espécies marinhas.


\subsection{Sensor TCS230}
O TCS230 é um sensor de cor que converte a intensidade luminosa em frequência, com as seguintes características:

\begin{itemize}
    \item Matriz de fotodiodos com filtros RGB
    \item Saída em onda quadrada (frequência proporcional à intensidade)
    \item Escalabilidade de saída (100\%, 20\% ou 2\%)
    \item Faixa espectral: 400-700nm
    \item Alimentação: 2.7-5.5V
\end{itemize}


\section{Materiais e Métodos}

\subsection{Materiais Atualizados}
\begin{itemize}
    \item Placa Arduino UNO R3
    \item Sensor de cor TCS230
    \item LEDs RGB (catodo comum)
    \item Resistores de 220Ω (3 unidades)
    \item Protoboard 400 pontos
    \item Jumpers macho-fêmea
    \item Recipiente transparente
    \item Materiais poluentes simulados
\end{itemize}

\subsection{Diagrama do Circuito}
\begin{figure}[ht]
\centering
\begin{tikzpicture}[circuit ee IEC]
    % Arduino
    \node[draw] (arduino) at (0,0) {Arduino UNO};
    
    % TCS230
    \node[draw] (tcs) at (-3,-2) {TCS230};
    \draw (tcs.east) -- ++(0.5,0) node[right, font=\tiny] {S0-S3};
    
    % LEDs
    \node[led] (ledR) at (3,1) {};
    \node[led] (ledG) at (3,0) {};
    \node[led] (ledB) at (3,-1) {};
    
    % Conexões
    \draw (arduino.south) |- (tcs.north);
    \draw (arduino.east) -| (ledR.west);
    \draw (arduino.east) -| (ledG.west);
    \draw (arduino.east) -| (ledB.west);
    
    % Legenda
    \node[draw, fill=white] at (0,-4) {
        \footnotesize
        \begin{tabular}{ll}
        S0-S1 & Pinos D7-D6 (escala) \\
        S2-S3 & Pinos D5-D4 (filtro) \\
        OUT & Pino D3 \\
        LEDs & Pinos D9-D11 (PWM) \\
        \end{tabular}
    };
\end{tikzpicture}
\caption{Diagrama simplificado do circuito com TCS230}
\end{figure}

\subsection{Programação}
\begin{verbatim}
// Pinos do TCS230
#define S0 7
#define S1 6
#define S2 5
#define S3 4
#define OUT 3

void setup() {
  pinMode(S0, OUTPUT);
  pinMode(S1, OUTPUT);
  // Configura escala 20%
  digitalWrite(S0, HIGH);
  digitalWrite(S1, LOW);
  
  // Configura LEDs RGB
  pinMode(9, OUTPUT); // R
  pinMode(10, OUTPUT); // G
  pinMode(11, OUTPUT); // B
}

int readColor() {
  digitalWrite(S2, LOW); digitalWrite(S3, LOW);
  return pulseIn(OUT, LOW); // Mede frequência
}

void loop() {
  int red = readColor();
  // Lógica de poluição (valores de exemplo)
  if(red > 500) { // Água poluída
    analogWrite(9, 255); // Vermelho
    analogWrite(10, 0);
  } else {
    analogWrite(9, 0);
    analogWrite(10, 255); // Verde
  }
  delay(300);
}
\end{verbatim}

\section{Resultados Esperados}

Com a implementação deste simulador, espera-se:

\begin{itemize}
    \item Detecção consistente de mudanças na qualidade da água através do sensor de cor
    \item Resposta visual imediata através dos LEDs RGB
    \item Geração de dados quantitativos sobre as variações
    \item Demonstração clara dos efeitos visuais da poluição marinha
    
\end{itemize}

\section{Discussão e Aplicações}

Este simulador oferece múltiplas possibilidades educacionais:

\begin{itemize}
    \item Demonstração prática dos efeitos da poluição
    \item Introdução à programação e eletrônica
    \item Base para projetos interdisciplinares (ciências, tecnologia, ecologia)
    \item Protótipo para sistemas de monitoramento real
    
    As limitações incluem:
    \item Escala reduzida (ambiente controlado)
    \item Sensibilidade à iluminação ambiente
    \item Necessidade de calibração precisa
\end{itemize}

Futuras melhorias podem incluir:
\begin{itemize}
    \item Adição de mais parâmetros (pH, temperatura)
    \item Interface gráfica para visualização de dados
    \item Módulo wireless para monitoramento remoto
\end{itemize}


%\section{Referências}
\begin{thebibliography}{99}
\bibitem{unep2021}
PROGRAMA DAS NAÇÕES UNIDAS PARA O MEIO AMBIENTE. \textit{Relatório da ONU sobre poluição plástica alerta sobre aumento da poluição nos oceanos}. 2021. Disponível em: \url{https://www.unep.org/pt-br/noticias-e-reportagens/comunicado-de-imprensa/relatorio-da-onu-sobre-poluicao-plastica-alerta-sobre}. Acesso em: 28 maio 2025.

\bibitem{banzi2014}
BANZI, Massimo; SHILOH, Michael. \textit{Getting Started with Arduino}. 3. ed. Sebastopol, CA: Maker Media, 2014.

\bibitem{monk2022}
MONK, Simon. \textit{Programming Arduino: Getting Started with Sketches}. 3. ed. New York: McGraw-Hill Education, 2022.

\bibitem{lastminuteengineers2021}
Last Minute Engineers. \emph{TCS230/TCS3200 Color Sensor with Arduino – Complete Guide}. Disponível em: \url{https://lastminuteengineers.com/tcs230-tcs3200-color-sensor-arduino-tutorial/}. Acesso em: 29 maio 2025.

\bibitem{mschoeffler2021}
SCHOEFFLER, Michael. \emph{Arduino Tutorial: Color Sensor TCS230 TCS3200}. 2021. Disponível em: \url{https://mschoeffler.com/2021/10/16/arduino-tutorial-color-sensor-tcs230-tcs3200/}. Acesso em: 29 maio 2025.

\end{thebibliography}

\end{document}
