\documentclass[12pt]{article}
\usepackage[utf8]{inputenc}
\usepackage[brazil]{babel}
\usepackage{graphicx}
\usepackage{amsmath}
\usepackage{geometry}
\usepackage{indentfirst}
\usepackage{setspace}
\usepackage{titlesec}
\usepackage{hyperref}

\geometry{a4paper, margin=2.5cm}
\onehalfspacing
\titleformat{\section}{\normalfont\bfseries\Large}{\thesection}{1em}{}

\begin{document}

% Capa
\begin{titlepage}
    \centering
    {\large EREFEM Monsenhor José Kerhle }\\[2cm]
    {\Large\bfseries SIMULADOR DE POLUIÇÃO MARINHA COM ARDUINO E SENSORES}\\[2cm]
    {\large Alicia Otacilia da Silva }\\[4cm]
    {\large Arcoverde-Pernambuco}
\end{titlepage}

% Folha de rosto
\begin{titlepage}
    \centering
    {\large Alicia Otacilia da Silva}\\[2cm]
    {\Large\bfseries SIMULADOR DE POLUIÇÃO MARINHA COM ARDUINO E SENSORES}\\[1.5cm]
    Orientador: Jefferson Bezerra dos Santos \\[3cm]
    {\large Arcoverde -- Pernambuco}\\
    {\large 2025}
\end{titlepage}

% Resumo
\begin{abstract}
    Este projeto tem como objetivo desenvolver um simulador educacional que demonstre como a poluição afeta o ecossistema marinho, utilizando uma plataforma Arduino acoplada a sensores de cor (TCS34725) e LEDs RGB. O sistema simula diferentes tipos de poluição (plásticos, produtos químicos) em um ambiente controlado, permitindo visualizar os efeitos através de indicadores luminosos e dados quantitativos. A proposta visa conscientizar sobre a qualidade dos oceanos e demonstrar o potencial das tecnologias acessíveis para o monitoramento ambiental.
\end{abstract}

% Sumário
\tableofcontents
\newpage

\section{Introdução}
A poluição dos oceanos é um dos principais desafios ambientais da atualidade, afetando diretamente a biodiversidade marinha e a qualidade da água. Este projeto tem como foco o desenvolvimento de um simulador educacional utilizando Arduino e sensores para demonstrar os efeitos da poluição em ambientes marinhos, combinando aspectos de educação ambiental com tecnologia acessível.

\section{Fundamentação Teórica}

\subsection{Poluição Marinha e Seus Impactos}

A poluição marinha é caracterizada pela introdução de substâncias ou energia no ambiente marinho que resultam em efeitos deletérios. Segundo o relatório da ONU sobre o Estado do Oceano (UNEP, 2021), os principais poluentes incluem:

\begin{itemize}
    \item Resíduos plásticos (micro e macro)
    \item Derramamentos de petróleo
    \item Esgoto doméstico e industrial
    \item Produtos químicos agrícolas
\end{itemize}

Estes poluentes alteram parâmetros físico-químicos da água como turbidez, pH e concentração de sólidos dissolvidos, podendo levar à eutrofização, redução de oxigênio e morte de espécies marinhas.

\subsection{Tecnologia Arduino como Ferramenta Educacional}

O Arduino é uma plataforma de prototipagem eletrônica que se tornou popular no ensino de STEM (Ciência, Tecnologia, Engenharia e Matemática) devido a:

\begin{itemize}
    \item Baixo custo
    \item Comunidade ativa
    \item Facilidade de integração com sensores
    \item Ambiente de programação acessível
\end{itemize}

No contexto educacional, permite demonstrar conceitos abstratos de poluição através de visualizações concretas.

\subsection{Sensores Utilizados no Projeto}

\subsubsection{Sensor de Cor TCS34725}

O sensor TCS34725 é capaz de detectar cores RGB (Red, Green, Blue) e intensidade luminosa, sendo ideal para identificar mudanças na coloração da água causadas por:

\begin{itemize}
    \item Turbidez (partículas em suspensão)
    \item Presença de corantes ou produtos químicos
    \item Algas ou matéria orgânica
\end{itemize}

\subsubsection{LEDs RGB como Indicadores Visuais}

Os LEDs RGB serão utilizados para representar graficamente os níveis de poluição:

\begin{itemize}
    \item Verde: Água limpa
    \item Amarelo: Poluição moderada
    \item Vermelho: Poluição crítica
\end{itemize}

\subsubsection{Sistema Integrado de Monitoramento}

A combinação sensor-atuador permite:
\begin{itemize}
    \item Detecção quantitativa de poluentes
    \item Feedback visual imediato
    \interface com o computador para análise de dados
\end{itemize}

\section{Materiais e Métodos}

\subsection{Materiais}
\begin{itemize}
    \item Placa Arduino UNO
    \item Sensor de cor TCS34725
    \item LEDs RGB (3 unidades)
    \item Protoboard e jumpers
    \item Recipiente transparente (20x30cm)
    \item Materiais para simulação de poluição:
    \begin{itemize}
        \item Fragmentos de plástico colorido
        \item Tinta não tóxica (simulando químicos)
        \item Óleo vegetal (simulando derramamento)
    \end{itemize}
    \item Cabo USB para comunicação serial
\end{itemize}

\subsection{Metodologia Experimental}

O experimento será conduzido em três etapas:

\subsubsection{Montagem do Sistema}
\begin{enumerate}
    \item Configurar o Arduino com o sensor TCS34725 posicionado dentro do recipiente com água limpa
    \item Instalar os LEDs RGB em posições estratégicas para visualização
    \item Conectar todos os componentes na protoboard
\end{enumerate}

\subsubsection{Programação}
\begin{itemize}
    \item Desenvolver código para:
    \begin{itemize}
        \item Ler valores RGB do sensor
        \item Classificar os níveis de poluição
        \item Controlar os LEDs conforme a classificação
        \item Enviar dados para o monitor serial
    \end{itemize}
    \item Calibrar o sistema com água limpa como referência
\end{itemize}

\subsubsection{Simulação de Cenários}
\begin{itemize}
    \item Cenário 1: Água limpa (controle)
    \item Cenário 2: Adição de plásticos (alteração de turbidez)
    \item Cenário 3: Adição de tinta (mudança de cor)
    \item Cenário 4: Mistura de poluentes
\end{itemize}

Para cada cenário serão registrados:
\begin{itemize}
    \item Valores RGB detectados
    \item Tempo de resposta do sistema
    \item Intensidade e cor dos LEDs
\end{itemize}

\section{Resultados Esperados}

Com a implementação deste simulador, espera-se:

\begin{itemize}
    \item Detecção consistente de mudanças na qualidade da água através do sensor de cor
    \item Resposta visual imediata através dos LEDs RGB
    \item Geração de dados quantitativos sobre as variações
    \item Demonstração clara dos efeitos visuais da poluição marinha
    
    \begin{table}[h]
    \centering
    \caption{Exemplo de saída esperada do sistema}
    \begin{tabular}{|l|l|l|l|}
    \hline
    Cenário & Valores RGB & LED & Nível de Poluição \\ \hline
    1 (Limpo) & R:50 G:150 B:200 & Verde & 0\% \\ \hline
    2 (Plástico) & R:80 G:120 B:180 & Amarelo & 40\% \\ \hline
    3 (Tinta) & R:200 G:100 B:50 & Vermelho & 80\% \\ \hline
    \end{tabular}
    \end{table}
\end{itemize}

\section{Discussão e Aplicações}

Este simulador oferece múltiplas possibilidades educacionais:

\begin{itemize}
    \item Demonstração prática dos efeitos da poluição
    \item Introdução à programação e eletrônica
    \item Base para projetos interdisciplinares (ciências, tecnologia, ecologia)
    \item Protótipo para sistemas de monitoramento real
    
    As limitações incluem:
    \item Escala reduzida (ambiente controlado)
    \item Sensibilidade à iluminação ambiente
    \item Necessidade de calibração precisa
\end{itemize}

Futuras melhorias podem incluir:
\begin{itemize}
    \item Adição de mais parâmetros (pH, temperatura)
    \item Interface gráfica para visualização de dados
    \item Módulo wireless para monitoramento remoto
\end{itemize}

\section{Referências}
\begin{thebibliography}{99}

\bibitem{unep2021}
PROGRAMA DAS NAÇÕES UNIDAS PARA O MEIO AMBIENTE. \textit{Relatório da ONU sobre poluição plástica alerta sobre aumento da poluição nos oceanos}. 2021. Disponível em: \url{https://www.unep.org/pt-br/noticias-e-reportagens/comunicado-de-imprensa/relatorio-da-onu-sobre-poluicao-plastica-alerta-sobre}. Acesso em: 28 maio 2025.

\bibitem{banzi2014}
BANZI, Massimo; SHILOH, Michael. \textit{Getting Started with Arduino}. 3. ed. Sebastopol, CA: Maker Media, 2014.

\bibitem{tcs34725}
AMS. \textit{TCS34725 Digital Color Sensor Datasheet}. 2018. Disponível em: \url{https://ams.com/documents/20143/36005/TCS3472_DS000390_3-00.pdf}. Acesso em: 28 maio 2025.

\bibitem{stem}
NATIONAL SCIENCE TEACHERS ASSOCIATION. \textit{Integrating STEM in the Classroom}. 2022.

\bibitem{marinepollution}
NATIONAL OCEANIC AND ATMOSPHERIC ADMINISTRATION. \textit{Marine Pollution Threats}. 2023. Disponível em: \url{https://www.noaa.gov/education/resource-collections/ocean-coasts/marine-pollution}. Acesso em: 28 maio 2025.

\end{thebibliography}

\section*{Anexos}
\subsection*{Diagrama do Circuito}
\begin{figure}[h]
\centering
\includegraphics[width=0.8\textwidth]{circuito.png}
\caption{Diagrama esquemático da montagem com Arduino, sensor TCS34725 e LEDs RGB}
\end{figure}

\subsection*{Código Fonte}
\begin{verbatim}
// Exemplo simplificado do código Arduino
#include <Wire.h>
#include "Adafruit_TCS34725.h"

Adafruit_TCS34725 tcs = Adafruit_TCS34725();

void setup() {
  pinMode(9, OUTPUT);  // LED Vermelho
  pinMode(10, OUTPUT); // LED Verde
  pinMode(11, OUTPUT); // LED Azul
  
  if (!tcs.begin()) {
    Serial.println("Sensor não encontrado!");
    while (1);
  }
}

void loop() {
  uint16_t r, g, b, c;
  tcs.getRawData(&r, &g, &b, &c);
  
  // Lógica de classificação
  if (r > 150) {
    setLED(255, 0, 0); // Vermelho - Poluição alta
  } else if (r > 100) {
    setLED(255, 255, 0); // Amarelo - Poluição média
  } else {
    setLED(0, 255, 0); // Verde - Água limpa
  }
  
  delay(500);
}

void setLED(int red, int green, int blue) {
  analogWrite(9, red);
  analogWrite(10, green);
  analogWrite(11, blue);
}
\end{verbatim}

\end{document}
