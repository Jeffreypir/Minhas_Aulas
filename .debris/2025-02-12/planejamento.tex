\documentclass{article}
\usepackage[utf8]{inputenc}
\usepackage{amsmath}
\usepackage{amssymb}
\usepackage{geometry}
\geometry{a4paper, margin=1.5cm}

\title{Planejamento Semanal de Aulas: Notação Científica, Medidas e Ordem de Grandeza}
\author{}
\date{}

\begin{document}

\maketitle

\section{Objetivos}
\begin{itemize}
    \item Compreender e aplicar a notação científica.
    \item Trabalhar com unidades de medida e conversões.
    \item Entender o conceito de ordem de grandeza e sua aplicação.
\end{itemize}



 \section{Aula 1: Introdução à Notação Científica}

\subsection{Duração:}
1 hora e 30 minutos.

\subsection{Conteúdo:}
\begin{enumerate}
    \item \textbf{O que é notação científica?}
    \begin{itemize}
        \item Formato: \( a \times 10^n \), onde \( 1 \leq |a| < 10 \) e \( n \) é um número inteiro.
        \item Utilizada para expressar números muito grandes ou muito pequenos.
    \end{itemize}

    \item \textbf{Exemplos:}
    \begin{itemize}
        \item Velocidade da luz: \( 3 \times 10^8 \, \text{m/s} \).
        \item Tamanho de um átomo: \( 1 \times 10^{10} \, \text{m} \).
    \end{itemize}

    \item \textbf{Conversão para notação científica:}
    \begin{itemize}
        \item Números grandes: \( 4500 = 4,5 \times 10^3 \).
        \item Números pequenos: \( 0,0023 = 2,3 \times 10^{3} \).
    \end{itemize}
\end{enumerate}

\subsection{Atividades:}
\begin{itemize}
    \item Converter números para notação científica.
    \item Resolver problemas envolvendo notação científica.
\end{itemize}

\subsection{Questões:}

1. Converta para notação científica: \\

    a) \( 12000 \)  
    b) \( 0,00045 \)  
    c) \( 567000 \)  
    d) \( 0,00000078 \)   \\

Respostas:
    a) \( 1,2 \times 10^4 \)  
    b) \( 4,5 \times 10^{4} \)  
    c) \( 5,67 \times 10^5 \)  
    d) \( 7,8 \times 10^{7} \)   \\

2. Escreva em forma decimal:
    a) \( 3,4 \times 10^3 \)  
    b) \( 2,1 \times 10^{2} \)  

   Respostas:
    a) \( 3400 \)  
    b) \( 0,021 \)  



\section{ Aula 2: Operações com Notação Científica}

\subsection{Duração:}
1 hora e 30 minutos.

\subsection{Conteúdo:}
\begin{enumerate}
    \item \textbf{Multiplicação e divisão:}
    \begin{itemize}
        \item Multiplicação: Multiplicar os coeficientes e somar os expoentes.
        \item Divisão: Dividir os coeficientes e subtrair os expoentes.
    \end{itemize}

    \item \textbf{Exemplos:}
    \begin{itemize}
        \item Multiplicação: \( (2 \times 10^3) \times (3 \times 10^2) = 6 \times 10^5 \).
        \item Divisão: \( (6 \times 10^4) \div (2 \times 10^2) = 3 \times 10^2 \).
    \end{itemize}
\end{enumerate}

\subsection{Atividades:}
\begin{itemize}
    \item Realizar operações de multiplicação e divisão com notação científica.
    \item Resolver problemas contextualizados.
\end{itemize}

\subsection{Questões:}

1. Efetue as operações: \\

    a) \( (4 \times 10^5) \times (2 \times 10^3) \)  
    b) \( (9 \times 10^7) \div (3 \times 10^2) \)   \\

   Respostas:
    a) \( 8 \times 10^8 \)  
    b) \( 3 \times 10^5 \)  

2. Calcule: \\

    a) \( (5 \times 10^{3}) \times (2 \times 10^4) \)  
    b) \( (6 \times 10^6) \div (2 \times 10^{2}) \)  

   Respostas:
    a) \( 1 \times 10^2 \)  
    b) \( 3 \times 10^8 \)   \\


 \section{Aula 3: Medidas e Unidades}

\subsection{Duração:}
1 hora e 30 minutos.

\subsection{Conteúdo:}
\begin{enumerate}
    \item \textbf{Sistema Internacional de Unidades (SI):}
    \begin{itemize}
        \item Unidades básicas: metro (m), quilograma (kg), segundo (s), etc.
        \item Prefixos: quilo (\( 10^3 \)), mili (\( 10^{3} \)), micro (\( 10^{6} \)), etc.
    \end{itemize}

    \item \textbf{Conversão de unidades:}
    \begin{itemize}
        \item Exemplo: \( 1 \, \text{km} = 1000 \, \text{m} \).
    \end{itemize}
\end{enumerate}

\subsection{Atividades:}
\begin{itemize}
    \item Converter unidades de medida.
    \item Resolver problemas envolvendo medidas.
\end{itemize}

\subsection{Questões:}
1. Converta: \\

    a) \( 2 \, \text{km} \) para metros.  
    b) \( 500 \, \text{mg} \) para gramas.   \\

   Respostas:
    a) \( 2000 \, \text{m} \)  
    b) \( 0,5 \, \text{g} \)   \\

2. Quantos segundos há em 2 horas?  
   Resposta: \( 7200 \, \text{s} \).



 \section{Aula 4: Ordem de Grandeza}

\subsection{Duração:}
1 hora e 30 minutos.

\subsection{Conteúdo:}
\begin{enumerate}
    \item \textbf{O que é ordem de grandeza?}
    \begin{itemize}
        \item Aproximação de um número para a potência de 10 mais próxima.
        \item Exemplo: \( 47 \) tem ordem de grandeza \( 10^2 \).
    \end{itemize}

    \item \textbf{Como calcular:}
    \begin{itemize}
        \item Se \( a \geq 5 \), a ordem de grandeza é \( 10^{n+1} \).
        \item Se \( a < 5 \), a ordem de grandeza é \( 10^n \).
    \end{itemize}
\end{enumerate}

\subsection{Atividades:}
\begin{itemize}
    \item Calcular a ordem de grandeza de números.
    \item Aplicar o conceito em problemas reais.
\end{itemize}

\subsection{Questões:}

1. Determine a ordem de grandeza: \\
    a) \( 320 \)  
    b) \( 0,0045 \)   \\

   Respostas:
    a) \( 10^3 \)  
    b) \( 10^{3} \)  

2. Qual é a ordem de grandeza da população mundial (aproximadamente 8 bilhões de pessoas)?  
   Resposta: \( 10^{10} \).



\section{Avaliação:}
\begin{itemize}
    \item Participação nas atividades.
    \item Resolução das questões propostas.
    \item Feedback individual e coletivo.
\end{itemize}

\section{Materiais Adicionais:}
\begin{itemize}
    \item Lista de exercícios extras.
    \item Vídeos explicativos sobre notação científica e ordem de grandeza.
    \item Simulações online de conversão de unidades.
\end{itemize}

\end{document}
